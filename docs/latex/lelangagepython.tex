%% Generated by Sphinx.
\def\sphinxdocclass{article}
\documentclass[letterpaper,10pt,english]{sphinxhowto}
\ifdefined\pdfpxdimen
   \let\sphinxpxdimen\pdfpxdimen\else\newdimen\sphinxpxdimen
\fi \sphinxpxdimen=.75bp\relax
\ifdefined\pdfimageresolution
    \pdfimageresolution= \numexpr \dimexpr1in\relax/\sphinxpxdimen\relax
\fi
%% let collapsable pdf bookmarks panel have high depth per default
\PassOptionsToPackage{bookmarksdepth=5}{hyperref}
%% turn off hyperref patch of \index as sphinx.xdy xindy module takes care of
%% suitable \hyperpage mark-up, working around hyperref-xindy incompatibility
\PassOptionsToPackage{hyperindex=false}{hyperref}
%% memoir class requires extra handling
\makeatletter\@ifclassloaded{memoir}
{\ifdefined\memhyperindexfalse\memhyperindexfalse\fi}{}\makeatother

\PassOptionsToPackage{warn}{textcomp}

\catcode`^^^^00a0\active\protected\def^^^^00a0{\leavevmode\nobreak\ }
\usepackage{cmap}
\usepackage{fontspec}
\defaultfontfeatures[\rmfamily,\sffamily,\ttfamily]{}
\usepackage{amsmath,amssymb,amstext}
\usepackage{polyglossia}
\setmainlanguage{english}



\setmainfont{FreeSerif}[
  Extension      = .otf,
  UprightFont    = *,
  ItalicFont     = *Italic,
  BoldFont       = *Bold,
  BoldItalicFont = *BoldItalic
]
\setsansfont{FreeSans}[
  Extension      = .otf,
  UprightFont    = *,
  ItalicFont     = *Oblique,
  BoldFont       = *Bold,
  BoldItalicFont = *BoldOblique,
]
\setmonofont{FreeMono}[
  Extension      = .otf,
  UprightFont    = *,
  ItalicFont     = *Oblique,
  BoldFont       = *Bold,
  BoldItalicFont = *BoldOblique,
]



\usepackage[Bjarne]{fncychap}
\usepackage{sphinx}

\fvset{fontsize=\small}
\usepackage{geometry}


% Include hyperref last.
\usepackage{hyperref}
% Fix anchor placement for figures with captions.
\usepackage{hypcap}% it must be loaded after hyperref.
% Set up styles of URL: it should be placed after hyperref.
\urlstyle{same}


\usepackage{sphinxmessages}
\setcounter{tocdepth}{2}



\title{Le langage Python}
\date{Mar 13, 2024}
\release{}
\author{Pierre Cladé}
\newcommand{\sphinxlogo}{\vbox{}}
\renewcommand{\releasename}{}
\makeindex
\begin{document}

\pagestyle{empty}
\sphinxmaketitle
\pagestyle{plain}

\pagestyle{normal}
\phantomsection\label{\detokenize{index::doc}}



\section{Feuilles de cours}
\label{\detokenize{feuilles_de_cours:feuilles-de-cours}}\label{\detokenize{feuilles_de_cours::doc}}

\subsection{Les fonctions}
\label{\detokenize{cours1_fonctions_cours:les-fonctions}}\label{\detokenize{cours1_fonctions_cours::doc}}
\sphinxAtStartPar
Les fonctions sont des sous\sphinxhyphen{}programme que l’on peut exécuter. Elles sont en particulier utilisées pour effectuer des tâches répétitives.


\subsubsection{Utiliser les fonctions}
\label{\detokenize{cours1_fonctions_cours:utiliser-les-fonctions}}
\sphinxAtStartPar
Il existe un grand nombre de fonctions déjà définies en Python. Certaines sont des fonctions natives, disponibles directement en Python (par exemple la fonction \sphinxcode{\sphinxupquote{print}}). D’autres sont dans une librairie, par exemples, les fonctions mathématiques sont dans la librairie \sphinxcode{\sphinxupquote{math}}.

\sphinxAtStartPar
Pour exécuter une fonction (on utilise aussi le mot appeler \sphinxhyphen{} call en anglais), il faut faire suivre le nom de la fonction par des parenthèses avec à l’intérieur les arguments séparés par des virgules.

\sphinxAtStartPar
Une fonction peut avoir zero argument, un ou plusieurs. Le nombre d’arguments n’est pas forcement fixé à l’avance.

\begin{sphinxVerbatim}[commandchars=\\\{\}]
\PYG{n+nb}{list}\PYG{p}{(}\PYG{p}{)} \PYG{c+c1}{\PYGZsh{} Fonction sans arguments}

\PYG{k+kn}{from} \PYG{n+nn}{math} \PYG{k+kn}{import} \PYG{n}{cos} \PYG{c+c1}{\PYGZsh{} on importe la fonction cosinus}
\PYG{n}{cos}\PYG{p}{(}\PYG{l+m+mi}{1}\PYG{p}{)} \PYG{c+c1}{\PYGZsh{} Cette fonction possède un seul argument}

\PYG{n+nb}{print}\PYG{p}{(}\PYG{l+s+s1}{\PYGZsq{}}\PYG{l+s+s1}{Bonjour}\PYG{l+s+s1}{\PYGZsq{}}\PYG{p}{,} \PYG{l+s+s1}{\PYGZsq{}}\PYG{l+s+s1}{Hello}\PYG{l+s+s1}{\PYGZsq{}}\PYG{p}{)} \PYG{c+c1}{\PYGZsh{} La fonction print peut avoir autant d\PYGZsq{}argument qu\PYGZsq{}on le souhaite}
\end{sphinxVerbatim}

\sphinxAtStartPar
Lorsqu’une fonction possède plusieurs arguments, il est important de respecter l’ordre. Lorsque l’on ne connait pas cet ordre, il faut regarder la documentation, ce qui peut se faire avec la commande \sphinxcode{\sphinxupquote{nom\_de\_la\_fonction?}} ou \sphinxcode{\sphinxupquote{help(nom\_de\_la\_fonction)}}.

\sphinxAtStartPar
Regardons par exemple la fonction date du module datetime. Cette fonction permet de manipuler des dates avec Python.

\begin{sphinxVerbatim}[commandchars=\\\{\}]
\PYG{k+kn}{from} \PYG{n+nn}{datetime} \PYG{k+kn}{import} \PYG{n}{date}
\PYG{c+c1}{\PYGZsh{}date?}
\end{sphinxVerbatim}

\sphinxAtStartPar
La documentation nous donne ;

\begin{sphinxVerbatim}[commandchars=\\\{\}]
date(year, month, day)
\end{sphinxVerbatim}

\sphinxAtStartPar
Cette fonction nécéssite donc 3 arguments (l’année, le mois et le jour).

\sphinxAtStartPar
On peut l’utiliser de la façon suivante :

\begin{sphinxVerbatim}[commandchars=\\\{\}]
\PYG{n}{bataille\PYGZus{}marignan} \PYG{o}{=} \PYG{n}{date}\PYG{p}{(}\PYG{l+m+mi}{1515}\PYG{p}{,} \PYG{l+m+mi}{9}\PYG{p}{,} \PYG{l+m+mi}{13}\PYG{p}{)}
\PYG{n+nb}{print}\PYG{p}{(}\PYG{n}{bataille\PYGZus{}marignan}\PYG{p}{)}
\end{sphinxVerbatim}

\sphinxAtStartPar
Pour lever l’ambiguité lorsqu’il y a plusieurs arguments, il est possible de nommer explicitement ceux\sphinxhyphen{}ci

\begin{sphinxVerbatim}[commandchars=\\\{\}]
\PYG{n}{bataille\PYGZus{}marignan} \PYG{o}{=} \PYG{n}{date}\PYG{p}{(}\PYG{n}{year}\PYG{o}{=}\PYG{l+m+mi}{1515}\PYG{p}{,} \PYG{n}{month}\PYG{o}{=}\PYG{l+m+mi}{9}\PYG{p}{,} \PYG{n}{day}\PYG{o}{=}\PYG{l+m+mi}{13}\PYG{p}{)}
\end{sphinxVerbatim}

\sphinxAtStartPar
Lorsque les arguments sont nommés, l’ordre n’a plus d’importance.

\begin{sphinxVerbatim}[commandchars=\\\{\}]
\PYG{k}{assert} \PYG{n}{date}\PYG{p}{(}\PYG{n}{year}\PYG{o}{=}\PYG{l+m+mi}{1515}\PYG{p}{,} \PYG{n}{month}\PYG{o}{=}\PYG{l+m+mi}{9}\PYG{p}{,} \PYG{n}{day}\PYG{o}{=}\PYG{l+m+mi}{13}\PYG{p}{)}\PYG{o}{==}\PYG{n}{date}\PYG{p}{(}\PYG{n}{day}\PYG{o}{=}\PYG{l+m+mi}{13}\PYG{p}{,} \PYG{n}{month}\PYG{o}{=}\PYG{l+m+mi}{9}\PYG{p}{,} \PYG{n}{year}\PYG{o}{=}\PYG{l+m+mi}{1515}\PYG{p}{)}
\end{sphinxVerbatim}

\sphinxAtStartPar
Attention, Python ne peut pas deviner ne nom de l’argument à partir du nom de la variable. Ceci ne fonctionnera pas :

\begin{sphinxVerbatim}[commandchars=\\\{\}]
\PYG{n}{year}\PYG{o}{=}\PYG{l+m+mi}{1515}
\PYG{n}{month}\PYG{o}{=}\PYG{l+m+mi}{9}
\PYG{n}{day}\PYG{o}{=}\PYG{l+m+mi}{13}

\PYG{n}{date}\PYG{p}{(}\PYG{n}{day}\PYG{p}{,} \PYG{n}{month}\PYG{p}{,} \PYG{n}{year}\PYG{p}{)}
\end{sphinxVerbatim}

\sphinxAtStartPar
Lorsqu’une fonction contient beaucoup d’arguments, il peut être utile de regrouper les variables dans une seule variable. Il peut s’agir soit d’une liste ou d’un n\sphinxhyphen{}uplet (tuple) auquel cas l’ordre sera important, soit d’un dictionnaire, auquel cas, les clés doivent correspondre aux noms des variables. Pour une liste on précéde l’objet d’une \sphinxcode{\sphinxupquote{*}} pour un dictionnaire de \sphinxcode{\sphinxupquote{**}}.

\begin{sphinxVerbatim}[commandchars=\\\{\}]
\PYG{n}{date\PYGZus{}tpl} \PYG{o}{=} \PYG{p}{(}\PYG{l+m+mi}{1515}\PYG{p}{,} \PYG{l+m+mi}{9}\PYG{p}{,} \PYG{l+m+mi}{13}\PYG{p}{)}
\PYG{n}{date\PYGZus{}dct} \PYG{o}{=} \PYG{p}{\PYGZob{}}\PYG{l+s+s2}{\PYGZdq{}}\PYG{l+s+s2}{year}\PYG{l+s+s2}{\PYGZdq{}}\PYG{p}{:}\PYG{l+m+mi}{1515}\PYG{p}{,} \PYG{l+s+s1}{\PYGZsq{}}\PYG{l+s+s1}{month}\PYG{l+s+s1}{\PYGZsq{}}\PYG{p}{:}\PYG{l+m+mi}{9}\PYG{p}{,} \PYG{l+s+s2}{\PYGZdq{}}\PYG{l+s+s2}{day}\PYG{l+s+s2}{\PYGZdq{}}\PYG{p}{:}\PYG{l+m+mi}{13}\PYG{p}{\PYGZcb{}}

\PYG{n+nb}{print}\PYG{p}{(}\PYG{n}{date}\PYG{p}{(}\PYG{o}{*}\PYG{n}{date\PYGZus{}tpl}\PYG{p}{)}\PYG{p}{)}
\PYG{n+nb}{print}\PYG{p}{(}\PYG{n}{date}\PYG{p}{(}\PYG{o}{*}\PYG{o}{*}\PYG{n}{date\PYGZus{}dct}\PYG{p}{)}\PYG{p}{)}
\end{sphinxVerbatim}


\subsubsection{Création d’une fonction}
\label{\detokenize{cours1_fonctions_cours:creation-d-une-fonction}}

\paragraph{Principe général}
\label{\detokenize{cours1_fonctions_cours:principe-general}}
\sphinxAtStartPar
Le mot clé \sphinxcode{\sphinxupquote{def}} est utilisé pour créer une fonction. Il doit être suivit du nom de la fonction, des arguments placés entre parenthèse. Comme toute instruction qui sera suivi d’un bloc d’instruction, il faut terminer la ligne par un \sphinxcode{\sphinxupquote{:}}. Le bloc d’instruction sera alors indenté.

\begin{sphinxVerbatim}[commandchars=\\\{\}]
\PYG{c+c1}{\PYGZsh{} Fonction sans argument}
\PYG{k}{def} \PYG{n+nf}{affiche\PYGZus{}bonjour}\PYG{p}{(}\PYG{p}{)}\PYG{p}{:}
    \PYG{n+nb}{print}\PYG{p}{(}\PYG{l+s+s1}{\PYGZsq{}}\PYG{l+s+s1}{Bonjour tout le monde}\PYG{l+s+s1}{\PYGZsq{}}\PYG{p}{)}
    \PYG{n+nb}{print}\PYG{p}{(}\PYG{l+s+s1}{\PYGZsq{}}\PYG{l+s+s1}{Hello World!}\PYG{l+s+s1}{\PYGZsq{}}\PYG{p}{)}
    
\PYG{n}{affiche\PYGZus{}bonjour}\PYG{p}{(}\PYG{p}{)}
\end{sphinxVerbatim}

\sphinxAtStartPar
Pour renvoyer une valeur, il faut utiliser l’instruction \sphinxcode{\sphinxupquote{return}}.

\begin{sphinxVerbatim}[commandchars=\\\{\}]
\PYG{k+kn}{from} \PYG{n+nn}{math} \PYG{k+kn}{import} \PYG{n}{pi}

\PYG{k}{def} \PYG{n+nf}{surface\PYGZus{}d\PYGZus{}un\PYGZus{}disque}\PYG{p}{(}\PYG{n}{r}\PYG{p}{)}\PYG{p}{:}
    \PYG{k}{return} \PYG{n}{pi}\PYG{o}{*}\PYG{n}{r}\PYG{o}{*}\PYG{o}{*}\PYG{l+m+mi}{2}

\PYG{n}{surface\PYGZus{}d\PYGZus{}un\PYGZus{}disque}\PYG{p}{(}\PYG{l+m+mi}{3}\PYG{p}{)}
\end{sphinxVerbatim}

\sphinxAtStartPar
Python quitte la fonction imédiatement après le return. Il peut y avoir plusieurs return dans une fonction. Python quitte la fonction après le premier return executé. Si il arrive à la fin de la fonction, alors il y a un return implicite. La valeur renvoyée est \sphinxcode{\sphinxupquote{None}}.

\sphinxAtStartPar
Dans cet exemple, le \sphinxcode{\sphinxupquote{print('B')}} ne sera jamais executé

\begin{sphinxVerbatim}[commandchars=\\\{\}]
\PYG{k}{def} \PYG{n+nf}{f}\PYG{p}{(}\PYG{p}{)}\PYG{p}{:}
    \PYG{n+nb}{print}\PYG{p}{(}\PYG{l+s+s1}{\PYGZsq{}}\PYG{l+s+s1}{A}\PYG{l+s+s1}{\PYGZsq{}}\PYG{p}{)}
    \PYG{k}{return}
    \PYG{n+nb}{print}\PYG{p}{(}\PYG{l+s+s1}{\PYGZsq{}}\PYG{l+s+s1}{B}\PYG{l+s+s1}{\PYGZsq{}}\PYG{p}{)}
    
\PYG{n}{f}\PYG{p}{(}\PYG{p}{)}
\end{sphinxVerbatim}

\begin{sphinxVerbatim}[commandchars=\\\{\}]
\PYG{k}{def} \PYG{n+nf}{f}\PYG{p}{(}\PYG{n}{x}\PYG{p}{)}\PYG{p}{:}
    \PYG{n}{a} \PYG{o}{=} \PYG{n}{x}\PYG{o}{*}\PYG{o}{*}\PYG{l+m+mi}{2}
    \PYG{c+c1}{\PYGZsh{} Il n\PYGZsq{}y a pas de return. Cette fonction ne sert donc à rien}
    
\PYG{n+nb}{print}\PYG{p}{(}\PYG{n}{f}\PYG{p}{(}\PYG{l+m+mi}{1}\PYG{p}{)}\PYG{p}{)}
\end{sphinxVerbatim}

\begin{sphinxVerbatim}[commandchars=\\\{\}]
\PYG{k+kn}{from} \PYG{n+nn}{math} \PYG{k+kn}{import} \PYG{n}{sin}\PYG{p}{,} \PYG{n}{cos}\PYG{p}{,} \PYG{n}{tan}

\PYG{k}{def} \PYG{n+nf}{f}\PYG{p}{(}\PYG{n}{x}\PYG{p}{,} \PYG{n}{case}\PYG{p}{)}\PYG{p}{:}
    \PYG{k}{if} \PYG{n}{case}\PYG{o}{==}\PYG{l+s+s1}{\PYGZsq{}}\PYG{l+s+s1}{A}\PYG{l+s+s1}{\PYGZsq{}}\PYG{p}{:}
        \PYG{k}{return} \PYG{n}{sin}\PYG{p}{(}\PYG{n}{x}\PYG{p}{)}
    \PYG{k}{if} \PYG{n}{case}\PYG{o}{==}\PYG{l+s+s1}{\PYGZsq{}}\PYG{l+s+s1}{B}\PYG{l+s+s1}{\PYGZsq{}}\PYG{p}{:}
        \PYG{k}{return} \PYG{n}{cos}\PYG{p}{(}\PYG{n}{x}\PYG{p}{)}
    \PYG{k}{if} \PYG{n}{case}\PYG{o}{==}\PYG{l+s+s1}{\PYGZsq{}}\PYG{l+s+s1}{C}\PYG{l+s+s1}{\PYGZsq{}}\PYG{p}{:}
        \PYG{k}{return} \PYG{n}{tan}\PYG{p}{(}\PYG{n}{x}\PYG{p}{)}
\end{sphinxVerbatim}

\sphinxAtStartPar
Une fonction peut renvoyer plusieurs valeurs. Pour cela on les sépare par des virgules. On peut récupérer les valeurs en séparant les variables par des virgules à gauche du signe \sphinxcode{\sphinxupquote{=}}.

\begin{sphinxVerbatim}[commandchars=\\\{\}]
\PYG{k+kn}{from} \PYG{n+nn}{math} \PYG{k+kn}{import} \PYG{n}{cos}\PYG{p}{,} \PYG{n}{sin}

\PYG{k}{def} \PYG{n+nf}{coordonnees}\PYG{p}{(}\PYG{n}{r}\PYG{p}{,} \PYG{n}{theta}\PYG{p}{)}\PYG{p}{:}
    \PYG{k}{return} \PYG{n}{r}\PYG{o}{*}\PYG{n}{cos}\PYG{p}{(}\PYG{n}{theta}\PYG{p}{)}\PYG{p}{,} \PYG{n}{r}\PYG{o}{*}\PYG{n}{sin}\PYG{p}{(}\PYG{n}{theta}\PYG{p}{)}

\PYG{n}{x}\PYG{p}{,} \PYG{n}{y} \PYG{o}{=} \PYG{n}{coordonnees}\PYG{p}{(}\PYG{l+m+mi}{1}\PYG{p}{,} \PYG{l+m+mi}{3}\PYG{p}{)}
\end{sphinxVerbatim}


\paragraph{Chaîne de documentation}
\label{\detokenize{cours1_fonctions_cours:chaine-de-documentation}}
\sphinxAtStartPar
Si la première ligne du bloc d’instruction d’une fonction est une chaîne de caractère litérale, alors cette chaîne sera la chaîne de documentation de la fonction. Cette chaîne correspond à la description de la fonction.

\begin{sphinxVerbatim}[commandchars=\\\{\}]
\PYG{k}{def} \PYG{n+nf}{surface\PYGZus{}d\PYGZus{}un\PYGZus{}disque}\PYG{p}{(}\PYG{n}{r}\PYG{p}{)}\PYG{p}{:}
    \PYG{l+s+s2}{\PYGZdq{}}\PYG{l+s+s2}{Calcule la surface d}\PYG{l+s+s2}{\PYGZsq{}}\PYG{l+s+s2}{un disque de rayon r}\PYG{l+s+s2}{\PYGZdq{}}
    \PYG{k}{return} \PYG{n}{pi}\PYG{o}{*}\PYG{n}{r}\PYG{o}{*}\PYG{o}{*}\PYG{l+m+mi}{2}

\PYG{n}{help}\PYG{p}{(}\PYG{n}{surface\PYGZus{}d\PYGZus{}un\PYGZus{}disque}\PYG{p}{)}
\end{sphinxVerbatim}

\sphinxAtStartPar
Lorsque la chaîne de caractère prend plusieurs lignes (ce qui est en générale le cas…), alors il est préférable d’utiliser une chaîne avec triple guillemets.

\begin{sphinxVerbatim}[commandchars=\\\{\}]
\PYG{k}{def} \PYG{n+nf}{surface\PYGZus{}d\PYGZus{}un\PYGZus{}disque}\PYG{p}{(}\PYG{n}{r}\PYG{p}{)}\PYG{p}{:}
    \PYG{l+s+sd}{\PYGZdq{}\PYGZdq{}\PYGZdq{}Calcule la surface d\PYGZsq{}un disque}
\PYG{l+s+sd}{    }
\PYG{l+s+sd}{    Utilise simplement la formule \PYGZdl{}\PYGZbs{}pi r\PYGZca{}2\PYGZdl{}}
\PYG{l+s+sd}{    }
\PYG{l+s+sd}{    Arguments :}
\PYG{l+s+sd}{        r (float) : rayon du disque}
\PYG{l+s+sd}{        }
\PYG{l+s+sd}{    Renvoie : }
\PYG{l+s+sd}{        float : le surface du disque}
\PYG{l+s+sd}{    \PYGZdq{}\PYGZdq{}\PYGZdq{}}
    \PYG{k}{return} \PYG{n}{pi}\PYG{o}{*}\PYG{n}{r}\PYG{o}{*}\PYG{o}{*}\PYG{l+m+mi}{2}

\PYG{n}{help}\PYG{p}{(}\PYG{n}{surface\PYGZus{}d\PYGZus{}un\PYGZus{}disque}\PYG{p}{)}
\end{sphinxVerbatim}

\sphinxAtStartPar
La chaîne de documentation est différente des commentaires (\sphinxcode{\sphinxupquote{\#}}): ces derniers ne sont pas disponibles aux utilsateurs, mais seulement à celui qui lira et voudra comprendre le code.


\paragraph{Variables locales}
\label{\detokenize{cours1_fonctions_cours:variables-locales}}
\sphinxAtStartPar
Une variable est locale lorsqu’elle n’est définie que à l’intérieur d’une fonction. Une variable est globale lorsque sa definition est à l’extérieure de la fonction. Il n’y a pas d’interaction entre une variable locale et une variable portant le même nom à l’extérieur de la fonction

\sphinxAtStartPar
Toutes les variables qui sont affectées dans la fonction (i.e. pour lesquels on a une ligne \sphinxcode{\sphinxupquote{variable=...}}) ainsi que les arguments de la fonction sont des variables locales.

\begin{sphinxVerbatim}[commandchars=\\\{\}]
\PYG{n}{x} \PYG{o}{=} \PYG{l+m+mi}{1}
\PYG{k}{def} \PYG{n+nf}{f}\PYG{p}{(}\PYG{p}{)}\PYG{p}{:} \PYG{c+c1}{\PYGZsh{} x est une variable globale}
    \PYG{n+nb}{print}\PYG{p}{(}\PYG{n}{x}\PYG{p}{)}
    
\PYG{n}{f}\PYG{p}{(}\PYG{p}{)}
\end{sphinxVerbatim}

\begin{sphinxVerbatim}[commandchars=\\\{\}]
\PYG{n}{x} \PYG{o}{=} \PYG{l+m+mi}{1}
\PYG{k}{def} \PYG{n+nf}{f}\PYG{p}{(}\PYG{p}{)}\PYG{p}{:} \PYG{c+c1}{\PYGZsh{} x est une variable locale}
    \PYG{n}{x} \PYG{o}{=} \PYG{l+m+mi}{3}
    \PYG{n+nb}{print}\PYG{p}{(}\PYG{n}{x}\PYG{p}{)}
    
\PYG{n+nb}{print}\PYG{p}{(}\PYG{n}{x}\PYG{p}{)}
\PYG{n}{f}\PYG{p}{(}\PYG{p}{)}
\PYG{n+nb}{print}\PYG{p}{(}\PYG{n}{x}\PYG{p}{)}
\end{sphinxVerbatim}

\sphinxAtStartPar
La valeur prise par une variable globale est la valeur au moment de l’exection de la fonction et non au moment de la définition de la fonction

\begin{sphinxVerbatim}[commandchars=\\\{\}]
\PYG{c+c1}{\PYGZsh{}\PYGZsh{}\PYGZsh{}\PYGZsh{} NE FONCTIONNE PAS \PYGZsh{}\PYGZsh{}\PYGZsh{}}
\PYG{n}{coef} \PYG{o}{=} \PYG{l+m+mi}{2}
\PYG{k}{def} \PYG{n+nf}{double}\PYG{p}{(}\PYG{n}{x}\PYG{p}{)}\PYG{p}{:}
    \PYG{k}{return} \PYG{n}{coef}\PYG{o}{*}\PYG{n}{x}

\PYG{n}{coef} \PYG{o}{=} \PYG{l+m+mi}{3}
\PYG{k}{def} \PYG{n+nf}{triple}\PYG{p}{(}\PYG{n}{x}\PYG{p}{)}\PYG{p}{:}
    \PYG{k}{return} \PYG{n}{coef}\PYG{o}{*}\PYG{n}{x}

\PYG{n+nb}{print}\PYG{p}{(}\PYG{n}{double}\PYG{p}{(}\PYG{l+m+mi}{3}\PYG{p}{)}\PYG{p}{)}
\PYG{n+nb}{print}\PYG{p}{(}\PYG{n}{triple}\PYG{p}{(}\PYG{l+m+mi}{3}\PYG{p}{)}\PYG{p}{)}
\end{sphinxVerbatim}

\begin{sphinxVerbatim}[commandchars=\\\{\}]
9
9
\end{sphinxVerbatim}


\paragraph{Arguments optionels}
\label{\detokenize{cours1_fonctions_cours:arguments-optionels}}
\sphinxAtStartPar
Il est possible de donner une valeur par défaut à un argument. Celui\sphinxhyphen{}ci sera alors optionnel. Cela se fait avec la syntaxe \sphinxcode{\sphinxupquote{def f(.., arg=default\_value)}}. Les arguments optionels doivent être définis à la fin de la liste des arguments.

\begin{sphinxVerbatim}[commandchars=\\\{\}]
\PYG{k}{def} \PYG{n+nf}{f}\PYG{p}{(}\PYG{n}{a}\PYG{p}{,} \PYG{n}{b}\PYG{o}{=}\PYG{l+m+mi}{1}\PYG{p}{)}\PYG{p}{:}
    \PYG{n+nb}{print}\PYG{p}{(}\PYG{n}{a}\PYG{o}{*}\PYG{n}{b}\PYG{o}{*}\PYG{o}{*}\PYG{l+m+mi}{2}\PYG{p}{)}

\PYG{n}{f}\PYG{p}{(}\PYG{l+m+mi}{1}\PYG{p}{)}
\PYG{n}{f}\PYG{p}{(}\PYG{l+m+mi}{1}\PYG{p}{,} \PYG{l+m+mi}{5}\PYG{p}{)}
\PYG{n}{f}\PYG{p}{(}\PYG{n}{b}\PYG{o}{=}\PYG{l+m+mi}{4}\PYG{p}{,} \PYG{n}{a}\PYG{o}{=}\PYG{l+m+mi}{1}\PYG{p}{)}
\end{sphinxVerbatim}


\paragraph{Fonction avec un nombre arbitraire d’arguments}
\label{\detokenize{cours1_fonctions_cours:fonction-avec-un-nombre-arbitraire-d-arguments}}
\sphinxAtStartPar
Il est possible de définir une fonction avec un nombre arbitraire d’arguments en utilisant la syntaxe \sphinxcode{\sphinxupquote{*arg}}. Dans ce cas, la variable arg sera un n\sphinxhyphen{}uplet (tuple) qui contiendra tous les arguments suplémentaires.

\begin{sphinxVerbatim}[commandchars=\\\{\}]
\PYG{k}{def} \PYG{n+nf}{mafonction}\PYG{p}{(}\PYG{n}{a}\PYG{p}{,} \PYG{n}{b}\PYG{p}{,} \PYG{o}{*}\PYG{n}{args}\PYG{p}{)}\PYG{p}{:}
    \PYG{n+nb}{print}\PYG{p}{(}\PYG{l+s+s1}{\PYGZsq{}}\PYG{l+s+s1}{a = }\PYG{l+s+s1}{\PYGZsq{}}\PYG{p}{,} \PYG{n}{a}\PYG{p}{)}
    \PYG{n+nb}{print}\PYG{p}{(}\PYG{l+s+s1}{\PYGZsq{}}\PYG{l+s+s1}{b = }\PYG{l+s+s1}{\PYGZsq{}}\PYG{p}{,} \PYG{n}{b}\PYG{p}{)}
    \PYG{n+nb}{print}\PYG{p}{(}\PYG{l+s+s1}{\PYGZsq{}}\PYG{l+s+s1}{args =}\PYG{l+s+s1}{\PYGZsq{}}\PYG{p}{,} \PYG{n}{args}\PYG{p}{)}
    
\PYG{n}{mafonction}\PYG{p}{(}\PYG{l+m+mi}{1}\PYG{p}{,} \PYG{l+m+mi}{2}\PYG{p}{,} \PYG{l+m+mi}{3}\PYG{p}{,} \PYG{l+m+mi}{4}\PYG{p}{)}
\end{sphinxVerbatim}

\sphinxAtStartPar
Il est aussi possible de définir une fonction avec un nombre arbitraire d’arguments nommés. Dans ce cas, il faut utiliser la syntaxe \sphinxcode{\sphinxupquote{**kwd}}. La variable \sphinxcode{\sphinxupquote{kwd}} est alors un dictionnaire dont les clés sont les noms des variables.

\begin{sphinxVerbatim}[commandchars=\\\{\}]
\PYG{k}{def} \PYG{n+nf}{mafonction}\PYG{p}{(}\PYG{n}{a}\PYG{p}{,} \PYG{n}{b}\PYG{p}{,} \PYG{o}{*}\PYG{o}{*}\PYG{n}{kwd}\PYG{p}{)}\PYG{p}{:}
    \PYG{n+nb}{print}\PYG{p}{(}\PYG{l+s+s1}{\PYGZsq{}}\PYG{l+s+s1}{a = }\PYG{l+s+s1}{\PYGZsq{}}\PYG{p}{,} \PYG{n}{a}\PYG{p}{)}
    \PYG{n+nb}{print}\PYG{p}{(}\PYG{l+s+s1}{\PYGZsq{}}\PYG{l+s+s1}{b = }\PYG{l+s+s1}{\PYGZsq{}}\PYG{p}{,} \PYG{n}{b}\PYG{p}{)}
    \PYG{n+nb}{print}\PYG{p}{(}\PYG{l+s+s1}{\PYGZsq{}}\PYG{l+s+s1}{kwd = }\PYG{l+s+s1}{\PYGZsq{}}\PYG{p}{,} \PYG{n}{kwd}\PYG{p}{)}
    
\PYG{c+c1}{\PYGZsh{} mafonction(1, 2, 3, 4) \PYGZsh{} erreur}
\PYG{n}{mafonction}\PYG{p}{(}\PYG{l+m+mi}{1}\PYG{p}{,} \PYG{l+m+mi}{2}\PYG{p}{,} \PYG{n}{alpha}\PYG{o}{=}\PYG{l+m+mi}{1}\PYG{p}{,} \PYG{n}{beta}\PYG{o}{=}\PYG{l+m+mi}{3}\PYG{p}{)}
\end{sphinxVerbatim}


\subsection{Les nombres}
\label{\detokenize{cours2_nombres_cours:les-nombres}}\label{\detokenize{cours2_nombres_cours::doc}}

\subsubsection{Les entiers}
\label{\detokenize{cours2_nombres_cours:les-entiers}}
\sphinxAtStartPar
Type \sphinxcode{\sphinxupquote{int}} en Python.

\sphinxAtStartPar
Il existe plusieurs façons d’entrer un entier sous forme litérale

\begin{sphinxVerbatim}[commandchars=\\\{\}]
\PYG{n}{a} \PYG{o}{=} \PYG{l+m+mi}{5} \PYG{c+c1}{\PYGZsh{} Décimal}
\PYG{n}{a} \PYG{o}{=} \PYG{l+m+mb}{0b1001} \PYG{c+c1}{\PYGZsh{} binaire}
\PYG{n}{a} \PYG{o}{=} \PYG{l+m+mh}{0x23} \PYG{c+c1}{\PYGZsh{} hexadécimal}
\end{sphinxVerbatim}

\sphinxAtStartPar
En Python la taille des entiers est illimitée. Par exemple:

\begin{sphinxVerbatim}[commandchars=\\\{\}]
\PYG{n+nb}{print}\PYG{p}{(}\PYG{l+m+mi}{3}\PYG{o}{*}\PYG{o}{*}\PYG{l+m+mi}{100}\PYG{p}{)}
\end{sphinxVerbatim}

\begin{sphinxVerbatim}[commandchars=\\\{\}]
515377520732011331036461129765621272702107522001
\end{sphinxVerbatim}

\sphinxAtStartPar
Attention, ce n’est plus le cas lorsque l’on utilise des librairies de calcul numérique (comme numpy ou pandas). Dans ce cas, les nombres sont enregistrés sous une taille finie. Par défaut, il s’agit de nombre enregistré avec 64 bits, en tenant compte du signe, les entiers sont alors compris entre \(-2^63\) et \(2^63\) (exclu)

\sphinxAtStartPar
Attention, lorsqu’il y a un débordement (overflow), il n’y a pas d’erreur et le comportement est inattendu.

\begin{sphinxVerbatim}[commandchars=\\\{\}]
\PYG{k+kn}{import} \PYG{n+nn}{numpy} \PYG{k}{as} \PYG{n+nn}{np}

\PYG{n}{a} \PYG{o}{=} \PYG{n}{np}\PYG{o}{.}\PYG{n}{array}\PYG{p}{(}\PYG{p}{[}\PYG{l+m+mi}{3}\PYG{p}{]}\PYG{p}{)}
\PYG{n}{a}\PYG{o}{*}\PYG{o}{*}\PYG{l+m+mi}{100}
\end{sphinxVerbatim}

\begin{sphinxVerbatim}[commandchars=\\\{\}]
array([\PYGZhy{}2984622845537545263])
\end{sphinxVerbatim}

\sphinxAtStartPar
Lorsqu’un nombre est enregistré sous un format de taille finie, il faut s’imaginer qu’il fonctionne comme une calculatrice dont ont aurait caché les premiers chiffres. Nous allons raisonner en décimal, mais dans la réalité ce sont des bits qui sont manipulés.

\sphinxAtStartPar
Si on ne regarde que les trois derniers chiffres (resp. 64 bits) alors les opérations sont faire modulo 1000 (resp. module \(2^64\)). Par exemple \(50 \times 50 = 2500 = 500\). C’est ce que l’on appel un débordement (overflow).

\sphinxAtStartPar
Les nombres négatifs sont enregistré en utilisant une astuce : regardons l’opération (modulo 1000) suivante : \(997 + 3 = 0\). Le nombre 997 est donc le nombre qui lorsqu’on lui rajoute 3 donne 0, c’est donc \(-3\). Cela explique pourquoi dans l’exemple précédent on obtient un nombre négatif.


\subsubsection{Les nombres à virgule flottante}
\label{\detokenize{cours2_nombres_cours:les-nombres-a-virgule-flottante}}
\sphinxAtStartPar
Type \sphinxcode{\sphinxupquote{float}}. Il existe plusieurs façon d’entrer un entier sous forme litérale : soit en mettant explicitement un \sphinxcode{\sphinxupquote{.}} décimal, soit en utilisant le \sphinxcode{\sphinxupquote{e}} de la notation scientifique

\begin{sphinxVerbatim}[commandchars=\\\{\}]
\PYG{n}{a} \PYG{o}{=} \PYG{l+m+mf}{1234.567}
\PYG{n}{c} \PYG{o}{=} \PYG{l+m+mf}{3e8} \PYG{c+c1}{\PYGZsh{} ou 3E8 soit 3 fois 10 à la puissance 8}
\end{sphinxVerbatim}

\sphinxAtStartPar
Attention, le comportement d’un nombre à virgule flottante est différent de celui d’un entier, même lorsqu’il représente un entier

\begin{sphinxVerbatim}[commandchars=\\\{\}]
\PYG{n}{a} \PYG{o}{=} \PYG{l+m+mi}{3}
\PYG{n}{b} \PYG{o}{=} \PYG{l+m+mf}{3.} 
\PYG{n+nb}{print}\PYG{p}{(}\PYG{n}{a}\PYG{o}{*}\PYG{o}{*}\PYG{l+m+mi}{100}\PYG{p}{)}
\PYG{n+nb}{print}\PYG{p}{(}\PYG{n}{b}\PYG{o}{*}\PYG{o}{*}\PYG{l+m+mi}{100}\PYG{p}{)}
\end{sphinxVerbatim}

\begin{sphinxVerbatim}[commandchars=\\\{\}]
515377520732011331036461129765621272702107522001
5.153775207320113e+47
\end{sphinxVerbatim}

\sphinxAtStartPar
Les nombres sont enregistrés en \sphinxhref{http://fr.wikipedia.org/wiki/IEEE\_754}{double précision}, sur 64 bits. Il sont enregistrés sous la forme \(s\times m \times 2^e\) où \(s\) est le signe (\(\pm 1\) sur un bit), \(m\) la mantisse, un nombre entre 0 et 1 sous la forme \(0.xxxxx\) avec en tout 52 bits, et \(e\) l’exposant, un nombre entier signé sur 11 bits (soit entre \sphinxhyphen{}1024 et 1023).

\sphinxAtStartPar
Attention, la précision des nombre à virgule flottante est limitée. Elle vaut \(2^{-52}\), soit environ \(10^{-16}\)

\begin{sphinxVerbatim}[commandchars=\\\{\}]
\PYG{n}{a} \PYG{o}{=} \PYG{l+m+mf}{3.14}
\PYG{n+nb}{print}\PYG{p}{(}\PYG{n}{a} \PYG{o}{==} \PYG{n}{a} \PYG{o}{+} \PYG{l+m+mf}{1E\PYGZhy{}15}\PYG{p}{)}
\PYG{n+nb}{print}\PYG{p}{(}\PYG{n}{a} \PYG{o}{==} \PYG{n}{a} \PYG{o}{+} \PYG{l+m+mf}{1E\PYGZhy{}16}\PYG{p}{)}
\end{sphinxVerbatim}

\begin{sphinxVerbatim}[commandchars=\\\{\}]
False
True
\end{sphinxVerbatim}


\subsubsection{Les nombres complexes}
\label{\detokenize{cours2_nombres_cours:les-nombres-complexes}}
\sphinxAtStartPar
Type \sphinxcode{\sphinxupquote{complex}}

\sphinxAtStartPar
Il sont toujours enregistrés sous la forme de deux nombres à virgules flottantes (partie réelle et partie imaginaire). Il faut utiliser le \sphinxcode{\sphinxupquote{J}} ou \sphinxcode{\sphinxupquote{j}} pour écrire un nombre complexe sous forme litérale

\begin{sphinxVerbatim}[commandchars=\\\{\}]
\PYG{n}{a} \PYG{o}{=} \PYG{l+m+mi}{1} \PYG{o}{+} \PYG{l+m+mi}{3}\PYG{n}{j}
\PYG{n}{a} \PYG{o}{=} \PYG{l+m+mf}{1.123}\PYG{n}{j}
\end{sphinxVerbatim}

\sphinxAtStartPar
Il faut forcement précéder le \sphinxcode{\sphinxupquote{j}} d’un nombre. Le symbole \sphinxcode{\sphinxupquote{j}} seul désignant une variable. Notons que si il est possible de placer des chiffres dans le nom d’un variable (par exemple \sphinxcode{\sphinxupquote{x1}}), il n’est pas possible de commencer une variable par un chiffre. Par exemple \sphinxcode{\sphinxupquote{j1}} pourra désigner une variable mais pas \sphinxcode{\sphinxupquote{1j}}.

\sphinxAtStartPar
On peut facilement accéder à la partie réelle et imaginaire des nombres complexe, ce sont des attributs du nombre

\begin{sphinxVerbatim}[commandchars=\\\{\}]
\PYG{n}{a} \PYG{o}{=} \PYG{l+m+mi}{1} \PYG{o}{+} \PYG{l+m+mi}{3}\PYG{n}{J}
\PYG{n+nb}{print}\PYG{p}{(}\PYG{n}{a}\PYG{o}{.}\PYG{n}{real}\PYG{p}{)}
\PYG{n+nb}{print}\PYG{p}{(}\PYG{n}{a}\PYG{o}{.}\PYG{n}{imag}\PYG{p}{)}
\end{sphinxVerbatim}

\begin{sphinxVerbatim}[commandchars=\\\{\}]
1.0
3.0
\end{sphinxVerbatim}


\subsubsection{Opérations sur le nombres}
\label{\detokenize{cours2_nombres_cours:operations-sur-le-nombres}}
\sphinxAtStartPar
Les opérations sur les nombres sont les suivantes :
\begin{itemize}
\item {} 
\sphinxAtStartPar
somme : \sphinxcode{\sphinxupquote{+}}

\item {} 
\sphinxAtStartPar
produit : \sphinxcode{\sphinxupquote{*}}

\item {} 
\sphinxAtStartPar
différence ou négation : \sphinxcode{\sphinxupquote{\sphinxhyphen{}}}

\item {} 
\sphinxAtStartPar
division : \sphinxcode{\sphinxupquote{/}}

\item {} 
\sphinxAtStartPar
division entière : \sphinxcode{\sphinxupquote{//}}

\item {} 
\sphinxAtStartPar
modulo (reste de la division euclidienne) : \sphinxcode{\sphinxupquote{\%}} (par exemple \sphinxcode{\sphinxupquote{7\%2}})

\item {} 
\sphinxAtStartPar
puissance : \sphinxcode{\sphinxupquote{**}} (par exemple \sphinxcode{\sphinxupquote{2**10}})

\end{itemize}


\subsubsection{Les booléens et comparaison}
\label{\detokenize{cours2_nombres_cours:les-booleens-et-comparaison}}
\sphinxAtStartPar
Il existe deux valeurs : \sphinxcode{\sphinxupquote{True}} et \sphinxcode{\sphinxupquote{False}} (attention à la casse).

\sphinxAtStartPar
Les comparaisons se font à l’aide des symboles \sphinxcode{\sphinxupquote{<}}, \sphinxcode{\sphinxupquote{<=}}, \sphinxcode{\sphinxupquote{==}}, \sphinxcode{\sphinxupquote{>}} et \sphinxcode{\sphinxupquote{>=}}. Pour savoir si deux valeurs sont différentes, on utilise \sphinxcode{\sphinxupquote{!=}}.

\sphinxAtStartPar
Les opérations sont par ordre de priorité : \sphinxcode{\sphinxupquote{not}}, \sphinxcode{\sphinxupquote{and}} et \sphinxcode{\sphinxupquote{or}}.

\begin{sphinxVerbatim}[commandchars=\\\{\}]
\PYG{n+nb}{print}\PYG{p}{(}\PYG{k+kc}{False} \PYG{o+ow}{and} \PYG{k+kc}{False} \PYG{o+ow}{or} \PYG{k+kc}{True}\PYG{p}{)}
\PYG{n+nb}{print}\PYG{p}{(}\PYG{k+kc}{False} \PYG{o+ow}{and} \PYG{p}{(}\PYG{k+kc}{False} \PYG{o+ow}{or} \PYG{k+kc}{True}\PYG{p}{)}\PYG{p}{)}
\end{sphinxVerbatim}

\begin{sphinxVerbatim}[commandchars=\\\{\}]
True
False
\end{sphinxVerbatim}

\sphinxAtStartPar
Les opérations \sphinxcode{\sphinxupquote{and}} et \sphinxcode{\sphinxupquote{or}} effectuent en fait un test conditionnel. L’instruction \sphinxcode{\sphinxupquote{A and B}} est interprétée comme \sphinxcode{\sphinxupquote{B if not A else A}}, de même \sphinxcode{\sphinxupquote{A or B}} équivaut à \sphinxcode{\sphinxupquote{A if A else B}}.

\begin{sphinxVerbatim}[commandchars=\\\{\}]
\PYG{k+kn}{from} \PYG{n+nn}{math} \PYG{k+kn}{import} \PYG{n}{sqrt}

\PYG{n}{x} \PYG{o}{=} \PYG{o}{\PYGZhy{}}\PYG{l+m+mi}{1}
\PYG{c+c1}{\PYGZsh{}if sqrt(x)\PYGZgt{}.2:}
\PYG{c+c1}{\PYGZsh{}        print(\PYGZsq{}Hello\PYGZsq{})}

\PYG{k}{if} \PYG{n}{x}\PYG{o}{\PYGZgt{}}\PYG{l+m+mi}{0}\PYG{p}{:}
    \PYG{k}{if} \PYG{n}{sqrt}\PYG{p}{(}\PYG{n}{x}\PYG{p}{)}\PYG{o}{\PYGZgt{}}\PYG{o}{.}\PYG{l+m+mi}{2}\PYG{p}{:}
        \PYG{n+nb}{print}\PYG{p}{(}\PYG{l+s+s1}{\PYGZsq{}}\PYG{l+s+s1}{Hello}\PYG{l+s+s1}{\PYGZsq{}}\PYG{p}{)}
        
\PYG{k}{if} \PYG{p}{(}\PYG{n}{x}\PYG{o}{\PYGZgt{}}\PYG{l+m+mi}{0}\PYG{p}{)} \PYG{o+ow}{and} \PYG{p}{(}\PYG{n}{sqrt}\PYG{p}{(}\PYG{n}{x}\PYG{p}{)}\PYG{o}{\PYGZgt{}}\PYG{o}{.}\PYG{l+m+mi}{2}\PYG{p}{)}\PYG{p}{:}
    \PYG{n+nb}{print}\PYG{p}{(}\PYG{l+s+s1}{\PYGZsq{}}\PYG{l+s+s1}{Hello}\PYG{l+s+s1}{\PYGZsq{}}\PYG{p}{)}
    
\end{sphinxVerbatim}

\begin{sphinxadmonition}{warning}{Warning:}
\sphinxAtStartPar
Les symboles \& et | sont des opérateurs binaires. Ils réalisent les opérations and et or sur les entiers bit par bit en binaire (par exemple 6 \& 5 donne 4). Il ne faut pas les utiliser pour les opérations sur des booléens. Ils ont aussi une priorité sur les comparaisons
\end{sphinxadmonition}

\begin{sphinxVerbatim}[commandchars=\\\{\}]
\PYG{c+c1}{\PYGZsh{}if (x\PYGZgt{}0) \PYGZam{} (sqrt(x)\PYGZgt{}0):}
\PYG{c+c1}{\PYGZsh{}    print(\PYGZsq{}Hello\PYGZsq{})}
\end{sphinxVerbatim}

\begin{sphinxVerbatim}[commandchars=\\\{\}]
\PYG{n}{x} \PYG{o}{=} \PYG{l+m+mi}{3}

\PYG{k}{if} \PYG{n}{x}\PYG{o}{==}\PYG{l+m+mi}{7} \PYG{o}{\PYGZam{}} \PYG{n}{x}\PYG{o}{==}\PYG{l+m+mi}{3}\PYG{p}{:}
    \PYG{n+nb}{print}\PYG{p}{(}\PYG{l+s+s1}{\PYGZsq{}}\PYG{l+s+s1}{Bonjour}\PYG{l+s+s1}{\PYGZsq{}}\PYG{p}{)}
    
\PYG{k}{if} \PYG{n}{x}\PYG{o}{==}\PYG{l+m+mi}{7} \PYG{o+ow}{and} \PYG{n}{x}\PYG{o}{==}\PYG{l+m+mi}{3}\PYG{p}{:}
    \PYG{n+nb}{print}\PYG{p}{(}\PYG{l+s+s1}{\PYGZsq{}}\PYG{l+s+s1}{Hello}\PYG{l+s+s1}{\PYGZsq{}}\PYG{p}{)}
\end{sphinxVerbatim}

\begin{sphinxVerbatim}[commandchars=\\\{\}]
Bonjour
\end{sphinxVerbatim}

\sphinxAtStartPar
Conclusion : il est préférable de toujours mettre des parenthèses….


\subsection{Les conteneurs en Python}
\label{\detokenize{cours3_conteneur_cours:les-conteneurs-en-python}}\label{\detokenize{cours3_conteneur_cours::doc}}

\subsubsection{Notions globales}
\label{\detokenize{cours3_conteneur_cours:notions-globales}}
\sphinxAtStartPar
On appelle conteneur (container) un objet ayant vocation à en contenir d’autres

\sphinxAtStartPar
Il existe plusieurs types de conteneurs :
\begin{itemize}
\item {} 
\sphinxAtStartPar
liste

\item {} 
\sphinxAtStartPar
dictionnaire

\item {} 
\sphinxAtStartPar
ensemble

\item {} 
\sphinxAtStartPar
n\sphinxhyphen{}uplet

\end{itemize}

\sphinxAtStartPar
Dans un certaine mesure, on peut aussi considérer les chaînes de caractères comme des conteneurs

\sphinxAtStartPar
Voici quelque exemples :

\begin{sphinxVerbatim}[commandchars=\\\{\}]
\PYG{n}{s} \PYG{o}{=} \PYG{l+s+s2}{\PYGZdq{}}\PYG{l+s+s2}{Bonjour}\PYG{l+s+s2}{\PYGZdq{}} \PYG{c+c1}{\PYGZsh{} chaîne de caractère}
\PYG{n}{l} \PYG{o}{=} \PYG{p}{[}\PYG{l+m+mi}{1}\PYG{p}{,} \PYG{l+m+mi}{2}\PYG{p}{,} \PYG{l+s+s2}{\PYGZdq{}}\PYG{l+s+s2}{bonjour}\PYG{l+s+s2}{\PYGZdq{}}\PYG{p}{,} \PYG{p}{[}\PYG{l+m+mi}{1}\PYG{p}{,} \PYG{l+m+mi}{2}\PYG{p}{]}\PYG{p}{]} \PYG{c+c1}{\PYGZsh{} liste}
\PYG{n}{d} \PYG{o}{=} \PYG{p}{\PYGZob{}}\PYG{l+s+s1}{\PYGZsq{}}\PYG{l+s+s1}{key1}\PYG{l+s+s1}{\PYGZsq{}}\PYG{p}{:}\PYG{l+m+mf}{123.45}\PYG{p}{,} \PYG{l+m+mi}{3}\PYG{p}{:}\PYG{l+s+s2}{\PYGZdq{}}\PYG{l+s+s2}{bonjour}\PYG{l+s+s2}{\PYGZdq{}}\PYG{p}{\PYGZcb{}} \PYG{c+c1}{\PYGZsh{} Dictionnaire}
\PYG{n}{e} \PYG{o}{=} \PYG{p}{\PYGZob{}}\PYG{l+m+mi}{1}\PYG{p}{,} \PYG{l+m+mi}{2}\PYG{p}{,} \PYG{l+m+mi}{4}\PYG{p}{\PYGZcb{}} \PYG{c+c1}{\PYGZsh{} ensemble}
\PYG{n}{t} \PYG{o}{=} \PYG{p}{(}\PYG{l+m+mi}{1}\PYG{p}{,} \PYG{l+m+mi}{2}\PYG{p}{,} \PYG{p}{[}\PYG{l+m+mi}{1}\PYG{p}{,} \PYG{l+m+mi}{2}\PYG{p}{]}\PYG{p}{)} \PYG{c+c1}{\PYGZsh{} n\PYGZhy{}uplet / tuple}
\end{sphinxVerbatim}


\paragraph{L’opérateur in}
\label{\detokenize{cours3_conteneur_cours:l-operateur-in}}
\sphinxAtStartPar
Il permet de tester si un le conteneur contient un objet donné.

\begin{sphinxVerbatim}[commandchars=\\\{\}]
\PYG{n+nb}{print}\PYG{p}{(}\PYG{l+m+mi}{1} \PYG{o+ow}{in} \PYG{n}{l}\PYG{p}{)}
\PYG{n+nb}{print}\PYG{p}{(}\PYG{l+m+mi}{3} \PYG{o+ow}{in} \PYG{n}{d}\PYG{p}{)} \PYG{c+c1}{\PYGZsh{} Pour les dictionnaires, c\PYGZsq{}est la clé}
\PYG{n+nb}{print}\PYG{p}{(}\PYG{l+s+s1}{\PYGZsq{}}\PYG{l+s+s1}{on}\PYG{l+s+s1}{\PYGZsq{}} \PYG{o+ow}{in} \PYG{n}{s}\PYG{p}{)} \PYG{c+c1}{\PYGZsh{} Pour les chaînes de caratères, n\PYGZsq{}importe quelle sous\PYGZhy{}chaîne}
\end{sphinxVerbatim}

\begin{sphinxVerbatim}[commandchars=\\\{\}]
True
True
True
\end{sphinxVerbatim}


\paragraph{Longueur}
\label{\detokenize{cours3_conteneur_cours:longueur}}
\sphinxAtStartPar
Un autre point commun partagé par de nombreux conteneurs est qu’ils possèdent une taille. C’est\sphinxhyphen{}à\sphinxhyphen{}dire qu’ils contiennent un nombre fini et connu d’éléments, et peuvent être passés en paramètre à la fonction len.

\begin{sphinxVerbatim}[commandchars=\\\{\}]
\PYG{n+nb}{print}\PYG{p}{(}\PYG{n+nb}{len}\PYG{p}{(}\PYG{n}{t}\PYG{p}{)}\PYG{p}{)}
\end{sphinxVerbatim}

\begin{sphinxVerbatim}[commandchars=\\\{\}]
3
\end{sphinxVerbatim}


\paragraph{Objet subscriptables}
\label{\detokenize{cours3_conteneur_cours:objet-subscriptables}}
\sphinxAtStartPar
Cela désigne les objets sur lesquels l’opérateur {[}{]} peut être utilisé. L’ensemble des types cités sont subscriptables, à l’exception de l’ensemble (set), qui n’implémente pas l’opération {[}{]}

\sphinxAtStartPar
Parmis les objets subscriptables, on distingues ceux qui sont indexables par un entier et ceux qui sont sliceables.

\sphinxAtStartPar
Les dictionnaires ne sont pas indexables. Les liste et les chaînes de caractères sont indexable et sliceables.

\begin{sphinxVerbatim}[commandchars=\\\{\}]
\PYG{n+nb}{print}\PYG{p}{(}\PYG{n}{d}\PYG{p}{[}\PYG{l+s+s2}{\PYGZdq{}}\PYG{l+s+s2}{key1}\PYG{l+s+s2}{\PYGZdq{}}\PYG{p}{]}\PYG{p}{)}

\PYG{n+nb}{print}\PYG{p}{(}\PYG{n}{l}\PYG{p}{[}\PYG{l+m+mi}{2}\PYG{p}{]}\PYG{p}{)}
\PYG{n+nb}{print}\PYG{p}{(}\PYG{n}{s}\PYG{p}{[}\PYG{l+m+mi}{1}\PYG{p}{]}\PYG{p}{)}
\end{sphinxVerbatim}

\begin{sphinxVerbatim}[commandchars=\\\{\}]
123.45
bonjour
o
\end{sphinxVerbatim}

\sphinxAtStartPar
Pour les objets indexables, on rappel que le premier élément est l’élément 0. Le dernier est dont n\sphinxhyphen{}1 ou n est la taille de l’objet.

\sphinxAtStartPar
Les slices permettent de récupérer une partie de l’objet initial.
La syntaxe est \sphinxcode{\sphinxupquote{{[}start:stop:step{]}}}. Si on omet step, alors le pas est de 1. Si on omet stop, alors il s’agit de n, si on omet start, il s’agit de 0.

\sphinxAtStartPar
La taille de l’objet renvoyé est (stop \sphinxhyphen{} start)//step .

\sphinxAtStartPar
ATTENTION : si on indexe avec {[}i:j{]}, alors le dernier élément est j\sphinxhyphen{}1

\begin{sphinxVerbatim}[commandchars=\\\{\}]
\PYG{n+nb}{print}\PYG{p}{(}\PYG{n}{s}\PYG{p}{[}\PYG{l+m+mi}{1}\PYG{p}{:}\PYG{l+m+mi}{4}\PYG{p}{]}\PYG{p}{)}
\PYG{n+nb}{print}\PYG{p}{(}\PYG{n}{l}\PYG{p}{[}\PYG{l+m+mi}{1}\PYG{p}{:}\PYG{l+m+mi}{2}\PYG{p}{]}\PYG{p}{)}
\PYG{n+nb}{print}\PYG{p}{(}\PYG{n}{s}\PYG{p}{[}\PYG{p}{:}\PYG{p}{:}\PYG{l+m+mi}{2}\PYG{p}{]}\PYG{p}{)}
\end{sphinxVerbatim}

\begin{sphinxVerbatim}[commandchars=\\\{\}]
onj
[2]
Bnor
\end{sphinxVerbatim}

\sphinxAtStartPar
Les indices négatifs, sont pris modulo la taille du conteneur.

\begin{sphinxVerbatim}[commandchars=\\\{\}]
\PYG{n+nb}{print}\PYG{p}{(}\PYG{n}{l}\PYG{p}{[}\PYG{p}{:}\PYG{o}{\PYGZhy{}}\PYG{l+m+mi}{1}\PYG{p}{]}\PYG{p}{)} \PYG{c+c1}{\PYGZsh{} Tous les éléments sauf le dernier}
\end{sphinxVerbatim}


\paragraph{Conteneurs modifiables}
\label{\detokenize{cours3_conteneur_cours:conteneurs-modifiables}}
\sphinxAtStartPar
Les listes, les dictionnaires et les ensembles sont modifiables. Les n\sphinxhyphen{}uplet et les chaines de caratères ne le sont pas.

\sphinxAtStartPar
Par modifiable, on entent par exemple que l’on peut rajouter, suprimer ou remplacer un élément.

\begin{sphinxVerbatim}[commandchars=\\\{\}]
\PYG{n}{liste1} \PYG{o}{=} \PYG{p}{[}\PYG{l+s+s1}{\PYGZsq{}}\PYG{l+s+s1}{Bonjour}\PYG{l+s+s1}{\PYGZsq{}}\PYG{p}{]}
\PYG{n}{liste1}\PYG{o}{.}\PYG{n}{append}\PYG{p}{(}\PYG{l+s+s1}{\PYGZsq{}}\PYG{l+s+s1}{Hello}\PYG{l+s+s1}{\PYGZsq{}}\PYG{p}{)}
\PYG{n}{liste1}\PYG{o}{.}\PYG{n}{insert}\PYG{p}{(}\PYG{l+m+mi}{1}\PYG{p}{,} \PYG{l+s+s1}{\PYGZsq{}}\PYG{l+s+s1}{Salut}\PYG{l+s+s1}{\PYGZsq{}}\PYG{p}{)}
\PYG{k}{del} \PYG{n}{liste1}\PYG{p}{[}\PYG{l+m+mi}{0}\PYG{p}{]}
\PYG{n}{liste1}\PYG{p}{[}\PYG{l+m+mi}{1}\PYG{p}{]} \PYG{o}{=} \PYG{l+s+s1}{\PYGZsq{}}\PYG{l+s+s1}{Coucou}\PYG{l+s+s1}{\PYGZsq{}}
\PYG{n}{liste1}
\end{sphinxVerbatim}

\begin{sphinxVerbatim}[commandchars=\\\{\}]
[\PYGZsq{}Salut\PYGZsq{}, \PYGZsq{}Coucou\PYGZsq{}]
\end{sphinxVerbatim}

\sphinxAtStartPar
Une liste, c’est comme un classeur, on peut rajouter ou suprimer des feuilles. Ce sera toujours le même classeur. Un objet non modifiable ne possède par la méthode append, insert. On ne peut pas faire objet{[}i{]} = qqc. C’est comme un livre, il n’est pas possible de le modifier. La seule chose que l’on peut faire, c’est imprimer un nouveau livre avec une modification.

\sphinxAtStartPar
Si vous acheter deux livres identiques, il seront toujours identiques. Si vous achetez deux classeurs identiques, leur contenu pourra être différent.

\sphinxAtStartPar
En Python, la plupart des objets sont modifiables. Les exceptions sont les nombres, les n\sphinxhyphen{}uplets et les chaines de caractères.

\sphinxAtStartPar
Attention : lorsque l’on passe un objet à une fonction, il n’est pas dupliqué. Si la fonction modifie l’objet, alors l’objet est modifié.

\begin{sphinxVerbatim}[commandchars=\\\{\}]
\PYG{c+c1}{\PYGZsh{} Dans cet exemple, il n\PYGZsq{}y a qu\PYGZsq{}une seule liste}
\PYG{k}{def} \PYG{n+nf}{f}\PYG{p}{(}\PYG{n}{c}\PYG{p}{)}\PYG{p}{:}
    \PYG{n}{c}\PYG{o}{.}\PYG{n}{append}\PYG{p}{(}\PYG{l+m+mi}{3}\PYG{p}{)}

\PYG{n}{a} \PYG{o}{=} \PYG{p}{[}\PYG{l+m+mi}{1}\PYG{p}{,} \PYG{l+m+mi}{2}\PYG{p}{]}
\PYG{n}{b} \PYG{o}{=} \PYG{n}{a}
\PYG{n}{f}\PYG{p}{(}\PYG{n}{b}\PYG{p}{)}
\PYG{n+nb}{print}\PYG{p}{(}\PYG{n}{a}\PYG{p}{)}
\end{sphinxVerbatim}

\begin{sphinxVerbatim}[commandchars=\\\{\}]
[1, 2, 3]
\end{sphinxVerbatim}


\paragraph{Conteneurs iterables}
\label{\detokenize{cours3_conteneur_cours:conteneurs-iterables}}
\sphinxAtStartPar
C’est le cas des conteneurs que l’on peut utiliser dans un boucle for. Tous les conteneurs ci dessus sont iterables.
Dans la mesure du possible, il est important de faire la boucle for directement sur l’objet, plutôt que par exemple sur ses indices.

\begin{sphinxVerbatim}[commandchars=\\\{\}]
\PYG{k}{for} \PYG{n}{lettre} \PYG{o+ow}{in} \PYG{n}{s}\PYG{p}{:}
    \PYG{n+nb}{print}\PYG{p}{(}\PYG{n}{lettre}\PYG{p}{)}
    
\PYG{k}{for} \PYG{n}{item} \PYG{o+ow}{in} \PYG{n}{l}\PYG{p}{:}
    \PYG{n+nb}{print}\PYG{p}{(}\PYG{n}{item}\PYG{p}{)}
\end{sphinxVerbatim}

\begin{sphinxVerbatim}[commandchars=\\\{\}]
B
o
n
j
o
u
r
1
2
bonjour
[1, 2]
\end{sphinxVerbatim}

\sphinxAtStartPar
Pour les dictionnaires, il est possible d’itérer sur les clés, les valeurs, ou les deux:

\begin{sphinxVerbatim}[commandchars=\\\{\}]
\PYG{k}{for} \PYG{n}{key} \PYG{o+ow}{in} \PYG{n}{d}\PYG{p}{:} \PYG{c+c1}{\PYGZsh{} On peut utiliser d.keys()}
    \PYG{n+nb}{print}\PYG{p}{(}\PYG{n}{key}\PYG{p}{)}
    
\PYG{k}{for} \PYG{n}{val} \PYG{o+ow}{in} \PYG{n}{d}\PYG{o}{.}\PYG{n}{values}\PYG{p}{(}\PYG{p}{)}\PYG{p}{:}
    \PYG{n+nb}{print}\PYG{p}{(}\PYG{n}{val}\PYG{p}{)}
    
\PYG{k}{for} \PYG{n}{key}\PYG{p}{,} \PYG{n}{val} \PYG{o+ow}{in} \PYG{n}{d}\PYG{o}{.}\PYG{n}{items}\PYG{p}{(}\PYG{p}{)}\PYG{p}{:}
    \PYG{n+nb}{print}\PYG{p}{(}\PYG{n}{key}\PYG{p}{,} \PYG{n}{val}\PYG{p}{)}
\end{sphinxVerbatim}

\begin{sphinxVerbatim}[commandchars=\\\{\}]
key1
3
123.45
bonjour
key1 123.45
3 bonjour
\end{sphinxVerbatim}

\sphinxAtStartPar
Si on souhaite parcourir une liste et avoir l’indice, il est possible d’utiliser la fonction enumerate:

\begin{sphinxVerbatim}[commandchars=\\\{\}]
\PYG{k}{for} \PYG{n}{i}\PYG{p}{,} \PYG{n}{item} \PYG{o+ow}{in} \PYG{n+nb}{enumerate}\PYG{p}{(}\PYG{n}{l}\PYG{p}{)}\PYG{p}{:}
    \PYG{n+nb}{print}\PYG{p}{(}\PYG{l+s+sa}{f}\PYG{l+s+s2}{\PYGZdq{}}\PYG{l+s+s2}{L}\PYG{l+s+s2}{\PYGZsq{}}\PYG{l+s+s2}{item numero }\PYG{l+s+si}{\PYGZob{}}\PYG{n}{i}\PYG{l+s+si}{\PYGZcb{}}\PYG{l+s+s2}{ est }\PYG{l+s+si}{\PYGZob{}}\PYG{n}{item}\PYG{l+s+si}{\PYGZcb{}}\PYG{l+s+s2}{\PYGZdq{}}\PYG{p}{)}
\end{sphinxVerbatim}

\begin{sphinxVerbatim}[commandchars=\\\{\}]
L\PYGZsq{}item numero 0 est 1
L\PYGZsq{}item numero 1 est 2
L\PYGZsq{}item numero 2 est bonjour
L\PYGZsq{}item numero 3 est [1, 2]
\end{sphinxVerbatim}

\sphinxAtStartPar
Si on souhaite parcourir deux listes en même temps, on peut utiliser la fonction zip

\begin{sphinxVerbatim}[commandchars=\\\{\}]
\PYG{n}{liste1} \PYG{o}{=} \PYG{p}{[}\PYG{l+s+s1}{\PYGZsq{}}\PYG{l+s+s1}{A}\PYG{l+s+s1}{\PYGZsq{}}\PYG{p}{,} \PYG{l+s+s1}{\PYGZsq{}}\PYG{l+s+s1}{B}\PYG{l+s+s1}{\PYGZsq{}}\PYG{p}{,} \PYG{l+s+s1}{\PYGZsq{}}\PYG{l+s+s1}{C}\PYG{l+s+s1}{\PYGZsq{}}\PYG{p}{]}
\PYG{n}{liste2} \PYG{o}{=} \PYG{p}{[}\PYG{l+m+mi}{10}\PYG{p}{,} \PYG{l+m+mi}{4}\PYG{p}{,} \PYG{l+m+mi}{24}\PYG{p}{]}
\PYG{k}{for} \PYG{n}{lettre}\PYG{p}{,} \PYG{n}{nombre} \PYG{o+ow}{in} \PYG{n+nb}{zip}\PYG{p}{(}\PYG{n}{liste1}\PYG{p}{,} \PYG{n}{liste2}\PYG{p}{)}\PYG{p}{:}
    \PYG{n+nb}{print}\PYG{p}{(}\PYG{n}{lettre}\PYG{p}{,} \PYG{n}{nombre}\PYG{p}{)}
\end{sphinxVerbatim}

\begin{sphinxVerbatim}[commandchars=\\\{\}]
A 10
B 4
C 24
\end{sphinxVerbatim}


\paragraph{List comprehension}
\label{\detokenize{cours3_conteneur_cours:list-comprehension}}
\sphinxAtStartPar
Il arrive fréquement que l’on souhaite créer un conteneur à partir d’un autre. Par exemple, on a une liste et on souhaite appliquer une fonction sur tous les élements. On souhaite filter un dictionnaire, …

\sphinxAtStartPar
Un façon simple consiste à créer un conteneur vide et ensuite le remplir au fur et à mesure :

\begin{sphinxVerbatim}[commandchars=\\\{\}]
\PYG{n}{ancienne\PYGZus{}liste} \PYG{o}{=} \PYG{p}{[}\PYG{l+m+mi}{1}\PYG{p}{,} \PYG{l+m+mi}{4}\PYG{p}{,} \PYG{l+m+mi}{6}\PYG{p}{,} \PYG{l+m+mi}{3}\PYG{p}{]}
\PYG{n}{nouvelle\PYGZus{}liste} \PYG{o}{=} \PYG{p}{[}\PYG{p}{]}
\PYG{k}{for} \PYG{n}{val} \PYG{o+ow}{in} \PYG{n}{ancienne\PYGZus{}liste}\PYG{p}{:}
    \PYG{n}{nouvelle\PYGZus{}liste}\PYG{o}{.}\PYG{n}{append}\PYG{p}{(}\PYG{n}{val}\PYG{o}{/}\PYG{l+m+mi}{2}\PYG{p}{)}
\end{sphinxVerbatim}

\sphinxAtStartPar
La technique de list comprehension permet de le faire en une seule ligne

\begin{sphinxVerbatim}[commandchars=\\\{\}]
\PYG{n}{nouvelle\PYGZus{}liset} \PYG{o}{=} \PYG{p}{[}\PYG{n}{val}\PYG{o}{*}\PYG{o}{*}\PYG{l+m+mi}{2} \PYG{k}{for} \PYG{n}{val} \PYG{o+ow}{in} \PYG{n}{ancienne\PYGZus{}liste}\PYG{p}{]}
\end{sphinxVerbatim}

\sphinxAtStartPar
Cette methode fonctionne aussi pour les dictionnaires ou les ensembles

\begin{sphinxVerbatim}[commandchars=\\\{\}]
\PYG{n}{liste2} \PYG{o}{=} \PYG{p}{[}\PYG{l+s+s2}{\PYGZdq{}}\PYG{l+s+s2}{bonjour}\PYG{l+s+s2}{\PYGZdq{}}\PYG{p}{,} \PYG{l+s+s2}{\PYGZdq{}}\PYG{l+s+s2}{hello}\PYG{l+s+s2}{\PYGZdq{}}\PYG{p}{]}
\PYG{n+nb}{print}\PYG{p}{(}\PYG{p}{\PYGZob{}}\PYG{n}{val}\PYG{p}{:}\PYG{n}{i} \PYG{k}{for} \PYG{n}{i}\PYG{p}{,} \PYG{n}{val} \PYG{o+ow}{in} \PYG{n+nb}{enumerate}\PYG{p}{(}\PYG{n}{liste2}\PYG{p}{)}\PYG{p}{\PYGZcb{}}\PYG{p}{)}

\PYG{p}{\PYGZob{}}\PYG{n}{i}\PYG{o}{*}\PYG{o}{*}\PYG{l+m+mi}{2} \PYG{k}{for} \PYG{n}{i} \PYG{o+ow}{in} \PYG{n+nb}{range}\PYG{p}{(}\PYG{o}{\PYGZhy{}}\PYG{l+m+mi}{5}\PYG{p}{,} \PYG{l+m+mi}{5}\PYG{p}{)}\PYG{p}{\PYGZcb{}}
\end{sphinxVerbatim}

\begin{sphinxVerbatim}[commandchars=\\\{\}]
\PYGZob{}\PYGZsq{}bonjour\PYGZsq{}: 0, \PYGZsq{}hello\PYGZsq{}: 1\PYGZcb{}
\end{sphinxVerbatim}

\begin{sphinxVerbatim}[commandchars=\\\{\}]
\PYGZob{}0, 1, 4, 9, 16, 25\PYGZcb{}
\end{sphinxVerbatim}

\sphinxAtStartPar
Il est possible en plus de filtrer une liste

\begin{sphinxVerbatim}[commandchars=\\\{\}]
\PYG{p}{[}\PYG{n}{i} \PYG{k}{for} \PYG{n}{i} \PYG{o+ow}{in} \PYG{n+nb}{range}\PYG{p}{(}\PYG{l+m+mi}{20}\PYG{p}{)} \PYG{k}{if} \PYG{n}{i}\PYG{o}{\PYGZpc{}}\PYG{k}{2}==0 and i\PYGZpc{}3!=1]
\end{sphinxVerbatim}

\begin{sphinxVerbatim}[commandchars=\\\{\}]
[0, 2, 6, 8, 12, 14, 18]
\end{sphinxVerbatim}


\subsubsection{Les différents types de conteneurs}
\label{\detokenize{cours3_conteneur_cours:les-differents-types-de-conteneurs}}

\paragraph{Les listes}
\label{\detokenize{cours3_conteneur_cours:les-listes}}
\sphinxAtStartPar
Pour créer une liste, on utiliser les {[}{]}. Il est aussi possible de créer une liste à partir d’un objet itérable.

\begin{sphinxVerbatim}[commandchars=\\\{\}]
\PYG{n}{l} \PYG{o}{=} \PYG{p}{[}\PYG{l+m+mi}{1}\PYG{p}{,} \PYG{l+m+mi}{2}\PYG{p}{]}
\PYG{n}{l} \PYG{o}{=} \PYG{n+nb}{list}\PYG{p}{(}\PYG{l+s+s1}{\PYGZsq{}}\PYG{l+s+s1}{Bonjour}\PYG{l+s+s1}{\PYGZsq{}}\PYG{p}{)}
\end{sphinxVerbatim}

\begin{sphinxVerbatim}[commandchars=\\\{\}]
[\PYGZsq{}B\PYGZsq{}, \PYGZsq{}o\PYGZsq{}, \PYGZsq{}n\PYGZsq{}, \PYGZsq{}j\PYGZsq{}, \PYGZsq{}o\PYGZsq{}, \PYGZsq{}u\PYGZsq{}, \PYGZsq{}r\PYGZsq{}]
\end{sphinxVerbatim}

\sphinxAtStartPar
Voici quelques méthodes et fonctions :
\begin{itemize}
\item {} 
\sphinxAtStartPar
append : rajoute un élément à la fin

\item {} 
\sphinxAtStartPar
insert : rajoute un élément à la position i

\item {} 
\sphinxAtStartPar
extend : étend la liste en rajoutant les éléments d’un autre liste

\item {} 
\sphinxAtStartPar
l1 + l2 : crée une nouvelle liste en concaténant les deux listes.

\item {} 
\sphinxAtStartPar
sort : modifie la liste en la triant (la fonction sorted renvoie une nouvelle liste)

\item {} 
\sphinxAtStartPar
index : trouve l’indice d’un élément (ou renvoie une ValueError si il n’existe pas)

\item {} 
\sphinxAtStartPar
count : compte le nombre d’élément ayant la valeur donnée en argument

\item {} 
\sphinxAtStartPar
pop : suprime et renvoie le dernier élément

\end{itemize}


\paragraph{Les dictionnaires}
\label{\detokenize{cours3_conteneur_cours:les-dictionnaires}}
\sphinxAtStartPar
Pour créer un dictionnaire :

\begin{sphinxVerbatim}[commandchars=\\\{\}]
\PYG{c+c1}{\PYGZsh{} ces trois dictionnaires sont identiques}
\PYG{n}{d} \PYG{o}{=} \PYG{p}{\PYGZob{}}\PYG{l+s+s1}{\PYGZsq{}}\PYG{l+s+s1}{key1}\PYG{l+s+s1}{\PYGZsq{}}\PYG{p}{:}\PYG{l+s+s1}{\PYGZsq{}}\PYG{l+s+s1}{Bonjour}\PYG{l+s+s1}{\PYGZsq{}}\PYG{p}{,} \PYG{l+s+s1}{\PYGZsq{}}\PYG{l+s+s1}{key2}\PYG{l+s+s1}{\PYGZsq{}}\PYG{p}{:}\PYG{l+s+s1}{\PYGZsq{}}\PYG{l+s+s1}{Hello}\PYG{l+s+s1}{\PYGZsq{}}\PYG{p}{\PYGZcb{}}
\PYG{n}{d} \PYG{o}{=} \PYG{n+nb}{dict}\PYG{p}{(}\PYG{n}{key1}\PYG{o}{=}\PYG{l+s+s2}{\PYGZdq{}}\PYG{l+s+s2}{Bonjour}\PYG{l+s+s2}{\PYGZdq{}}\PYG{p}{,} \PYG{n}{key2}\PYG{o}{=}\PYG{l+s+s2}{\PYGZdq{}}\PYG{l+s+s2}{Hello}\PYG{l+s+s2}{\PYGZdq{}}\PYG{p}{)}
\PYG{n}{l} \PYG{o}{=} \PYG{p}{[}\PYG{p}{(}\PYG{l+s+s1}{\PYGZsq{}}\PYG{l+s+s1}{key1}\PYG{l+s+s1}{\PYGZsq{}}\PYG{p}{,} \PYG{l+s+s1}{\PYGZsq{}}\PYG{l+s+s1}{Bonjour}\PYG{l+s+s1}{\PYGZsq{}}\PYG{p}{)}\PYG{p}{,} \PYG{p}{(}\PYG{l+s+s1}{\PYGZsq{}}\PYG{l+s+s1}{key2}\PYG{l+s+s1}{\PYGZsq{}}\PYG{p}{,} \PYG{l+s+s1}{\PYGZsq{}}\PYG{l+s+s1}{Hello}\PYG{l+s+s1}{\PYGZsq{}}\PYG{p}{)}\PYG{p}{]}
\PYG{n}{d} \PYG{o}{=} \PYG{n+nb}{dict}\PYG{p}{(}\PYG{n}{l}\PYG{p}{)}
\end{sphinxVerbatim}

\sphinxAtStartPar
Quelques méthodes:
\begin{itemize}
\item {} 
\sphinxAtStartPar
keys, values, items : renvoie une ‘liste’ (en fait ce n’est pas vraiment une liste) sur laquelle on peut faire une boucle for (voir ci\sphinxhyphen{}dessus)

\item {} 
\sphinxAtStartPar
get : récupère une clé, l’intérêt est la possibilité d’utiliser une valeur par défaut si la clé n’existe pas

\item {} 
\sphinxAtStartPar
setdefault : défini une valeur si celle ci n’existe pas

\item {} 
\sphinxAtStartPar
update : modifie le dictionnaire à partir d’un nouveau dictionnaire. Il n’est pas possible de faire un ‘+’ ente deux dictionnaires

\end{itemize}

\sphinxAtStartPar
Remarques sur les clés : souvent les clés sont des chaînes de caractères, mais ce n’est pas obligatoire. On peut utiliser n’importe quel objet non modifiable ne contenant pas d’objet modifiable: nombre, chaine de caractère ou tuple contenant des objets non modifiables.


\paragraph{Les ensembles}
\label{\detokenize{cours3_conteneur_cours:les-ensembles}}
\sphinxAtStartPar
Correspond à la notion mathématique. Il ne peuvent contenir deux objets identiques. Ils sont rarement utilisés, mais pratique lorsque l’on en a besoin. On peut aussi créer un ensemble à partir de n’importe quel objet itérable

\begin{sphinxVerbatim}[commandchars=\\\{\}]
\PYG{n}{s} \PYG{o}{=} \PYG{p}{\PYGZob{}}\PYG{l+m+mi}{1}\PYG{p}{,} \PYG{l+m+mi}{5}\PYG{p}{\PYGZcb{}}
\PYG{n}{s} \PYG{o}{=} \PYG{n+nb}{set}\PYG{p}{(}\PYG{n+nb}{range}\PYG{p}{(}\PYG{l+m+mi}{3}\PYG{p}{)}\PYG{p}{)}
\end{sphinxVerbatim}

\sphinxAtStartPar
Opérations sur les ensembles :
\begin{itemize}
\item {} 
\sphinxAtStartPar
\sphinxcode{\sphinxupquote{\&}} : intersection

\item {} 
\sphinxAtStartPar
\sphinxcode{\sphinxupquote{|}} : union

\item {} 
\sphinxAtStartPar
\sphinxcode{\sphinxupquote{\sphinxhyphen{}}} : difference

\item {} 
\sphinxAtStartPar
\sphinxcode{\sphinxupquote{\textasciicircum{}}} : difference symétrique

\end{itemize}

\begin{sphinxVerbatim}[commandchars=\\\{\}]
\PYG{n}{s1} \PYG{o}{=} \PYG{p}{\PYGZob{}}\PYG{l+m+mi}{1}\PYG{p}{,} \PYG{l+m+mi}{2}\PYG{p}{,} \PYG{l+m+mi}{3}\PYG{p}{,} \PYG{l+m+mi}{4}\PYG{p}{\PYGZcb{}}
\PYG{n}{s2} \PYG{o}{=} \PYG{p}{\PYGZob{}}\PYG{l+m+mi}{2}\PYG{p}{,} \PYG{l+m+mi}{3}\PYG{p}{,} \PYG{l+m+mi}{4}\PYG{p}{,} \PYG{l+m+mi}{5}\PYG{p}{\PYGZcb{}}

\PYG{n+nb}{print}\PYG{p}{(}\PYG{n}{s1} \PYG{o}{\PYGZam{}} \PYG{n}{s2}\PYG{p}{)}
\PYG{n+nb}{print}\PYG{p}{(}\PYG{n}{s1} \PYG{o}{|} \PYG{n}{s2}\PYG{p}{)}
\PYG{n+nb}{print}\PYG{p}{(}\PYG{n}{s1} \PYG{o}{\PYGZhy{}} \PYG{n}{s2}\PYG{p}{)}
\PYG{n+nb}{print}\PYG{p}{(}\PYG{n}{s1} \PYG{o}{\PYGZca{}} \PYG{n}{s2}\PYG{p}{)}
\end{sphinxVerbatim}

\begin{sphinxVerbatim}[commandchars=\\\{\}]
\PYGZob{}2, 3, 4\PYGZcb{}
\PYGZob{}1, 2, 3, 4, 5\PYGZcb{}
\PYGZob{}1\PYGZcb{}
\PYGZob{}1, 5\PYGZcb{}
\end{sphinxVerbatim}


\paragraph{Les n\sphinxhyphen{}uplets}
\label{\detokenize{cours3_conteneur_cours:les-n-uplets}}
\sphinxAtStartPar
Similaires aux listes, il ne sont pas modofiables, ce qui permet de les utiliser comme clé dans un dictionnaires. Les n\sphinxhyphen{}uplets (tuple en anglais) sont aussi utilisé lorsqu’un fonction renvoie plusieurs éléments.

\sphinxAtStartPar
Il sont créé avec des (). Attention aux cas particulier du 1\sphinxhyphen{}uplet

\begin{sphinxVerbatim}[commandchars=\\\{\}]
\PYG{n}{t} \PYG{o}{=} \PYG{p}{(}\PYG{p}{)} 
\PYG{n}{t} \PYG{o}{=} \PYG{p}{(}\PYG{l+m+mi}{1}\PYG{p}{,}\PYG{p}{)} \PYG{c+c1}{\PYGZsh{} (1) n\PYGZsq{}est pas un 1\PYGZhy{}uplet, mais juste le nombre 1}
\PYG{n}{t} \PYG{o}{=} \PYG{p}{(}\PYG{l+m+mi}{1}\PYG{p}{,} \PYG{l+m+mi}{2}\PYG{p}{,} \PYG{l+m+mi}{56}\PYG{p}{)}
\end{sphinxVerbatim}

\sphinxAtStartPar
Seules les méthode count et index existent (et font la même chose que pour une liste).
Le \sphinxcode{\sphinxupquote{+}} permet la concaténation


\subsection{Chaînes de caractères}
\label{\detokenize{cours4_chaine_caractere_cours:chaines-de-caracteres}}\label{\detokenize{cours4_chaine_caractere_cours::doc}}

\subsubsection{Création d’une chaine}
\label{\detokenize{cours4_chaine_caractere_cours:creation-d-une-chaine}}
\sphinxAtStartPar
On peut les créer avec des \sphinxcode{\sphinxupquote{'}} ou \sphinxcode{\sphinxupquote{"}}.  Ces caractères servent à délimiter les
début et la fin du texte de la chaîne de caractère. Les guillemets
simples \sphinxcode{\sphinxupquote{'}} et doubles \sphinxcode{\sphinxupquote{"}} sont équivalents. On pourra choisir l’un ou
l’autre. Il sera cependant judicieux, si une chaîne de caractère doit contenir
un de ces guillemets, d’utiliser l’autre pour le début et la fin de la chaîne.

\begin{sphinxVerbatim}[commandchars=\\\{\}]
\PYG{n}{s} \PYG{o}{=} \PYG{l+s+s2}{\PYGZdq{}}\PYG{l+s+s2}{Bonjour}\PYG{l+s+s2}{\PYGZdq{}}
\PYG{n}{s} \PYG{o}{=} \PYG{l+s+s1}{\PYGZsq{}}\PYG{l+s+s1}{Bonjour}\PYG{l+s+s1}{\PYGZsq{}}
\PYG{n}{s} \PYG{o}{=} \PYG{l+s+s2}{\PYGZdq{}}\PYG{l+s+s2}{Aujourd}\PYG{l+s+s2}{\PYGZsq{}}\PYG{l+s+s2}{hui}\PYG{l+s+s2}{\PYGZdq{}}
\end{sphinxVerbatim}

\sphinxAtStartPar
Pour créer une chaîne de caractère sur plus d’une ligne on utilise \sphinxcode{\sphinxupquote{'''}} ou \sphinxcode{\sphinxupquote{"""}}

\begin{sphinxVerbatim}[commandchars=\\\{\}]
\PYG{n}{s} \PYG{o}{=} \PYG{l+s+s2}{\PYGZdq{}\PYGZdq{}\PYGZdq{}}\PYG{l+s+s2}{Bonjour, }
\PYG{l+s+s2}{Comment allez\PYGZhy{}vous ?}\PYG{l+s+s2}{\PYGZdq{}\PYGZdq{}\PYGZdq{}}
\end{sphinxVerbatim}

\sphinxAtStartPar
Les \sphinxstylestrong{caractères spéciaux} sont les caractères qui ne sont pas affichables et en tant que tel.
Par exemple, il existe un caractère pour le retour à la ligne. Il est possible
d’utiliser ce caractère dans une chaîne en utilisant \sphinxcode{\sphinxupquote{\textbackslash{}n}}. L’antislash sert
ici de caractère d’échappement pour indiquer que l’on va entrer un caractère
spécial. La lettre \sphinxcode{\sphinxupquote{n}} indique ici qu’il s’agit d’un retour à la ligne.

\sphinxAtStartPar
Dans les exemples suivants, le retour à la ligne est un caractère. On peut le créer en utilisant \sphinxcode{\sphinxupquote{\textbackslash{}n}}.

\begin{sphinxVerbatim}[commandchars=\\\{\}]
\PYG{n}{s} \PYG{o}{=} \PYG{l+s+s2}{\PYGZdq{}\PYGZdq{}\PYGZdq{}}\PYG{l+s+s2}{a}
\PYG{l+s+s2}{b}\PYG{l+s+s2}{\PYGZdq{}\PYGZdq{}\PYGZdq{}}
\PYG{n+nb}{print}\PYG{p}{(}\PYG{n+nb}{len}\PYG{p}{(}\PYG{n}{s}\PYG{p}{)}\PYG{p}{)}
\PYG{n}{s2} \PYG{o}{=} \PYG{l+s+s2}{\PYGZdq{}}\PYG{l+s+s2}{a}\PYG{l+s+se}{\PYGZbs{}n}\PYG{l+s+s2}{b}\PYG{l+s+s2}{\PYGZdq{}}
\PYG{k}{assert} \PYG{n}{s}\PYG{o}{==}\PYG{n}{s2}
\end{sphinxVerbatim}

\begin{sphinxVerbatim}[commandchars=\\\{\}]
3
\end{sphinxVerbatim}

\sphinxAtStartPar
L’antislash sert aussi à insérer un guillemet dans une chaîne :

\begin{sphinxVerbatim}[commandchars=\\\{\}]
\PYG{n}{s} \PYG{o}{=} \PYG{l+s+s1}{\PYGZsq{}}\PYG{l+s+s1}{Aujourd}\PYG{l+s+se}{\PYGZbs{}\PYGZsq{}}\PYG{l+s+s1}{hui}\PYG{l+s+s1}{\PYGZsq{}}
\end{sphinxVerbatim}


\subsubsection{Manipulation des chaînes de caractères}
\label{\detokenize{cours4_chaine_caractere_cours:manipulation-des-chaines-de-caracteres}}
\sphinxAtStartPar
Comme tout conteneur indexable, il est possible d’accéder à chaque caractère d’une chaîne ou à une partie d’une
chaîne. La longueur de la chaîne s’obtient avec la fonction \sphinxcode{\sphinxupquote{len}}. On peut aussi faire une boucle \sphinxcode{\sphinxupquote{for}} sur chacun des éléments de la chaîne.

\begin{sphinxVerbatim}[commandchars=\\\{\}]
\PYG{n}{s} \PYG{o}{=} \PYG{l+s+s2}{\PYGZdq{}}\PYG{l+s+s2}{Pierre}\PYG{l+s+s2}{\PYGZdq{}}
\PYG{n+nb}{print}\PYG{p}{(}\PYG{n}{s}\PYG{p}{[}\PYG{l+m+mi}{0}\PYG{p}{]}\PYG{p}{)}
\PYG{n+nb}{print}\PYG{p}{(}\PYG{n}{s}\PYG{p}{[}\PYG{l+m+mi}{2}\PYG{p}{:}\PYG{l+m+mi}{4}\PYG{p}{]}\PYG{p}{)}
\end{sphinxVerbatim}

\begin{sphinxVerbatim}[commandchars=\\\{\}]
P
er
\end{sphinxVerbatim}

\sphinxAtStartPar
Cependant, il n’est pas possible de modifier une chaîne de caractères (l’opération \sphinxcode{\sphinxupquote{s{[}0{]}='p'}} échoue).

\sphinxAtStartPar
L’opérateur \sphinxcode{\sphinxupquote{+}} permet de concaténer des chaînes de caractères. L’opérateur \sphinxcode{\sphinxupquote{*}} permet de répeter \sphinxcode{\sphinxupquote{n}} fois la même chaîne de caractère

\begin{sphinxVerbatim}[commandchars=\\\{\}]
\PYG{n}{s1} \PYG{o}{=} \PYG{l+s+s2}{\PYGZdq{}}\PYG{l+s+s2}{Bonjour}\PYG{l+s+s2}{\PYGZdq{}}
\PYG{n}{s2} \PYG{o}{=} \PYG{l+s+s1}{\PYGZsq{}}\PYG{l+s+s1}{tout le monde}\PYG{l+s+s1}{\PYGZsq{}}
\PYG{n+nb}{print}\PYG{p}{(}\PYG{n}{s1} \PYG{o}{+} \PYG{l+s+s1}{\PYGZsq{}}\PYG{l+s+s1}{ }\PYG{l+s+s1}{\PYGZsq{}} \PYG{o}{+} \PYG{n}{s2}\PYG{p}{)}
\end{sphinxVerbatim}

\begin{sphinxVerbatim}[commandchars=\\\{\}]
Bonjour tout le monde
\end{sphinxVerbatim}

\begin{sphinxVerbatim}[commandchars=\\\{\}]
\PYG{n+nb}{print}\PYG{p}{(}\PYG{l+s+s1}{\PYGZsq{}}\PYG{l+s+s1}{ha!}\PYG{l+s+s1}{\PYGZsq{}}\PYG{o}{*}\PYG{l+m+mi}{10}\PYG{p}{)}
\end{sphinxVerbatim}

\begin{sphinxVerbatim}[commandchars=\\\{\}]
ha!ha!ha!ha!ha!ha!ha!ha!ha!ha!
\end{sphinxVerbatim}


\subsubsection{Formatage des chaînes de caractère}
\label{\detokenize{cours4_chaine_caractere_cours:formatage-des-chaines-de-caractere}}
\sphinxAtStartPar
Le formatage d’une chaîne de caractère consiste à mettre dans une chaine un élément variable. Cette opération est souvent utilisée lorsque l’on veut afficher proprement
un résultat

\begin{sphinxVerbatim}[commandchars=\\\{\}]
\PYG{n}{heure} \PYG{o}{=} \PYG{l+m+mi}{15}
\PYG{n}{minute} \PYG{o}{=} \PYG{l+m+mi}{30}
\PYG{l+s+s2}{\PYGZdq{}}\PYG{l+s+s2}{Il est }\PYG{l+s+si}{\PYGZob{}0\PYGZcb{}}\PYG{l+s+s2}{h}\PYG{l+s+si}{\PYGZob{}1\PYGZcb{}}\PYG{l+s+s2}{\PYGZdq{}}\PYG{o}{.}\PYG{n}{format}\PYG{p}{(}\PYG{n}{heure}\PYG{p}{,} \PYG{n}{minute}\PYG{p}{)}
\end{sphinxVerbatim}

\begin{sphinxVerbatim}[commandchars=\\\{\}]
\PYGZsq{}Il est 15h30\PYGZsq{}
\end{sphinxVerbatim}

\sphinxAtStartPar
Pour insérer un élément ou plusieurs éléments variables
dans une chaîne de caractère, on crée d’abord cette chaîne en mettant à la
place des ces éléments une accolade avec un numéro d’ordre \sphinxcode{\sphinxupquote{\{i\}}}. En appliquant la
méthode \sphinxcode{\sphinxupquote{format}} sur cette chaîne, les accolades seront remplacées par
le ième argument.

\sphinxAtStartPar
Il est possible de passer l’argument par nom dans ce cas la clé est le nom de l’argument.

\begin{sphinxVerbatim}[commandchars=\\\{\}]
\PYG{l+s+s2}{\PYGZdq{}}\PYG{l+s+s2}{Il est }\PYG{l+s+si}{\PYGZob{}heure\PYGZcb{}}\PYG{l+s+s2}{h}\PYG{l+s+si}{\PYGZob{}minute\PYGZcb{}}\PYG{l+s+s2}{\PYGZdq{}}\PYG{o}{.}\PYG{n}{format}\PYG{p}{(}\PYG{n}{heure}\PYG{o}{=}\PYG{n}{heure}\PYG{p}{,} \PYG{n}{minute}\PYG{o}{=}\PYG{n}{minute}\PYG{p}{)}
\end{sphinxVerbatim}

\begin{sphinxVerbatim}[commandchars=\\\{\}]
\PYGZsq{}Il est 15h30\PYGZsq{}
\end{sphinxVerbatim}

\sphinxAtStartPar
Depuis la version 3.6 de Python, il est possible de demander à Python d’utiliser automatiquement les variables locales à l’aide du préfix \sphinxcode{\sphinxupquote{f}}.

\begin{sphinxVerbatim}[commandchars=\\\{\}]
\PYG{l+s+sa}{f}\PYG{l+s+s2}{\PYGZdq{}}\PYG{l+s+s2}{Il est }\PYG{l+s+si}{\PYGZob{}}\PYG{n}{heure}\PYG{l+s+si}{\PYGZcb{}}\PYG{l+s+s2}{h}\PYG{l+s+si}{\PYGZob{}}\PYG{n}{minute}\PYG{l+s+si}{\PYGZcb{}}\PYG{l+s+s2}{\PYGZdq{}}
\end{sphinxVerbatim}

\begin{sphinxVerbatim}[commandchars=\\\{\}]
\PYGZsq{}Il est 15h30\PYGZsq{}
\end{sphinxVerbatim}

\sphinxAtStartPar
Il est aussi possible de demander d’utiliser un attribut d’un objet :

\begin{sphinxVerbatim}[commandchars=\\\{\}]
\PYG{n}{z} \PYG{o}{=} \PYG{l+m+mi}{1} \PYG{o}{+} \PYG{l+m+mi}{2}\PYG{n}{J}
\PYG{n+nb}{print}\PYG{p}{(}\PYG{l+s+sa}{f}\PYG{l+s+s1}{\PYGZsq{}}\PYG{l+s+s1}{Re(z) = }\PYG{l+s+si}{\PYGZob{}}\PYG{n}{z}\PYG{o}{.}\PYG{n}{real}\PYG{l+s+si}{\PYGZcb{}}\PYG{l+s+s1}{\PYGZsq{}}\PYG{p}{)}
\end{sphinxVerbatim}

\begin{sphinxVerbatim}[commandchars=\\\{\}]
Re(z) = 1.0
\end{sphinxVerbatim}

\sphinxAtStartPar
En utilisant le formatage de chaîne de caractère, il est possible de spécifier en détail comment ce nombre doit s’afficher. Par exemple, si il s’agit d’un nombre à virgule flottante, combien de décimales faut\sphinxhyphen{}il afficher, faut il utiliser la notation scientifique, etc. Pour cela, on rajoute à l’intérieur des accolades un code particulier. Ce code est précédé du signe ‘:’.

\begin{sphinxVerbatim}[commandchars=\\\{\}]
\PYG{k+kn}{from} \PYG{n+nn}{math} \PYG{k+kn}{import} \PYG{n}{pi}
\PYG{l+s+s1}{\PYGZsq{}}\PYG{l+s+si}{\PYGZob{}0:.5f\PYGZcb{}}\PYG{l+s+s1}{\PYGZsq{}}\PYG{o}{.}\PYG{n}{format}\PYG{p}{(}\PYG{n}{pi}\PYG{p}{)}
\PYG{n}{c} \PYG{o}{=} \PYG{l+m+mf}{299792458.} \PYG{c+c1}{\PYGZsh{} Vitesse de la lumière en m/s}
\PYG{l+s+s1}{\PYGZsq{}}\PYG{l+s+s1}{c = }\PYG{l+s+si}{\PYGZob{}0:.3e\PYGZcb{}}\PYG{l+s+s1}{ m/s}\PYG{l+s+s1}{\PYGZsq{}}\PYG{o}{.}\PYG{n}{format}\PYG{p}{(}\PYG{n}{c}\PYG{p}{)}
\end{sphinxVerbatim}

\begin{sphinxVerbatim}[commandchars=\\\{\}]
\PYGZsq{}c = 2.998e+08 m/s\PYGZsq{}
\end{sphinxVerbatim}

\sphinxAtStartPar
Le ‘f’ indique que l’on veut une notation a virgule fixe, le ‘e’ une notation scientifique. Le chiffre que l’on indique après le ‘.’ donne le nombre de chiffre après la virgule que l’on souhaite.

\sphinxAtStartPar
\sphinxhref{https://docs.python.org/fr/3.5/library/string.html\#formatstrings}{L’aide en ligne} de Python fournit d’autres exemples et des détails.


\subsubsection{Quelques méthodes utiles}
\label{\detokenize{cours4_chaine_caractere_cours:quelques-methodes-utiles}}\begin{itemize}
\item {} 
\sphinxAtStartPar
\sphinxcode{\sphinxupquote{strip}} (enlève les espaces blanc au début et fin de la chaîne)

\item {} 
\sphinxAtStartPar
\sphinxcode{\sphinxupquote{split}} (coupe la chaine et renvoie une liste de chaîne)

\item {} 
\sphinxAtStartPar
\sphinxcode{\sphinxupquote{join}} (inverse de slit : rassemble une liste de chaîne avec un chaîne)

\item {} 
\sphinxAtStartPar
\sphinxcode{\sphinxupquote{startswith}}, \sphinxcode{\sphinxupquote{endswith}}

\item {} 
\sphinxAtStartPar
\sphinxcode{\sphinxupquote{lower}}, \sphinxcode{\sphinxupquote{upper}} (convertit en minuscule ou majuscule)

\item {} 
\sphinxAtStartPar
\sphinxcode{\sphinxupquote{replace}}

\end{itemize}

\begin{sphinxVerbatim}[commandchars=\\\{\}]
\PYG{n}{s}\PYG{o}{=}\PYG{l+s+s2}{\PYGZdq{}}\PYG{l+s+s2}{     bonjour   }\PYG{l+s+s2}{\PYGZdq{}}
\PYG{n+nb}{print}\PYG{p}{(}\PYG{n}{s}\PYG{p}{)}
\PYG{n+nb}{print}\PYG{p}{(}\PYG{n}{s}\PYG{o}{.}\PYG{n}{strip}\PYG{p}{(}\PYG{p}{)}\PYG{p}{)}

\PYG{n}{s}\PYG{o}{=}\PYG{l+s+s1}{\PYGZsq{}}\PYG{l+s+s1}{un deux trois}\PYG{l+s+s1}{\PYGZsq{}}
\PYG{n+nb}{print}\PYG{p}{(}\PYG{n}{s}\PYG{o}{.}\PYG{n}{split}\PYG{p}{(}\PYG{p}{)}\PYG{p}{)}

\PYG{n}{l} \PYG{o}{=} \PYG{p}{[}\PYG{l+s+s1}{\PYGZsq{}}\PYG{l+s+s1}{pomme}\PYG{l+s+s1}{\PYGZsq{}}\PYG{p}{,} \PYG{l+s+s1}{\PYGZsq{}}\PYG{l+s+s1}{pêche}\PYG{l+s+s1}{\PYGZsq{}}\PYG{p}{,} \PYG{l+s+s1}{\PYGZsq{}}\PYG{l+s+s1}{poire}\PYG{l+s+s1}{\PYGZsq{}}\PYG{p}{,} \PYG{l+s+s1}{\PYGZsq{}}\PYG{l+s+s1}{abricot}\PYG{l+s+s1}{\PYGZsq{}}\PYG{p}{]}
\PYG{n}{s} \PYG{o}{=} \PYG{l+s+s1}{\PYGZsq{}}\PYG{l+s+s1}{, }\PYG{l+s+s1}{\PYGZsq{}}\PYG{o}{.}\PYG{n}{join}\PYG{p}{(}\PYG{n}{l}\PYG{p}{)}
\PYG{n+nb}{print}\PYG{p}{(}\PYG{n}{s}\PYG{p}{)}
\end{sphinxVerbatim}

\begin{sphinxVerbatim}[commandchars=\\\{\}]
     bonjour   
bonjour
[\PYGZsq{}un\PYGZsq{}, \PYGZsq{}deux\PYGZsq{}, \PYGZsq{}trois\PYGZsq{}]
pomme, pêche, poire, abricot
\end{sphinxVerbatim}


\subsubsection{Unicode}
\label{\detokenize{cours4_chaine_caractere_cours:unicode}}
\sphinxAtStartPar
Unicode est un standard informatique qui permet des échanges de textes dans différentes langues, à un niveau mondial. Il vise au codage de texte écrit en donnant à tout caractère de n’importe quel système d’écriture un nom et un identifiant numérique, et ce de manière unifiée, quelle que soit la plateforme informatique ou le logiciel utilisé.

\sphinxAtStartPar
On peut créer des chaînes directement en unicode. On peut aussi utiliser le code en hexadecimal.

\begin{sphinxVerbatim}[commandchars=\\\{\}]
\PYG{n}{s1} \PYG{o}{=} \PYG{l+s+s2}{\PYGZdq{}}\PYG{l+s+s2}{Rayon γ}\PYG{l+s+s2}{\PYGZdq{}}
\PYG{n}{s2} \PYG{o}{=} \PYG{l+s+s2}{\PYGZdq{}}\PYG{l+s+s2}{Rayon }\PYG{l+s+se}{\PYGZbs{}u03B3}\PYG{l+s+s2}{\PYGZdq{}}
\PYG{n+nb}{print}\PYG{p}{(}\PYG{n}{s2}\PYG{p}{)}
\PYG{k}{assert} \PYG{n}{s1}\PYG{o}{==}\PYG{n}{s2}
\end{sphinxVerbatim}

\begin{sphinxVerbatim}[commandchars=\\\{\}]
Rayon γ
\end{sphinxVerbatim}

\sphinxAtStartPar
Il est possible de convertir un caractère en nombre et vice\sphinxhyphen{}versa

\begin{sphinxVerbatim}[commandchars=\\\{\}]
\PYG{n+nb}{print}\PYG{p}{(}\PYG{n+nb}{ord}\PYG{p}{(}\PYG{l+s+s1}{\PYGZsq{}}\PYG{l+s+s1}{€}\PYG{l+s+s1}{\PYGZsq{}}\PYG{p}{)}\PYG{p}{)}
\PYG{n+nb}{print}\PYG{p}{(}\PYG{n+nb}{hex}\PYG{p}{(}\PYG{n+nb}{ord}\PYG{p}{(}\PYG{l+s+s1}{\PYGZsq{}}\PYG{l+s+s1}{€}\PYG{l+s+s1}{\PYGZsq{}}\PYG{p}{)}\PYG{p}{)}\PYG{p}{)}
\end{sphinxVerbatim}

\begin{sphinxVerbatim}[commandchars=\\\{\}]
8364
0x20ac
\end{sphinxVerbatim}

\begin{sphinxVerbatim}[commandchars=\\\{\}]
\PYG{l+s+s1}{\PYGZsq{}}\PYG{l+s+s1}{ }\PYG{l+s+s1}{\PYGZsq{}}\PYG{o}{.}\PYG{n}{join}\PYG{p}{(}\PYG{p}{[}\PYG{n+nb}{chr}\PYG{p}{(}\PYG{l+m+mi}{97} \PYG{o}{+} \PYG{n}{i}\PYG{p}{)} \PYG{k}{for} \PYG{n}{i} \PYG{o+ow}{in} \PYG{n+nb}{range}\PYG{p}{(}\PYG{l+m+mi}{26}\PYG{p}{)}\PYG{p}{]}\PYG{p}{)}
\end{sphinxVerbatim}

\begin{sphinxVerbatim}[commandchars=\\\{\}]
\PYGZsq{}a b c d e f g h i j k l m n o p q r s t u v w x y z\PYGZsq{}
\end{sphinxVerbatim}

\sphinxAtStartPar
On trouve même des émoticones

\begin{sphinxVerbatim}[commandchars=\\\{\}]
\PYG{n}{s} \PYG{o}{=} \PYG{l+s+s2}{\PYGZdq{}}\PYG{l+s+se}{\PYGZbs{}U0001f600}\PYG{l+s+s2}{\PYGZdq{}}
\PYG{n+nb}{print}\PYG{p}{(}\PYG{n}{s}\PYG{p}{)}
\end{sphinxVerbatim}

\begin{sphinxVerbatim}[commandchars=\\\{\}]
😀
\end{sphinxVerbatim}

\sphinxAtStartPar
On peut aussi utilier le nom unicode

\begin{sphinxVerbatim}[commandchars=\\\{\}]
\PYG{n+nb}{print}\PYG{p}{(}\PYG{l+s+s2}{\PYGZdq{}}\PYG{l+s+se}{\PYGZbs{}N\PYGZob{}slightly smiling face\PYGZcb{}}\PYG{l+s+s2}{\PYGZdq{}}\PYG{p}{)}
\end{sphinxVerbatim}

\begin{sphinxVerbatim}[commandchars=\\\{\}]
🙂
\end{sphinxVerbatim}


\section{Feuilles d’exercices}
\label{\detokenize{feuilles_d_exercices:feuilles-d-exercices}}\label{\detokenize{feuilles_d_exercices::doc}}

\subsection{Exercices sur les fonctions}
\label{\detokenize{cours1_fonctions_exercices:exercices-sur-les-fonctions}}\label{\detokenize{cours1_fonctions_exercices::doc}}

\subsubsection{Que fait cette fonction ?}
\label{\detokenize{cours1_fonctions_exercices:que-fait-cette-fonction}}
\sphinxAtStartPar
Répondre aux questions de cet exercice sans s’aider de l’ordinateur, puis vérifier.
\begin{enumerate}
\sphinxsetlistlabels{\arabic}{enumi}{enumii}{}{.}%
\item {} 
\sphinxAtStartPar
On souhaite calculer \(\sum \frac{1}{i}\). Quelle est la différence entre les différentes fonctions si dessous ? Laquelle est la “bonne” fonction, laquelle la “pire”

\end{enumerate}

\begin{sphinxVerbatim}[commandchars=\\\{\}]
\PYG{k}{def} \PYG{n+nf}{serie\PYGZus{}1}\PYG{p}{(}\PYG{n}{n}\PYG{p}{)}\PYG{p}{:}
    \PYG{n}{output} \PYG{o}{=} \PYG{l+m+mi}{0}
    \PYG{k}{for} \PYG{n}{i} \PYG{o+ow}{in} \PYG{n+nb}{range}\PYG{p}{(}\PYG{l+m+mi}{1}\PYG{p}{,} \PYG{n}{n}\PYG{o}{+}\PYG{l+m+mi}{1}\PYG{p}{)}\PYG{p}{:}
        \PYG{n}{output} \PYG{o}{=} \PYG{n}{output} \PYG{o}{+} \PYG{l+m+mi}{1}\PYG{o}{/}\PYG{n}{i}
        \PYG{n+nb}{print}\PYG{p}{(}\PYG{n}{output}\PYG{p}{)}
        
\PYG{n}{serie\PYGZus{}1}\PYG{p}{(}\PYG{l+m+mi}{5}\PYG{p}{)}
        
\PYG{k}{def} \PYG{n+nf}{serie\PYGZus{}2}\PYG{p}{(}\PYG{n}{n}\PYG{p}{)}\PYG{p}{:}
    \PYG{n}{output} \PYG{o}{=} \PYG{l+m+mi}{0}
    \PYG{k}{for} \PYG{n}{i} \PYG{o+ow}{in} \PYG{n+nb}{range}\PYG{p}{(}\PYG{l+m+mi}{1}\PYG{p}{,} \PYG{n}{n}\PYG{o}{+}\PYG{l+m+mi}{1}\PYG{p}{)}\PYG{p}{:}
        \PYG{n}{output} \PYG{o}{=} \PYG{n}{output} \PYG{o}{+} \PYG{l+m+mi}{1}\PYG{o}{/}\PYG{n}{i}
        \PYG{k}{return} \PYG{n}{output}

\PYG{n}{serie\PYGZus{}2}\PYG{p}{(}\PYG{l+m+mi}{5}\PYG{p}{)}    
    
\PYG{k}{def} \PYG{n+nf}{serie\PYGZus{}3}\PYG{p}{(}\PYG{n}{n}\PYG{p}{)}\PYG{p}{:}
    \PYG{n}{output} \PYG{o}{=} \PYG{l+m+mi}{0}
    \PYG{k}{for} \PYG{n}{i} \PYG{o+ow}{in} \PYG{n+nb}{range}\PYG{p}{(}\PYG{l+m+mi}{1}\PYG{p}{,} \PYG{n}{n}\PYG{o}{+}\PYG{l+m+mi}{1}\PYG{p}{)}\PYG{p}{:}
        \PYG{n}{output} \PYG{o}{=} \PYG{n}{output} \PYG{o}{+} \PYG{l+m+mi}{1}\PYG{o}{/}\PYG{n}{i}
    \PYG{n+nb}{print}\PYG{p}{(}\PYG{n}{output}\PYG{p}{)}
    
\PYG{n}{serie\PYGZus{}3}\PYG{p}{(}\PYG{l+m+mi}{5}\PYG{p}{)}

        
\PYG{k}{def} \PYG{n+nf}{serie\PYGZus{}4}\PYG{p}{(}\PYG{n}{n}\PYG{p}{)}\PYG{p}{:}
    \PYG{n}{output} \PYG{o}{=} \PYG{l+m+mi}{0}
    \PYG{k}{for} \PYG{n}{i} \PYG{o+ow}{in} \PYG{n+nb}{range}\PYG{p}{(}\PYG{l+m+mi}{1}\PYG{p}{,} \PYG{n}{n}\PYG{o}{+}\PYG{l+m+mi}{1}\PYG{p}{)}\PYG{p}{:}
        \PYG{n}{output} \PYG{o}{=} \PYG{n}{output} \PYG{o}{+} \PYG{l+m+mi}{1}\PYG{o}{/}\PYG{n}{i}
    \PYG{k}{return} \PYG{n}{output}

\PYG{n}{serie\PYGZus{}4}\PYG{p}{(}\PYG{l+m+mi}{5}\PYG{p}{)}
\end{sphinxVerbatim}
\begin{enumerate}
\sphinxsetlistlabels{\arabic}{enumi}{enumii}{}{.}%
\setcounter{enumi}{1}
\item {} 
\sphinxAtStartPar
Parmi tous les appels de la fonction \sphinxcode{\sphinxupquote{f}} ci dessous, lequels vont faire une erreur ? Quelle sera l’erreur ?

\end{enumerate}

\begin{sphinxVerbatim}[commandchars=\\\{\}]
\PYG{k}{def} \PYG{n+nf}{f}\PYG{p}{(}\PYG{n}{a}\PYG{p}{,} \PYG{n}{b}\PYG{p}{)}\PYG{p}{:}
    \PYG{k}{return} \PYG{n}{a}\PYG{o}{*}\PYG{l+m+mi}{2} \PYG{o}{+} \PYG{n}{b}

\PYG{n}{p} \PYG{o}{=} \PYG{p}{[}\PYG{l+m+mi}{1}\PYG{p}{,} \PYG{l+m+mi}{2}\PYG{p}{]}
\end{sphinxVerbatim}

\begin{sphinxVerbatim}[commandchars=\\\{\}]
\PYG{n}{f}\PYG{p}{(}\PYG{l+m+mi}{1}\PYG{p}{,} \PYG{l+m+mi}{3}\PYG{p}{,} \PYG{l+m+mi}{4}\PYG{p}{)}
\PYG{n}{f}\PYG{p}{(}\PYG{l+m+mi}{1}\PYG{p}{,} \PYG{l+m+mi}{2}\PYG{p}{)}
\PYG{n}{f}\PYG{p}{(}\PYG{n}{Bonjour}\PYG{p}{,} \PYG{n}{Hello}\PYG{p}{)}
\PYG{n}{f}\PYG{p}{(}\PYG{l+m+mi}{1}\PYG{p}{,} \PYG{n}{a}\PYG{o}{=}\PYG{l+m+mi}{2}\PYG{p}{)}
\PYG{n}{f}\PYG{p}{(}\PYG{n}{b}\PYG{o}{=}\PYG{l+m+mi}{1}\PYG{p}{,} \PYG{n}{a}\PYG{o}{=}\PYG{l+m+mi}{2}\PYG{p}{)}
\PYG{n}{f}\PYG{p}{(}\PYG{l+s+s2}{\PYGZdq{}}\PYG{l+s+s2}{Bonjour}\PYG{l+s+s2}{\PYGZdq{}}\PYG{p}{,} \PYG{l+s+s2}{\PYGZdq{}}\PYG{l+s+s2}{Hello}\PYG{l+s+s2}{\PYGZdq{}}\PYG{p}{)}
\PYG{n}{f}\PYG{p}{[}\PYG{l+m+mi}{1}\PYG{p}{,} \PYG{l+m+mi}{2}\PYG{p}{]}
\PYG{n}{f}\PYG{p}{(}\PYG{l+s+s2}{\PYGZdq{}}\PYG{l+s+s2}{Bonjour, Hello}\PYG{l+s+s2}{\PYGZdq{}}\PYG{p}{)}
\PYG{n}{f}\PYG{p}{(}\PYG{l+m+mi}{1}\PYG{p}{)}
\PYG{n}{f}\PYG{p}{(}\PYG{o}{*}\PYG{o}{*}\PYG{n}{p}\PYG{p}{)}
\PYG{n}{f}\PYG{p}{(}\PYG{o}{*}\PYG{n}{p}\PYG{p}{)}
\end{sphinxVerbatim}
\begin{enumerate}
\sphinxsetlistlabels{\arabic}{enumi}{enumii}{}{.}%
\setcounter{enumi}{2}
\item {} 
\sphinxAtStartPar
Qu’est il affiché ? Dans quels fonction \sphinxcode{\sphinxupquote{x}} est une variable globale ?

\end{enumerate}

\begin{sphinxVerbatim}[commandchars=\\\{\}]
\PYG{k}{def} \PYG{n+nf}{f1}\PYG{p}{(}\PYG{n}{x}\PYG{p}{,} \PYG{n}{y}\PYG{p}{)}\PYG{p}{:}
    \PYG{n+nb}{print}\PYG{p}{(}\PYG{n}{x} \PYG{o}{+} \PYG{n}{y}\PYG{p}{)}

\PYG{k}{def} \PYG{n+nf}{f2}\PYG{p}{(}\PYG{n}{y}\PYG{p}{)}\PYG{p}{:}
    \PYG{n}{x} \PYG{o}{=} \PYG{l+m+mi}{15}
    \PYG{n+nb}{print}\PYG{p}{(}\PYG{n}{x} \PYG{o}{+} \PYG{n}{y}\PYG{p}{)}

\PYG{k}{def} \PYG{n+nf}{f3}\PYG{p}{(}\PYG{n}{x}\PYG{p}{,} \PYG{n}{y}\PYG{p}{)}\PYG{p}{:}
    \PYG{n+nb}{print}\PYG{p}{(}\PYG{n}{x} \PYG{o}{+} \PYG{n}{y}\PYG{p}{)}
    \PYG{n}{x} \PYG{o}{=} \PYG{l+m+mi}{15}

\PYG{k}{def} \PYG{n+nf}{f4}\PYG{p}{(}\PYG{n}{y}\PYG{p}{)}\PYG{p}{:}
    \PYG{n+nb}{print}\PYG{p}{(}\PYG{n}{x} \PYG{o}{+} \PYG{n}{y}\PYG{p}{)}
    \PYG{n}{x} \PYG{o}{=} \PYG{l+m+mi}{15}

\PYG{n}{x} \PYG{o}{=} \PYG{l+m+mi}{10}
\PYG{n}{f1}\PYG{p}{(}\PYG{l+m+mi}{1}\PYG{p}{,} \PYG{l+m+mi}{2}\PYG{p}{)}
\PYG{n}{x} \PYG{o}{=} \PYG{l+m+mi}{5}
\PYG{n}{f1}\PYG{p}{(}\PYG{l+m+mi}{1}\PYG{p}{,} \PYG{l+m+mi}{2}\PYG{p}{)}
\PYG{n}{f2}\PYG{p}{(}\PYG{l+m+mi}{1}\PYG{p}{)}
\PYG{n+nb}{print}\PYG{p}{(}\PYG{n}{x}\PYG{p}{)}    
\PYG{n}{f3}\PYG{p}{(}\PYG{l+m+mi}{1}\PYG{p}{,} \PYG{l+m+mi}{2}\PYG{p}{)}
\PYG{n}{f4}\PYG{p}{(}\PYG{l+m+mi}{1}\PYG{p}{)} \PYG{c+c1}{\PYGZsh{} Attention, il y a un piège ici.}
\end{sphinxVerbatim}
\begin{enumerate}
\sphinxsetlistlabels{\arabic}{enumi}{enumii}{}{.}%
\setcounter{enumi}{3}
\item {} 
\sphinxAtStartPar
Qu’est il affiché ?

\end{enumerate}

\begin{sphinxVerbatim}[commandchars=\\\{\}]
\PYG{k+kn}{from} \PYG{n+nn}{math} \PYG{k+kn}{import} \PYG{n}{sin}

\PYG{n}{pi} \PYG{o}{=} \PYG{l+m+mf}{3.141592653589793}

\PYG{k}{def} \PYG{n+nf}{g}\PYG{p}{(}\PYG{p}{)}\PYG{p}{:}
    \PYG{k}{return} \PYG{n}{sin}\PYG{p}{(}\PYG{n}{pi}\PYG{o}{/}\PYG{l+m+mi}{2}\PYG{p}{)}
    
\PYG{k}{def} \PYG{n+nf}{f}\PYG{p}{(}\PYG{p}{)}\PYG{p}{:}
    \PYG{n}{pi} \PYG{o}{=} \PYG{l+m+mi}{0}
    \PYG{k}{return} \PYG{n}{sin}\PYG{p}{(}\PYG{n}{pi}\PYG{o}{/}\PYG{l+m+mi}{2}\PYG{p}{)}

\PYG{n+nb}{print}\PYG{p}{(}\PYG{n}{f}\PYG{p}{(}\PYG{p}{)}\PYG{p}{)}
\PYG{n+nb}{print}\PYG{p}{(}\PYG{n}{g}\PYG{p}{(}\PYG{p}{)}\PYG{p}{)}
\PYG{n}{pi} \PYG{o}{=} \PYG{l+m+mi}{0}
\PYG{n+nb}{print}\PYG{p}{(}\PYG{n}{f}\PYG{p}{(}\PYG{p}{)}\PYG{p}{)}
\PYG{n+nb}{print}\PYG{p}{(}\PYG{n}{g}\PYG{p}{(}\PYG{p}{)}\PYG{p}{)}
\end{sphinxVerbatim}
\begin{enumerate}
\sphinxsetlistlabels{\arabic}{enumi}{enumii}{}{.}%
\setcounter{enumi}{3}
\item {} 
\sphinxAtStartPar
Qu’est il affiché ?

\end{enumerate}

\begin{sphinxVerbatim}[commandchars=\\\{\}]
\PYG{k}{def} \PYG{n+nf}{f}\PYG{p}{(}\PYG{n}{a}\PYG{p}{,} \PYG{n}{b}\PYG{p}{,} \PYG{n}{c}\PYG{p}{)}\PYG{p}{:}
    \PYG{n+nb}{print}\PYG{p}{(}\PYG{l+m+mi}{100}\PYG{o}{*}\PYG{n}{a} \PYG{o}{+} \PYG{l+m+mi}{10}\PYG{o}{*}\PYG{n}{b} \PYG{o}{+} \PYG{n}{c}\PYG{p}{)}

\PYG{n}{a} \PYG{o}{=} \PYG{l+m+mi}{1}
\PYG{n}{b} \PYG{o}{=} \PYG{l+m+mi}{2}
\PYG{n}{c} \PYG{o}{=} \PYG{l+m+mi}{3}
\PYG{n}{f}\PYG{p}{(}\PYG{n}{a}\PYG{p}{,} \PYG{n}{b}\PYG{p}{,} \PYG{n}{c}\PYG{p}{)}
\PYG{n}{f}\PYG{p}{(}\PYG{n}{c}\PYG{p}{,} \PYG{n}{b}\PYG{p}{,} \PYG{n}{a}\PYG{p}{)}
\PYG{n}{f}\PYG{p}{(}\PYG{n}{a}\PYG{p}{,} \PYG{n}{b}\PYG{o}{=}\PYG{n}{c}\PYG{p}{,} \PYG{n}{c}\PYG{o}{=}\PYG{n}{b}\PYG{p}{)}
\PYG{n}{f}\PYG{p}{(}\PYG{n}{a}\PYG{o}{=}\PYG{n}{a}\PYG{p}{,} \PYG{n}{b}\PYG{o}{=}\PYG{n}{a}\PYG{p}{,} \PYG{n}{c}\PYG{o}{=}\PYG{n}{a}\PYG{p}{)}
\end{sphinxVerbatim}


\subsubsection{Fonction cos\_deg}
\label{\detokenize{cours1_fonctions_exercices:fonction-cos-deg}}
\sphinxAtStartPar
Ecrire une fonction qui renvoie le cosinus d’un angle exprimé en degré


\subsubsection{Volume d’un cône}
\label{\detokenize{cours1_fonctions_exercices:volume-d-un-cone}}\begin{enumerate}
\sphinxsetlistlabels{\arabic}{enumi}{enumii}{}{.}%
\item {} 
\sphinxAtStartPar
Ecrire une fonction qui renoit le volume d’un cône de rayon \(r\) et hauteur \(h\).

\item {} 
\sphinxAtStartPar
Ecrire une fonction qui envoit le volume d’un tronc de cône de rayon \(r_1\) et \(r_2\).

\item {} 
\sphinxAtStartPar
Ecrire une seule fonction pour laquelle le tronc de cône est par défaut un cône (i.e. \(r_2=1\))

\end{enumerate}


\subsubsection{Fonction datetime}
\label{\detokenize{cours1_fonctions_exercices:fonction-datetime}}
\sphinxAtStartPar
Importer la fonction \sphinxcode{\sphinxupquote{datetime}} du module \sphinxcode{\sphinxupquote{datetime}} et regarder se documentation.
\begin{enumerate}
\sphinxsetlistlabels{\arabic}{enumi}{enumii}{}{.}%
\item {} 
\sphinxAtStartPar
Utiliser cette fonction pour entrer votre date de naissance (et l’heure si vous la connaissez) en nomant explicitement les arguments.

\item {} 
\sphinxAtStartPar
Même question mais après avoir mis les arguments dans une liste

\item {} 
\sphinxAtStartPar
Dans un dictionnaire

\end{enumerate}


\subsubsection{Fonction date en français}
\label{\detokenize{cours1_fonctions_exercices:fonction-date-en-francais}}\begin{enumerate}
\sphinxsetlistlabels{\arabic}{enumi}{enumii}{}{.}%
\item {} 
\sphinxAtStartPar
Ecrire une fonction \sphinxcode{\sphinxupquote{date\_en\_francais}} qui renvoie un objet \sphinxcode{\sphinxupquote{date}} mais dont les arguments sont en français (\sphinxcode{\sphinxupquote{annee}}, \sphinxcode{\sphinxupquote{mois}}, \sphinxcode{\sphinxupquote{jour}})

\item {} 
\sphinxAtStartPar
Idem avec la fonction \sphinxcode{\sphinxupquote{datetime}} du module \sphinxcode{\sphinxupquote{datetime}} (il faudra rajouter les arguments optionels \sphinxcode{\sphinxupquote{heure}}, \sphinxcode{\sphinxupquote{minute}}, \sphinxcode{\sphinxupquote{seconde}})

\end{enumerate}


\subsubsection{Polynômes}
\label{\detokenize{cours1_fonctions_exercices:polynomes}}
\sphinxAtStartPar
On considère un polynôme de degrés \(n\) : \(a_0 + a_1x + a_2x^2 + ... + a_n x^n\).
\begin{enumerate}
\sphinxsetlistlabels{\arabic}{enumi}{enumii}{}{.}%
\item {} 
\sphinxAtStartPar
Ecrire une fonction \sphinxcode{\sphinxupquote{eval\_polynome\_troisieme\_degres(x, a\_0, a\_1, a\_2, a\_3)}} qui évalue un tel polynôme en \(x\).

\item {} 
\sphinxAtStartPar
Ecrire une fonction qui permettra d’évaluer un polynôme de degré inférieur à 4 est que l’on pourra utiliser de la façon suivante :

\begin{sphinxVerbatim}[commandchars=\\\{\}]
 eval\PYGZus{}polynome(x, 3, 4) \PYGZsh{} 3x + 4
 eval\PYGZus{}polynome(x, a\PYGZus{}0=2, a\PYGZus{}2=4) \PYGZsh{} 4x\PYGZca{}2 + 4
\end{sphinxVerbatim}

\item {} 
\sphinxAtStartPar
Faire en sorte que cette fonction marche pour n’importe quel degré.

\end{enumerate}


\subsubsection{Equation du second degré}
\label{\detokenize{cours1_fonctions_exercices:equation-du-second-degre}}
\sphinxAtStartPar
On souhaite trouver les solutions de l’équation : \(ax^2 + bx + c=0\). On rappelle que pour cela on calcule le discriminant \(\Delta = b^2 - 4*a*c\).
\begin{itemize}
\item {} 
\sphinxAtStartPar
Si \(\Delta>0\), les solutions sont \(\frac{-b \pm \sqrt{Delta}}{2a}\)

\item {} 
\sphinxAtStartPar
Si \(\Delta=0\), il y a une solution (double) \(\frac{-b }{2a}\)

\item {} 
\sphinxAtStartPar
Si \(\Delta<0\), les solutions sont \(\frac{-b \pm i\sqrt{Delta}}{2a}\)

\end{itemize}
\begin{enumerate}
\sphinxsetlistlabels{\arabic}{enumi}{enumii}{}{.}%
\item {} 
\sphinxAtStartPar
Ecrire une fonction equation\_second\_degre qui donne les solutions.

\item {} 
\sphinxAtStartPar
Tester l’équation \(x^2 -3x + 2\)

\item {} 
\sphinxAtStartPar
Ecrire une fonction qui résout une équation du premier degré.

\item {} 
\sphinxAtStartPar
Faire en sorte que cette fonction soit appelée lorsque \(a=0\).

\end{enumerate}


\subsubsection{Coordonnée polaire d’un nombre complexe}
\label{\detokenize{cours1_fonctions_exercices:coordonnee-polaire-d-un-nombre-complexe}}
\sphinxAtStartPar
On considère une nombre complexe \(z\) et sa représentation polaire : \(z = re^{i\theta}\)
\begin{enumerate}
\sphinxsetlistlabels{\arabic}{enumi}{enumii}{}{.}%
\item {} 
\sphinxAtStartPar
Ecrire une fonction qui à partir de \(r\) et \(\theta\) renvoie \(z\)

\end{enumerate}


\subsubsection{Nombres premiers}
\label{\detokenize{cours1_fonctions_exercices:nombres-premiers}}
\sphinxAtStartPar
Ecrire une fonction qui renvoie True si un nombre est premier et False sinon. Pour cela, on testera si le nombre
est divisible par les entiers successifs à partir de 2.

\sphinxAtStartPar
On admet que si \(n\) n’est pas premier, alors il existe un entier \(p \ge 2\) tel que \(p^2 \le n\) qui est un diviseur de \(n\).
\begin{enumerate}
\sphinxsetlistlabels{\arabic}{enumi}{enumii}{}{.}%
\item {} 
\sphinxAtStartPar
Ecrire la fonction à l’aide d’une boucle.

\item {} 
\sphinxAtStartPar
Combien y a t\sphinxhyphen{}il d’années au XXIème siècle qui sont des nombres premiers ?

\item {} 
\sphinxAtStartPar
Si vous n’êtes pas convaincu que c’est mieux avec des fonctions, faites le sans…

\end{enumerate}


\subsubsection{Mention}
\label{\detokenize{cours1_fonctions_exercices:mention}}
\sphinxAtStartPar
Ecrire une fonction qui à partir de la note sur 20 donne la mention


\subsection{Exercices sur les nombres}
\label{\detokenize{cours2_nombres_exercices:exercices-sur-les-nombres}}\label{\detokenize{cours2_nombres_exercices::doc}}

\subsubsection{Fonctions mathématiques}
\label{\detokenize{cours2_nombres_exercices:fonctions-mathematiques}}\begin{itemize}
\item {} 
\sphinxAtStartPar
Est ce que la fonction log est le logarithme décimal ou népérien ?

\item {} 
\sphinxAtStartPar
Calculer \(x = \sqrt{2}\) puis calculer \(x^2\). Que se passe\sphinxhyphen{}t\sphinxhyphen{}il ?

\item {} 
\sphinxAtStartPar
Calculer \(\arccos{\frac{\sqrt{2}}{2}}\) et comparer à sa valeur théorique.

\end{itemize}


\subsubsection{Constante de structure fine}
\label{\detokenize{cours2_nombres_exercices:constante-de-structure-fine}}
\sphinxAtStartPar
La constante de structure fine est définie en physique comme étant égale à
\begin{equation*}
\begin{split} 
\alpha = \frac{e^2}{2\epsilon_0 h c}
\end{split}
\end{equation*}
\sphinxAtStartPar
où
\begin{itemize}
\item {} 
\sphinxAtStartPar
\(e\) est la charge de l’électron et vaut \(1.602176634 \times 10^{-19} C\)

\item {} 
\sphinxAtStartPar
\(h\) est la constante de Planck et vaut \(6.626\,070\,15 \times 10^{-34} J s\)

\item {} 
\sphinxAtStartPar
\(\epsilon_0\) la permitivité du vide et vaut \(8.8541878128 \times 10^{-12} F/m\)

\item {} 
\sphinxAtStartPar
\(c\) la célérité de la lumière dans le vide, \(c=299792458 m/s\)

\end{itemize}

\sphinxAtStartPar
Définissez en Python les variables \sphinxcode{\sphinxupquote{e}}, \sphinxcode{\sphinxupquote{hbar}}, \sphinxcode{\sphinxupquote{epsilon\_0}} et \sphinxcode{\sphinxupquote{c}}. Calculez \(\alpha\) et \(1/\alpha\)


\subsubsection{Précision des nombres}
\label{\detokenize{cours2_nombres_exercices:precision-des-nombres}}\begin{itemize}
\item {} 
\sphinxAtStartPar
Soit \(x=1\) et \(\epsilon = 10^{-15}\). Calculez \(y=x + \epsilon\) et ensuite \(y - x\).

\item {} 
\sphinxAtStartPar
Pourquoi le résultat est différent de \(10^{-15}\).

\item {} 
\sphinxAtStartPar
Que vaut cette valeur ?

\end{itemize}


\subsubsection{Calcul d’une dérivée}
\label{\detokenize{cours2_nombres_exercices:calcul-d-une-derivee}}
\sphinxAtStartPar
On considère une fonction \(f(x)\). On rappelle que la dérivée peut se définir comme
\begin{equation*}
\begin{split}
f^\prime(x) = \lim_{\epsilon\rightarrow0}\frac{f(x+\epsilon) - f(x)}{\epsilon}
\end{split}
\end{equation*}
\sphinxAtStartPar
Pour calculer numériquement une dérivée, il faut évaluer la limite en prenant une valeur ‘petite’ de \(\epsilon\).

\sphinxAtStartPar
On prendra comme exemple \(f(x) = \sin(x)\).
\begin{itemize}
\item {} 
\sphinxAtStartPar
Calculer numériquement la dérivée de \(f\) en \(\pi/4\) en utilisant la formule pour \(\epsilon = 10^{-6}\).

\item {} 
\sphinxAtStartPar
Comparer à la valeur théorique \(\cos(x)\) pour différentes valeurs de \(\epsilon\) que l’on prendra comme puissance de 10 (\(\epsilon = 10^{-n}\)). Que se passe\sphinxhyphen{}t\sphinxhyphen{}il si \(\epsilon\) est trop petit ? trop grand ?

\item {} 
\sphinxAtStartPar
Ecrire la fonction \sphinxcode{\sphinxupquote{sin\_prime(x, epsilon)}} qui calcule la dérivée de sin en \(x\)

\item {} 
\sphinxAtStartPar
Ecrire une fonction qui prend une fonction quelconque et renvoie la fonction dérivée.

\end{itemize}


\subsubsection{Nombre complexe}
\label{\detokenize{cours2_nombres_exercices:nombre-complexe}}\begin{itemize}
\item {} 
\sphinxAtStartPar
Ecrire une fonction qui calcule le module d’un nombre complexe \(z\)

\item {} 
\sphinxAtStartPar
Ecrire une fonction qui à partir de \(r\) et \(\theta\) renvoie le nombre \(z = re^{i\theta} = r\cos(\theta) + ir\sin(\theta)\)

\end{itemize}


\subsection{Exercices sur les conteneurs}
\label{\detokenize{cours3_conteneur_exercices:exercices-sur-les-conteneurs}}\label{\detokenize{cours3_conteneur_exercices::doc}}

\subsubsection{Manipulation des listes}
\label{\detokenize{cours3_conteneur_exercices:manipulation-des-listes}}
\sphinxAtStartPar
On considère la liste {[}1, 5, 3, 5, 6, 2{]}
\begin{enumerate}
\sphinxsetlistlabels{\arabic}{enumi}{enumii}{}{.}%
\item {} 
\sphinxAtStartPar
Écrire une fonction ‘somme’ qui renvoie la somme des éléments d’une liste de nombres. On fera explicitement la boucle for.

\item {} 
\sphinxAtStartPar
Écrire une fonction ‘maximum’ qui renvoie le maximum des éléments d’une liste de nombres. On fera explicitement la boucle for.

\item {} 
\sphinxAtStartPar
Écrire une fonction ‘arg\_maximum’ qui renvoie l’indice du maximum d’une liste de nombres. On fera explicitement la boucle for.

\item {} 
\sphinxAtStartPar
Écrire une fonction ‘trouve’ qui renvoie l’indice correspondant à l’argument. On fera explicitement la boucle for.

\item {} 
\sphinxAtStartPar
Comment répondre aux questions 1, 2, 3, 4 en utilisant des fonctions déjà existantes ?

\end{enumerate}


\subsubsection{Liste comprehension}
\label{\detokenize{cours3_conteneur_exercices:liste-comprehension}}\begin{enumerate}
\sphinxsetlistlabels{\arabic}{enumi}{enumii}{}{.}%
\item {} 
\sphinxAtStartPar
Créer une liste nomée \sphinxcode{\sphinxupquote{nombres}} contenant les entiers de 0 à 9 inclus

\item {} 
\sphinxAtStartPar
Créer une liste contenant la racine carré des éléments de \sphinxcode{\sphinxupquote{nombres}} (on utilisera une comprehension de liste)

\item {} 
\sphinxAtStartPar
Créer une liste contenant tous les nombres pairs de la listes \sphinxcode{\sphinxupquote{nombres}} (on utilisera une comprehension de liste)

\item {} 
\sphinxAtStartPar
Toujours en utilisant un comprehension de liste, considérant deux listes \sphinxcode{\sphinxupquote{l1}} et \sphinxcode{\sphinxupquote{l2}}, créer une nouvelle liste contenant les couples pris deux à deux de l1 et l2. On supposera que les deux liste ont la même longueur. Quelle fonction python fait la même chose ?

\item {} 
\sphinxAtStartPar
En utilisant la fonction de la question 4 et la liste de la question 2 vérifier que l’on a bien \(y=x^2\) pour chaque élément.

\end{enumerate}


\subsubsection{Exercice de base sur les dictionnaires}
\label{\detokenize{cours3_conteneur_exercices:exercice-de-base-sur-les-dictionnaires}}\begin{enumerate}
\sphinxsetlistlabels{\arabic}{enumi}{enumii}{}{.}%
\item {} 
\sphinxAtStartPar
Tout d’abord, nous allons créer un petit dictionnaire qui contient des informations sur un étudiant. Utilisons les clés et valeurs suivantes en exemple :
\begin{itemize}
\item {} 
\sphinxAtStartPar
‘nom’: ‘Jean Dupont’

\item {} 
\sphinxAtStartPar
‘âge’: 20

\item {} 
\sphinxAtStartPar
‘filière’: ‘Informatique’

\end{itemize}

\item {} 
\sphinxAtStartPar
Modifier l’age pour qu’il soit égal à 21

\item {} 
\sphinxAtStartPar
Afficher le genre de l’étudiant si il possède une telle clé sinon afficher un message inquant que l’on ne connait pas son genre.

\end{enumerate}


\subsubsection{Exercice sur les ensembles}
\label{\detokenize{cours3_conteneur_exercices:exercice-sur-les-ensembles}}
\sphinxAtStartPar
La fonction chr permet de convertir un code ASCII en un caractère. La liste des lettres majuscules peut être obtenue à partir de la commande suivante :

\begin{sphinxVerbatim}[commandchars=\\\{\}]
\PYG{n}{liste\PYGZus{}majuscules} \PYG{o}{=} \PYG{p}{[}\PYG{n+nb}{chr}\PYG{p}{(}\PYG{l+m+mi}{65}\PYG{o}{+}\PYG{n}{i}\PYG{p}{)} \PYG{k}{for} \PYG{n}{i} \PYG{o+ow}{in} \PYG{n+nb}{range}\PYG{p}{(}\PYG{l+m+mi}{26}\PYG{p}{)}\PYG{p}{]}
\end{sphinxVerbatim}

\sphinxAtStartPar
On souhaite vérifier qu’un mot de passe entré par un utilisateur est sufisament compliqué. Voici les règles :
\begin{itemize}
\item {} 
\sphinxAtStartPar
Il doit contenir 12 caractères différents

\item {} 
\sphinxAtStartPar
Il doit contenir au moins 2 majuscules différentes

\item {} 
\sphinxAtStartPar
Il doit contenir au moins un caractère de ponctuation \sphinxcode{\sphinxupquote{.,;:!?}}

\item {} 
\sphinxAtStartPar
Il ne doit pas contenir d’espace

\end{itemize}

\sphinxAtStartPar
Ecrire une fonction qui renvoie True si toutes les conditions sont vérifiées et False sinon


\subsection{Chaînes de caractères}
\label{\detokenize{cours4_chaine_caractere_exercices:chaines-de-caracteres}}\label{\detokenize{cours4_chaine_caractere_exercices::doc}}

\subsubsection{Liste de prix}
\label{\detokenize{cours4_chaine_caractere_exercices:liste-de-prix}}
\sphinxAtStartPar
Voici un liste de prix sous forme d’un dictionnaire. Afficher la liste de prix de la façon suivante :



\begin{sphinxVerbatim}[commandchars=\\\{\}]
\PYG{n}{price\PYGZus{}liste} \PYG{o}{=} \PYG{p}{\PYGZob{}}\PYG{l+s+s1}{\PYGZsq{}}\PYG{l+s+s1}{tomates}\PYG{l+s+s1}{\PYGZsq{}}\PYG{p}{:}\PYG{l+m+mf}{3.4}\PYG{p}{,}
     \PYG{l+s+s1}{\PYGZsq{}}\PYG{l+s+s1}{pommes}\PYG{l+s+s1}{\PYGZsq{}}\PYG{p}{:}\PYG{l+m+mf}{2.49}\PYG{p}{,}
     \PYG{l+s+s1}{\PYGZsq{}}\PYG{l+s+s1}{oignons}\PYG{l+s+s1}{\PYGZsq{}}\PYG{p}{:}\PYG{l+m+mf}{1.45}\PYG{p}{\PYGZcb{}}
\end{sphinxVerbatim}


\subsubsection{Unicode}
\label{\detokenize{cours4_chaine_caractere_exercices:unicode}}
\sphinxAtStartPar
Afficher toutes les lettres greques de α à ω. (On pourra copier coller ces lettres pour avoir leurs unicodes)


\subsubsection{Combien La Fontaine a t\sphinxhyphen{}il utilisé de mots différents dans ses fables ?}
\label{\detokenize{cours4_chaine_caractere_exercices:combien-la-fontaine-a-t-il-utilise-de-mots-differents-dans-ses-fables}}
\sphinxAtStartPar
Les lignes suivantes permettent de télécharger l’ensemble des fables de la Fontaine.
\begin{itemize}
\item {} 
\sphinxAtStartPar
Combien y a\sphinxhyphen{}t\sphinxhyphen{}il de mots différents ? On pourra d’abord remplacer toutes les ponctuations par des espaces, puis créer une liste de mots que l’on mettra en minuscule. On créera ensuite un ensemble dont on regardera la taille.

\item {} 
\sphinxAtStartPar
Quelle est le mot le plus long ?

\end{itemize}


\section{Correction}
\label{\detokenize{correction:correction}}\label{\detokenize{correction::doc}}

\subsection{Exercices sur les fonctions}
\label{\detokenize{cours1_fonctions_corr_exercices:exercices-sur-les-fonctions}}\label{\detokenize{cours1_fonctions_corr_exercices::doc}}

\subsubsection{Que fait cette fonction ?}
\label{\detokenize{cours1_fonctions_corr_exercices:que-fait-cette-fonction}}
\sphinxAtStartPar
Répondre aux questions de cet exercice sans s’aider de l’ordinateur, puis vérifier.
\begin{enumerate}
\sphinxsetlistlabels{\arabic}{enumi}{enumii}{}{.}%
\item {} 
\sphinxAtStartPar
On souhaite calculer \(\sum \frac{1}{i}\). Quelle est la différence entre les différentes fonctions si dessous ? Laquelle est la “bonne” fonction, laquelle la “pire”

\end{enumerate}

\begin{sphinxVerbatim}[commandchars=\\\{\}]
\PYG{k}{def} \PYG{n+nf}{serie\PYGZus{}1}\PYG{p}{(}\PYG{n}{n}\PYG{p}{)}\PYG{p}{:}
    \PYG{n}{output} \PYG{o}{=} \PYG{l+m+mi}{0}
    \PYG{k}{for} \PYG{n}{i} \PYG{o+ow}{in} \PYG{n+nb}{range}\PYG{p}{(}\PYG{l+m+mi}{1}\PYG{p}{,} \PYG{n}{n}\PYG{o}{+}\PYG{l+m+mi}{1}\PYG{p}{)}\PYG{p}{:}
        \PYG{n}{output} \PYG{o}{=} \PYG{n}{output} \PYG{o}{+} \PYG{l+m+mi}{1}\PYG{o}{/}\PYG{n}{i}
        \PYG{n+nb}{print}\PYG{p}{(}\PYG{n}{output}\PYG{p}{)}
        
\PYG{n}{serie\PYGZus{}1}\PYG{p}{(}\PYG{l+m+mi}{5}\PYG{p}{)}
        
\PYG{k}{def} \PYG{n+nf}{serie\PYGZus{}2}\PYG{p}{(}\PYG{n}{n}\PYG{p}{)}\PYG{p}{:}
    \PYG{n}{output} \PYG{o}{=} \PYG{l+m+mi}{0}
    \PYG{k}{for} \PYG{n}{i} \PYG{o+ow}{in} \PYG{n+nb}{range}\PYG{p}{(}\PYG{l+m+mi}{1}\PYG{p}{,} \PYG{n}{n}\PYG{o}{+}\PYG{l+m+mi}{1}\PYG{p}{)}\PYG{p}{:}
        \PYG{n}{output} \PYG{o}{=} \PYG{n}{output} \PYG{o}{+} \PYG{l+m+mi}{1}\PYG{o}{/}\PYG{n}{i}
        \PYG{k}{return} \PYG{n}{output}

\PYG{n}{serie\PYGZus{}2}\PYG{p}{(}\PYG{l+m+mi}{5}\PYG{p}{)}    
    
\PYG{k}{def} \PYG{n+nf}{serie\PYGZus{}3}\PYG{p}{(}\PYG{n}{n}\PYG{p}{)}\PYG{p}{:}
    \PYG{n}{output} \PYG{o}{=} \PYG{l+m+mi}{0}
    \PYG{k}{for} \PYG{n}{i} \PYG{o+ow}{in} \PYG{n+nb}{range}\PYG{p}{(}\PYG{l+m+mi}{1}\PYG{p}{,} \PYG{n}{n}\PYG{o}{+}\PYG{l+m+mi}{1}\PYG{p}{)}\PYG{p}{:}
        \PYG{n}{output} \PYG{o}{=} \PYG{n}{output} \PYG{o}{+} \PYG{l+m+mi}{1}\PYG{o}{/}\PYG{n}{i}
    \PYG{n+nb}{print}\PYG{p}{(}\PYG{n}{output}\PYG{p}{)}
    
\PYG{n}{serie\PYGZus{}3}\PYG{p}{(}\PYG{l+m+mi}{5}\PYG{p}{)}

        
\PYG{k}{def} \PYG{n+nf}{serie\PYGZus{}4}\PYG{p}{(}\PYG{n}{n}\PYG{p}{)}\PYG{p}{:}
    \PYG{n}{output} \PYG{o}{=} \PYG{l+m+mi}{0}
    \PYG{k}{for} \PYG{n}{i} \PYG{o+ow}{in} \PYG{n+nb}{range}\PYG{p}{(}\PYG{l+m+mi}{1}\PYG{p}{,} \PYG{n}{n}\PYG{o}{+}\PYG{l+m+mi}{1}\PYG{p}{)}\PYG{p}{:}
        \PYG{n}{output} \PYG{o}{=} \PYG{n}{output} \PYG{o}{+} \PYG{l+m+mi}{1}\PYG{o}{/}\PYG{n}{i}
    \PYG{k}{return} \PYG{n}{output}

\PYG{n}{serie\PYGZus{}4}\PYG{p}{(}\PYG{l+m+mi}{5}\PYG{p}{)}
\end{sphinxVerbatim}

\begin{sphinxVerbatim}[commandchars=\\\{\}]
1.0
1.5
1.8333333333333333
2.083333333333333
2.283333333333333
2.283333333333333
\end{sphinxVerbatim}

\begin{sphinxVerbatim}[commandchars=\\\{\}]
2.283333333333333
\end{sphinxVerbatim}
\begin{enumerate}
\sphinxsetlistlabels{\arabic}{enumi}{enumii}{}{.}%
\setcounter{enumi}{1}
\item {} 
\sphinxAtStartPar
Parmi tous les appels de la fonction \sphinxcode{\sphinxupquote{f}} ci dessous, lequels vont faire une erreur ? Quelle sera l’erreur ?

\end{enumerate}

\begin{sphinxVerbatim}[commandchars=\\\{\}]
\PYG{k}{def} \PYG{n+nf}{f}\PYG{p}{(}\PYG{n}{a}\PYG{p}{,} \PYG{n}{b}\PYG{p}{)}\PYG{p}{:}
    \PYG{k}{return} \PYG{n}{a}\PYG{o}{*}\PYG{l+m+mi}{2} \PYG{o}{+} \PYG{n}{b}

\PYG{n}{p} \PYG{o}{=} \PYG{p}{[}\PYG{l+m+mi}{1}\PYG{p}{,} \PYG{l+m+mi}{2}\PYG{p}{]}
\end{sphinxVerbatim}

\begin{sphinxVerbatim}[commandchars=\\\{\}]
\PYG{n}{f}\PYG{p}{(}\PYG{l+m+mi}{1}\PYG{p}{,} \PYG{l+m+mi}{3}\PYG{p}{,} \PYG{l+m+mi}{4}\PYG{p}{)}
\PYG{n}{f}\PYG{p}{(}\PYG{l+m+mi}{1}\PYG{p}{,} \PYG{l+m+mi}{2}\PYG{p}{)}
\PYG{n}{f}\PYG{p}{(}\PYG{n}{Bonjour}\PYG{p}{,} \PYG{n}{Hello}\PYG{p}{)}
\PYG{n}{f}\PYG{p}{(}\PYG{l+m+mi}{1}\PYG{p}{,} \PYG{n}{a}\PYG{o}{=}\PYG{l+m+mi}{2}\PYG{p}{)}
\PYG{n}{f}\PYG{p}{(}\PYG{n}{b}\PYG{o}{=}\PYG{l+m+mi}{1}\PYG{p}{,} \PYG{n}{a}\PYG{o}{=}\PYG{l+m+mi}{2}\PYG{p}{)}
\PYG{n}{f}\PYG{p}{(}\PYG{l+s+s2}{\PYGZdq{}}\PYG{l+s+s2}{Bonjour}\PYG{l+s+s2}{\PYGZdq{}}\PYG{p}{,} \PYG{l+s+s2}{\PYGZdq{}}\PYG{l+s+s2}{Hello}\PYG{l+s+s2}{\PYGZdq{}}\PYG{p}{)}
\PYG{n}{f}\PYG{p}{[}\PYG{l+m+mi}{1}\PYG{p}{,} \PYG{l+m+mi}{2}\PYG{p}{]}
\PYG{n}{f}\PYG{p}{(}\PYG{l+s+s2}{\PYGZdq{}}\PYG{l+s+s2}{Bonjour, Hello}\PYG{l+s+s2}{\PYGZdq{}}\PYG{p}{)}
\PYG{n}{f}\PYG{p}{(}\PYG{l+m+mi}{1}\PYG{p}{)}
\PYG{n}{f}\PYG{p}{(}\PYG{o}{*}\PYG{o}{*}\PYG{n}{p}\PYG{p}{)}
\PYG{n}{f}\PYG{p}{(}\PYG{o}{*}\PYG{n}{p}\PYG{p}{)}
\end{sphinxVerbatim}
\begin{enumerate}
\sphinxsetlistlabels{\arabic}{enumi}{enumii}{}{.}%
\setcounter{enumi}{2}
\item {} 
\sphinxAtStartPar
Qu’est il affiché ? Dans quels fonction \sphinxcode{\sphinxupquote{x}} est une variable globale ?

\end{enumerate}

\begin{sphinxVerbatim}[commandchars=\\\{\}]
\PYG{k}{def} \PYG{n+nf}{f1}\PYG{p}{(}\PYG{n}{x}\PYG{p}{,} \PYG{n}{y}\PYG{p}{)}\PYG{p}{:}
    \PYG{n+nb}{print}\PYG{p}{(}\PYG{n}{x} \PYG{o}{+} \PYG{n}{y}\PYG{p}{)}

\PYG{k}{def} \PYG{n+nf}{f2}\PYG{p}{(}\PYG{n}{y}\PYG{p}{)}\PYG{p}{:}
    \PYG{n}{x} \PYG{o}{=} \PYG{l+m+mi}{15}
    \PYG{n+nb}{print}\PYG{p}{(}\PYG{n}{x} \PYG{o}{+} \PYG{n}{y}\PYG{p}{)}

\PYG{k}{def} \PYG{n+nf}{f3}\PYG{p}{(}\PYG{n}{x}\PYG{p}{,} \PYG{n}{y}\PYG{p}{)}\PYG{p}{:}
    \PYG{n+nb}{print}\PYG{p}{(}\PYG{n}{x} \PYG{o}{+} \PYG{n}{y}\PYG{p}{)}
    \PYG{n}{x} \PYG{o}{=} \PYG{l+m+mi}{15}

\PYG{k}{def} \PYG{n+nf}{f4}\PYG{p}{(}\PYG{n}{y}\PYG{p}{)}\PYG{p}{:}
    \PYG{n+nb}{print}\PYG{p}{(}\PYG{n}{x} \PYG{o}{+} \PYG{n}{y}\PYG{p}{)}
    \PYG{n}{x} \PYG{o}{=} \PYG{l+m+mi}{15}

\PYG{n}{x} \PYG{o}{=} \PYG{l+m+mi}{10}
\PYG{n}{f1}\PYG{p}{(}\PYG{l+m+mi}{1}\PYG{p}{,} \PYG{l+m+mi}{2}\PYG{p}{)}
\PYG{n}{x} \PYG{o}{=} \PYG{l+m+mi}{5}
\PYG{n}{f1}\PYG{p}{(}\PYG{l+m+mi}{1}\PYG{p}{,} \PYG{l+m+mi}{2}\PYG{p}{)}
\PYG{n}{f2}\PYG{p}{(}\PYG{l+m+mi}{1}\PYG{p}{)}
\PYG{n+nb}{print}\PYG{p}{(}\PYG{n}{x}\PYG{p}{)}    
\PYG{n}{f3}\PYG{p}{(}\PYG{l+m+mi}{1}\PYG{p}{,} \PYG{l+m+mi}{2}\PYG{p}{)}
\PYG{n}{f4}\PYG{p}{(}\PYG{l+m+mi}{1}\PYG{p}{)} \PYG{c+c1}{\PYGZsh{} Attention, il y a un piège ici.}
\end{sphinxVerbatim}

\begin{sphinxVerbatim}[commandchars=\\\{\}]
3
3
16
5
3
\end{sphinxVerbatim}

\begin{sphinxVerbatim}[commandchars=\\\{\}]
\PYG{g+gt}{\PYGZhy{}\PYGZhy{}\PYGZhy{}\PYGZhy{}\PYGZhy{}\PYGZhy{}\PYGZhy{}\PYGZhy{}\PYGZhy{}\PYGZhy{}\PYGZhy{}\PYGZhy{}\PYGZhy{}\PYGZhy{}\PYGZhy{}\PYGZhy{}\PYGZhy{}\PYGZhy{}\PYGZhy{}\PYGZhy{}\PYGZhy{}\PYGZhy{}\PYGZhy{}\PYGZhy{}\PYGZhy{}\PYGZhy{}\PYGZhy{}\PYGZhy{}\PYGZhy{}\PYGZhy{}\PYGZhy{}\PYGZhy{}\PYGZhy{}\PYGZhy{}\PYGZhy{}\PYGZhy{}\PYGZhy{}\PYGZhy{}\PYGZhy{}\PYGZhy{}\PYGZhy{}\PYGZhy{}\PYGZhy{}\PYGZhy{}\PYGZhy{}\PYGZhy{}\PYGZhy{}\PYGZhy{}\PYGZhy{}\PYGZhy{}\PYGZhy{}\PYGZhy{}\PYGZhy{}\PYGZhy{}\PYGZhy{}\PYGZhy{}\PYGZhy{}\PYGZhy{}\PYGZhy{}\PYGZhy{}\PYGZhy{}\PYGZhy{}\PYGZhy{}\PYGZhy{}\PYGZhy{}\PYGZhy{}\PYGZhy{}\PYGZhy{}\PYGZhy{}\PYGZhy{}\PYGZhy{}\PYGZhy{}\PYGZhy{}\PYGZhy{}\PYGZhy{}}
\PYG{n+ne}{UnboundLocalError}\PYG{g+gWhitespace}{                         }Traceback (most recent call last)
\PYG{o}{\PYGZlt{}}\PYG{n}{ipython}\PYG{o}{\PYGZhy{}}\PYG{n+nb}{input}\PYG{o}{\PYGZhy{}}\PYG{l+m+mi}{3}\PYG{o}{\PYGZhy{}}\PYG{n}{bf7078096a9a}\PYG{o}{\PYGZgt{}} \PYG{o+ow}{in} \PYG{o}{\PYGZlt{}}\PYG{n}{module}\PYG{o}{\PYGZgt{}}
\PYG{g+gWhitespace}{     }\PYG{l+m+mi}{21} \PYG{n+nb}{print}\PYG{p}{(}\PYG{n}{x}\PYG{p}{)}
\PYG{g+gWhitespace}{     }\PYG{l+m+mi}{22} \PYG{n}{f3}\PYG{p}{(}\PYG{l+m+mi}{1}\PYG{p}{,} \PYG{l+m+mi}{2}\PYG{p}{)}
\PYG{n+ne}{\PYGZhy{}\PYGZhy{}\PYGZhy{}\PYGZgt{} }\PYG{l+m+mi}{23} \PYG{n}{f4}\PYG{p}{(}\PYG{l+m+mi}{1}\PYG{p}{)}

\PYG{n+nn}{\PYGZlt{}ipython\PYGZhy{}input\PYGZhy{}3\PYGZhy{}bf7078096a9a\PYGZgt{}} in \PYG{n+ni}{f4}\PYG{n+nt}{(y)}
\PYG{g+gWhitespace}{     }\PYG{l+m+mi}{11} 
\PYG{g+gWhitespace}{     }\PYG{l+m+mi}{12} \PYG{k}{def} \PYG{n+nf}{f4}\PYG{p}{(}\PYG{n}{y}\PYG{p}{)}\PYG{p}{:}
\PYG{n+ne}{\PYGZhy{}\PYGZhy{}\PYGZhy{}\PYGZgt{} }\PYG{l+m+mi}{13}     \PYG{n+nb}{print}\PYG{p}{(}\PYG{n}{x} \PYG{o}{+} \PYG{n}{y}\PYG{p}{)}
\PYG{g+gWhitespace}{     }\PYG{l+m+mi}{14}     \PYG{n}{x} \PYG{o}{=} \PYG{l+m+mi}{15}
\PYG{g+gWhitespace}{     }\PYG{l+m+mi}{15} 

\PYG{n+ne}{UnboundLocalError}: local variable \PYGZsq{}x\PYGZsq{} referenced before assignment
\end{sphinxVerbatim}
\begin{enumerate}
\sphinxsetlistlabels{\arabic}{enumi}{enumii}{}{.}%
\setcounter{enumi}{3}
\item {} 
\sphinxAtStartPar
Qu’est il affiché ?

\end{enumerate}

\begin{sphinxVerbatim}[commandchars=\\\{\}]
\PYG{k+kn}{from} \PYG{n+nn}{math} \PYG{k+kn}{import} \PYG{n}{sin}

\PYG{n}{pi} \PYG{o}{=} \PYG{l+m+mf}{3.141592653589793}

\PYG{k}{def} \PYG{n+nf}{g}\PYG{p}{(}\PYG{p}{)}\PYG{p}{:}
    \PYG{k}{return} \PYG{n}{sin}\PYG{p}{(}\PYG{n}{pi}\PYG{o}{/}\PYG{l+m+mi}{2}\PYG{p}{)}
    
\PYG{k}{def} \PYG{n+nf}{f}\PYG{p}{(}\PYG{p}{)}\PYG{p}{:}
    \PYG{n}{pi} \PYG{o}{=} \PYG{l+m+mi}{0}
    \PYG{k}{return} \PYG{n}{sin}\PYG{p}{(}\PYG{n}{pi}\PYG{o}{/}\PYG{l+m+mi}{2}\PYG{p}{)}

\PYG{n+nb}{print}\PYG{p}{(}\PYG{n}{f}\PYG{p}{(}\PYG{p}{)}\PYG{p}{)}
\PYG{n+nb}{print}\PYG{p}{(}\PYG{n}{g}\PYG{p}{(}\PYG{p}{)}\PYG{p}{)}
\PYG{n}{pi} \PYG{o}{=} \PYG{l+m+mi}{0}
\PYG{n+nb}{print}\PYG{p}{(}\PYG{n}{f}\PYG{p}{(}\PYG{p}{)}\PYG{p}{)}
\PYG{n+nb}{print}\PYG{p}{(}\PYG{n}{g}\PYG{p}{(}\PYG{p}{)}\PYG{p}{)}
\end{sphinxVerbatim}

\begin{sphinxVerbatim}[commandchars=\\\{\}]
0.0
1.0
0.0
0.0
\end{sphinxVerbatim}
\begin{enumerate}
\sphinxsetlistlabels{\arabic}{enumi}{enumii}{}{.}%
\setcounter{enumi}{3}
\item {} 
\sphinxAtStartPar
Qu’est il affiché ?

\end{enumerate}

\begin{sphinxVerbatim}[commandchars=\\\{\}]
\PYG{k}{def} \PYG{n+nf}{f}\PYG{p}{(}\PYG{n}{a}\PYG{p}{,} \PYG{n}{b}\PYG{p}{,} \PYG{n}{c}\PYG{p}{)}\PYG{p}{:}
    \PYG{n+nb}{print}\PYG{p}{(}\PYG{l+m+mi}{100}\PYG{o}{*}\PYG{n}{a} \PYG{o}{+} \PYG{l+m+mi}{10}\PYG{o}{*}\PYG{n}{b} \PYG{o}{+} \PYG{n}{c}\PYG{p}{)}

\PYG{n}{a} \PYG{o}{=} \PYG{l+m+mi}{1}
\PYG{n}{b} \PYG{o}{=} \PYG{l+m+mi}{2}
\PYG{n}{c} \PYG{o}{=} \PYG{l+m+mi}{3}
\PYG{n}{f}\PYG{p}{(}\PYG{n}{a}\PYG{p}{,} \PYG{n}{b}\PYG{p}{,} \PYG{n}{c}\PYG{p}{)}
\PYG{n}{f}\PYG{p}{(}\PYG{n}{c}\PYG{p}{,} \PYG{n}{b}\PYG{p}{,} \PYG{n}{a}\PYG{p}{)}
\PYG{n}{f}\PYG{p}{(}\PYG{n}{a}\PYG{p}{,} \PYG{n}{b}\PYG{o}{=}\PYG{n}{c}\PYG{p}{,} \PYG{n}{c}\PYG{o}{=}\PYG{n}{b}\PYG{p}{)}
\PYG{n}{f}\PYG{p}{(}\PYG{n}{a}\PYG{o}{=}\PYG{n}{a}\PYG{p}{,} \PYG{n}{b}\PYG{o}{=}\PYG{n}{a}\PYG{p}{,} \PYG{n}{c}\PYG{o}{=}\PYG{n}{a}\PYG{p}{)}
\end{sphinxVerbatim}

\begin{sphinxVerbatim}[commandchars=\\\{\}]
123
321
132
111
\end{sphinxVerbatim}


\subsubsection{Fonction cos\_deg}
\label{\detokenize{cours1_fonctions_corr_exercices:fonction-cos-deg}}
\sphinxAtStartPar
Ecrire une fonction qui renvoie le cosinus d’un angle exprimé en degré

\begin{sphinxVerbatim}[commandchars=\\\{\}]
\PYG{k+kn}{from} \PYG{n+nn}{math} \PYG{k+kn}{import} \PYG{n}{cos}\PYG{p}{,} \PYG{n}{pi}

\PYG{k}{def} \PYG{n+nf}{cos\PYGZus{}deg}\PYG{p}{(}\PYG{n}{angle}\PYG{p}{)}\PYG{p}{:}
    \PYG{k}{return} \PYG{n}{cos}\PYG{p}{(}\PYG{n}{angle}\PYG{o}{/}\PYG{l+m+mi}{180}\PYG{o}{*}\PYG{n}{pi}\PYG{p}{)}
\end{sphinxVerbatim}


\subsubsection{Volume d’un cône}
\label{\detokenize{cours1_fonctions_corr_exercices:volume-d-un-cone}}\begin{enumerate}
\sphinxsetlistlabels{\arabic}{enumi}{enumii}{}{.}%
\item {} 
\sphinxAtStartPar
Ecrire une fonction qui renoit le volume d’un cône de rayon \(r\) et hauteur \(h\).

\item {} 
\sphinxAtStartPar
Ecrire une fonction qui envoit le volume d’un tronc de cône de rayon \(r_1\) et \(r_2\).

\item {} 
\sphinxAtStartPar
Ecrire une seule fonction pour laquelle le tronc de cône est par défaut un cône (i.e. \(r_2=1\))

\end{enumerate}

\begin{sphinxVerbatim}[commandchars=\\\{\}]
\PYG{k+kn}{from} \PYG{n+nn}{math} \PYG{k+kn}{import} \PYG{n}{pi}

\PYG{k}{def} \PYG{n+nf}{volume\PYGZus{}cone}\PYG{p}{(}\PYG{n}{h}\PYG{p}{,} \PYG{n}{r}\PYG{p}{)}\PYG{p}{:}
    \PYG{k}{return} \PYG{n}{pi}\PYG{o}{*}\PYG{n}{h}\PYG{o}{*}\PYG{n}{r}\PYG{o}{*}\PYG{o}{*}\PYG{l+m+mi}{2}\PYG{o}{/}\PYG{l+m+mi}{3}

\PYG{k}{def} \PYG{n+nf}{volume\PYGZus{}tronc\PYGZus{}cone}\PYG{p}{(}\PYG{n}{h}\PYG{p}{,} \PYG{n}{r1}\PYG{p}{,} \PYG{n}{r2}\PYG{p}{)}\PYG{p}{:}
    \PYG{k}{return} \PYG{n}{pi}\PYG{o}{*}\PYG{n}{h}\PYG{o}{*}\PYG{p}{(}\PYG{n}{r1}\PYG{o}{*}\PYG{o}{*}\PYG{l+m+mi}{2} \PYG{o}{\PYGZhy{}} \PYG{n}{r2}\PYG{o}{*}\PYG{o}{*}\PYG{l+m+mi}{2}\PYG{p}{)}\PYG{o}{/}\PYG{l+m+mi}{3}

\PYG{k}{def} \PYG{n+nf}{volume\PYGZus{}tronc\PYGZus{}cone}\PYG{p}{(}\PYG{n}{h}\PYG{p}{,} \PYG{n}{r1}\PYG{p}{,} \PYG{n}{r2}\PYG{o}{=}\PYG{l+m+mi}{0}\PYG{p}{)}\PYG{p}{:}
    \PYG{k}{return} \PYG{n}{pi}\PYG{o}{*}\PYG{n}{h}\PYG{o}{*}\PYG{p}{(}\PYG{n}{r1}\PYG{o}{*}\PYG{o}{*}\PYG{l+m+mi}{2} \PYG{o}{\PYGZhy{}} \PYG{n}{r2}\PYG{o}{*}\PYG{o}{*}\PYG{l+m+mi}{2}\PYG{p}{)}\PYG{o}{/}\PYG{l+m+mi}{3}
\end{sphinxVerbatim}


\subsubsection{Fonction datetime}
\label{\detokenize{cours1_fonctions_corr_exercices:fonction-datetime}}
\sphinxAtStartPar
Importer la fonction \sphinxcode{\sphinxupquote{datetime}} du module \sphinxcode{\sphinxupquote{datetime}} et regarder se documentation.
\begin{enumerate}
\sphinxsetlistlabels{\arabic}{enumi}{enumii}{}{.}%
\item {} 
\sphinxAtStartPar
Utiliser cette fonction pour entrer votre date de naissance (et l’heure si vous la connaissez) en nomant explicitement les arguments.

\item {} 
\sphinxAtStartPar
Même question mais après avoir mis les arguments dans une liste

\item {} 
\sphinxAtStartPar
Dans un dictionnaire

\end{enumerate}

\begin{sphinxVerbatim}[commandchars=\\\{\}]
\PYG{k+kn}{from} \PYG{n+nn}{datetime} \PYG{k+kn}{import} \PYG{n}{date}\PYG{p}{,} \PYG{n}{datetime}

\PYG{n+nb}{print}\PYG{p}{(}\PYG{n}{date}\PYG{p}{(}\PYG{n}{year}\PYG{o}{=}\PYG{l+m+mi}{2011}\PYG{p}{,} \PYG{n}{month}\PYG{o}{=}\PYG{l+m+mi}{6}\PYG{p}{,} \PYG{n}{day}\PYG{o}{=}\PYG{l+m+mi}{2}\PYG{p}{)}\PYG{p}{)}

\PYG{n}{liste} \PYG{o}{=} \PYG{p}{[}\PYG{l+m+mi}{2011}\PYG{p}{,} \PYG{l+m+mi}{6}\PYG{p}{,} \PYG{l+m+mi}{2}\PYG{p}{]}

\PYG{n+nb}{print}\PYG{p}{(}\PYG{n}{date}\PYG{p}{(}\PYG{o}{*}\PYG{n}{liste}\PYG{p}{)}\PYG{p}{)}

\PYG{n}{parametre} \PYG{o}{=} \PYG{p}{\PYGZob{}}\PYG{l+s+s1}{\PYGZsq{}}\PYG{l+s+s1}{year}\PYG{l+s+s1}{\PYGZsq{}}\PYG{p}{:}\PYG{l+m+mi}{2011}\PYG{p}{,} \PYG{l+s+s1}{\PYGZsq{}}\PYG{l+s+s1}{month}\PYG{l+s+s1}{\PYGZsq{}}\PYG{p}{:}\PYG{l+m+mi}{6}\PYG{p}{,} \PYG{l+s+s1}{\PYGZsq{}}\PYG{l+s+s1}{day}\PYG{l+s+s1}{\PYGZsq{}}\PYG{p}{:}\PYG{l+m+mi}{2}\PYG{p}{\PYGZcb{}}

\PYG{n+nb}{print}\PYG{p}{(}\PYG{n}{date}\PYG{p}{(}\PYG{o}{*}\PYG{o}{*}\PYG{n}{parametre}\PYG{p}{)}\PYG{p}{)}
\end{sphinxVerbatim}

\begin{sphinxVerbatim}[commandchars=\\\{\}]
2011\PYGZhy{}06\PYGZhy{}02
2011\PYGZhy{}06\PYGZhy{}02
2011\PYGZhy{}06\PYGZhy{}02
\end{sphinxVerbatim}


\subsubsection{Fonction date en français}
\label{\detokenize{cours1_fonctions_corr_exercices:fonction-date-en-francais}}\begin{enumerate}
\sphinxsetlistlabels{\arabic}{enumi}{enumii}{}{.}%
\item {} 
\sphinxAtStartPar
Ecrire une fonction \sphinxcode{\sphinxupquote{date\_en\_francais}} qui renvoie un objet \sphinxcode{\sphinxupquote{date}} mais dont les arguments sont en français (\sphinxcode{\sphinxupquote{annee}}, \sphinxcode{\sphinxupquote{mois}}, \sphinxcode{\sphinxupquote{jour}})

\item {} 
\sphinxAtStartPar
Idem avec la fonction \sphinxcode{\sphinxupquote{datetime}} du module \sphinxcode{\sphinxupquote{datetime}} (il faudra rajouter les arguments optionels \sphinxcode{\sphinxupquote{heure}}, \sphinxcode{\sphinxupquote{minute}}, \sphinxcode{\sphinxupquote{seconde}})

\end{enumerate}

\begin{sphinxVerbatim}[commandchars=\\\{\}]
\PYG{k}{def} \PYG{n+nf}{date\PYGZus{}fr}\PYG{p}{(}\PYG{n}{annee}\PYG{p}{,} \PYG{n}{mois}\PYG{p}{,} \PYG{n}{jour}\PYG{p}{)}\PYG{p}{:}
    \PYG{k}{return} \PYG{n}{date}\PYG{p}{(}\PYG{n}{annee}\PYG{p}{,} \PYG{n}{mois}\PYG{p}{,} \PYG{n}{jour}\PYG{p}{)}

\PYG{n+nb}{print}\PYG{p}{(}\PYG{n}{date\PYGZus{}fr}\PYG{p}{(}\PYG{n}{annee}\PYG{o}{=}\PYG{l+m+mi}{2011}\PYG{p}{,} \PYG{n}{mois}\PYG{o}{=}\PYG{l+m+mi}{6}\PYG{p}{,} \PYG{n}{jour}\PYG{o}{=}\PYG{l+m+mi}{2}\PYG{p}{)}\PYG{p}{)}

\PYG{k}{def} \PYG{n+nf}{datetime\PYGZus{}fr}\PYG{p}{(}\PYG{n}{annee}\PYG{p}{,} \PYG{n}{mois}\PYG{p}{,} \PYG{n}{jour}\PYG{p}{,} \PYG{n}{heure}\PYG{o}{=}\PYG{l+m+mi}{0}\PYG{p}{,} \PYG{n}{minute}\PYG{o}{=}\PYG{l+m+mi}{0}\PYG{p}{,} \PYG{n}{seconde}\PYG{o}{=}\PYG{l+m+mi}{0}\PYG{p}{)}\PYG{p}{:}
    \PYG{k}{return} \PYG{n}{datetime}\PYG{p}{(}\PYG{n}{annee}\PYG{p}{,} \PYG{n}{mois}\PYG{p}{,} \PYG{n}{jour}\PYG{p}{,} \PYG{n}{heure}\PYG{p}{,} \PYG{n}{minute}\PYG{p}{,} \PYG{n}{seconde}\PYG{p}{)}

\PYG{n+nb}{print}\PYG{p}{(}\PYG{n}{datetime\PYGZus{}fr}\PYG{p}{(}\PYG{n}{annee}\PYG{o}{=}\PYG{l+m+mi}{2011}\PYG{p}{,} \PYG{n}{mois}\PYG{o}{=}\PYG{l+m+mi}{6}\PYG{p}{,} \PYG{n}{jour}\PYG{o}{=}\PYG{l+m+mi}{2}\PYG{p}{,} \PYG{n}{heure}\PYG{o}{=}\PYG{l+m+mi}{15}\PYG{p}{)}\PYG{p}{)}
\end{sphinxVerbatim}

\begin{sphinxVerbatim}[commandchars=\\\{\}]
2011\PYGZhy{}06\PYGZhy{}02
2011\PYGZhy{}06\PYGZhy{}02 15:00:00
\end{sphinxVerbatim}


\subsubsection{Polynômes}
\label{\detokenize{cours1_fonctions_corr_exercices:polynomes}}
\sphinxAtStartPar
On considère un polynôme de degrés \(n\) : \(a_0 + a_1x + a_2x^2 + ... + a_n x^n\).
\begin{enumerate}
\sphinxsetlistlabels{\arabic}{enumi}{enumii}{}{.}%
\item {} 
\sphinxAtStartPar
Ecrire une fonction \sphinxcode{\sphinxupquote{eval\_polynome\_troisieme\_degres(x, a\_0, a\_1, a\_2, a\_3)}} qui évalue un tel polynôme en \(x\).

\item {} 
\sphinxAtStartPar
Ecrire une fonction qui permettra d’évaluer un polynôme de degré inférieur à 4 est que l’on pourra utiliser de la façon suivante :

\begin{sphinxVerbatim}[commandchars=\\\{\}]
 eval\PYGZus{}polynome(x, 3, 4) \PYGZsh{} 3x + 4
 eval\PYGZus{}polynome(x, a\PYGZus{}0=2, a\PYGZus{}2=4) \PYGZsh{} 4x\PYGZca{}2 + 4
\end{sphinxVerbatim}

\item {} 
\sphinxAtStartPar
Faire en sorte que cette fonction marche pour n’importe quel degré.

\end{enumerate}

\begin{sphinxVerbatim}[commandchars=\\\{\}]
\PYG{k}{def} \PYG{n+nf}{eval\PYGZus{}polynome\PYGZus{}troisieme\PYGZus{}degres}\PYG{p}{(}\PYG{n}{x}\PYG{p}{,} \PYG{n}{a\PYGZus{}0}\PYG{p}{,} \PYG{n}{a\PYGZus{}1}\PYG{p}{,} \PYG{n}{a\PYGZus{}2}\PYG{p}{,} \PYG{n}{a\PYGZus{}3}\PYG{p}{)}\PYG{p}{:}
    \PYG{k}{return} \PYG{n}{a\PYGZus{}0} \PYG{o}{+} \PYG{n}{a\PYGZus{}1}\PYG{o}{*}\PYG{n}{x} \PYG{o}{+} \PYG{n}{a\PYGZus{}2}\PYG{o}{*}\PYG{n}{x}\PYG{o}{*}\PYG{o}{*}\PYG{l+m+mi}{2} \PYG{o}{+} \PYG{n}{a\PYGZus{}3}\PYG{o}{*}\PYG{n}{x}\PYG{o}{*}\PYG{o}{*}\PYG{l+m+mi}{3}

\PYG{k}{def} \PYG{n+nf}{eval\PYGZus{}polynome}\PYG{p}{(}\PYG{n}{x}\PYG{p}{,} \PYG{n}{a\PYGZus{}0}\PYG{o}{=}\PYG{l+m+mi}{0}\PYG{p}{,} \PYG{n}{a\PYGZus{}1}\PYG{o}{=}\PYG{l+m+mi}{0}\PYG{p}{,} \PYG{n}{a\PYGZus{}2}\PYG{o}{=}\PYG{l+m+mi}{0}\PYG{p}{,} \PYG{n}{a\PYGZus{}3}\PYG{o}{=}\PYG{l+m+mi}{0}\PYG{p}{,} \PYG{n}{a\PYGZus{}4}\PYG{o}{=}\PYG{l+m+mi}{0}\PYG{p}{)}\PYG{p}{:}
    \PYG{k}{return} \PYG{n}{a\PYGZus{}0} \PYG{o}{+} \PYG{n}{a\PYGZus{}1}\PYG{o}{*}\PYG{n}{x} \PYG{o}{+} \PYG{n}{a\PYGZus{}2}\PYG{o}{*}\PYG{n}{x}\PYG{o}{*}\PYG{o}{*}\PYG{l+m+mi}{2} \PYG{o}{+} \PYG{n}{a\PYGZus{}3}\PYG{o}{*}\PYG{n}{x}\PYG{o}{*}\PYG{o}{*}\PYG{l+m+mi}{3} \PYG{o}{+} \PYG{n}{a\PYGZus{}4}\PYG{o}{*}\PYG{o}{*}\PYG{l+m+mi}{4}

\PYG{k}{def} \PYG{n+nf}{eval\PYGZus{}polynome}\PYG{p}{(}\PYG{n}{x}\PYG{p}{,} \PYG{o}{*}\PYG{n}{a}\PYG{p}{)}\PYG{p}{:}
    \PYG{n}{output} \PYG{o}{=} \PYG{l+m+mi}{0}
    \PYG{k}{for} \PYG{n}{i}\PYG{p}{,} \PYG{n}{val\PYGZus{}a} \PYG{o+ow}{in} \PYG{n+nb}{enumerate}\PYG{p}{(}\PYG{n}{a}\PYG{p}{)}\PYG{p}{:}
        \PYG{n}{output} \PYG{o}{+}\PYG{o}{=} \PYG{n}{val\PYGZus{}a}\PYG{o}{*}\PYG{n}{x}\PYG{o}{*}\PYG{o}{*}\PYG{n}{i}
    \PYG{k}{return} \PYG{n}{output}
   
\PYG{n}{x} \PYG{o}{=} \PYG{l+m+mi}{10}
\PYG{n}{eval\PYGZus{}polynome}\PYG{p}{(}\PYG{n}{x}\PYG{p}{,} \PYG{l+m+mi}{3}\PYG{p}{,} \PYG{l+m+mi}{4}\PYG{p}{)}
\end{sphinxVerbatim}

\begin{sphinxVerbatim}[commandchars=\\\{\}]
43
\end{sphinxVerbatim}


\subsubsection{Equation du second degré}
\label{\detokenize{cours1_fonctions_corr_exercices:equation-du-second-degre}}
\sphinxAtStartPar
On souhaite trouver les solutions de l’équation : \(ax^2 + bx + c=0\). On rappelle que pour cela on calcule le discriminant \(\Delta = b^2 - 4*a*c\).
\begin{itemize}
\item {} 
\sphinxAtStartPar
Si \(\Delta>0\), les solutions sont \(\frac{-b \pm \sqrt{Delta}}{2a}\)

\item {} 
\sphinxAtStartPar
Si \(\Delta=0\), il y a une solution (double) \(\frac{-b }{2a}\)

\item {} 
\sphinxAtStartPar
Si \(\Delta<0\), les solutions sont \(\frac{-b \pm i\sqrt{Delta}}{2a}\)

\end{itemize}
\begin{enumerate}
\sphinxsetlistlabels{\arabic}{enumi}{enumii}{}{.}%
\item {} 
\sphinxAtStartPar
Ecrire une fonction equation\_second\_degre qui donne les solutions.

\item {} 
\sphinxAtStartPar
Tester l’équation \(x^2 -3x + 2\)

\item {} 
\sphinxAtStartPar
Ecrire une fonction qui résout une équation du premier degré.

\item {} 
\sphinxAtStartPar
Faire en sorte que cette fonction soit appelée lorsque \(a=0\).

\end{enumerate}

\begin{sphinxVerbatim}[commandchars=\\\{\}]
\PYG{k+kn}{from} \PYG{n+nn}{math} \PYG{k+kn}{import} \PYG{n}{sqrt}

\PYG{k}{def} \PYG{n+nf}{equation\PYGZus{}seconde\PYGZus{}degre}\PYG{p}{(}\PYG{n}{a}\PYG{p}{,} \PYG{n}{b}\PYG{p}{,} \PYG{n}{c}\PYG{p}{)}\PYG{p}{:}
    \PYG{n}{Delta} \PYG{o}{=} \PYG{n}{b}\PYG{o}{*}\PYG{o}{*}\PYG{l+m+mi}{2} \PYG{o}{\PYGZhy{}} \PYG{l+m+mi}{4}\PYG{o}{*}\PYG{n}{a}\PYG{o}{*}\PYG{n}{c}
    \PYG{k}{if} \PYG{n}{Delta}\PYG{o}{\PYGZgt{}}\PYG{l+m+mi}{0}\PYG{p}{:}
        \PYG{k}{return} \PYG{p}{(}\PYG{o}{\PYGZhy{}}\PYG{n}{b} \PYG{o}{+} \PYG{n}{sqrt}\PYG{p}{(}\PYG{n}{Delta}\PYG{p}{)}\PYG{p}{)}\PYG{o}{/}\PYG{p}{(}\PYG{l+m+mi}{2}\PYG{o}{*}\PYG{n}{a}\PYG{p}{)}\PYG{p}{,} \PYG{p}{(}\PYG{o}{\PYGZhy{}}\PYG{n}{b} \PYG{o}{\PYGZhy{}} \PYG{n}{sqrt}\PYG{p}{(}\PYG{n}{Delta}\PYG{p}{)}\PYG{p}{)}\PYG{o}{/}\PYG{p}{(}\PYG{l+m+mi}{2}\PYG{o}{*}\PYG{n}{a}\PYG{p}{)}
    \PYG{k}{if} \PYG{n}{Delta}\PYG{o}{==}\PYG{l+m+mi}{0}\PYG{p}{:}
        \PYG{k}{return} \PYG{o}{\PYGZhy{}}\PYG{n}{b}\PYG{o}{/}\PYG{p}{(}\PYG{l+m+mi}{2}\PYG{o}{*}\PYG{n}{a}\PYG{p}{)}
    \PYG{k}{if} \PYG{n}{Delta}\PYG{o}{\PYGZlt{}}\PYG{l+m+mi}{0}\PYG{p}{:}
        \PYG{k}{return} \PYG{p}{(}\PYG{o}{\PYGZhy{}}\PYG{n}{b} \PYG{o}{+} \PYG{n}{sqrt}\PYG{p}{(}\PYG{o}{\PYGZhy{}}\PYG{n}{Delta}\PYG{p}{)}\PYG{o}{*}\PYG{l+m+mi}{1}\PYG{n}{J}\PYG{p}{)}\PYG{o}{/}\PYG{p}{(}\PYG{l+m+mi}{2}\PYG{o}{*}\PYG{n}{a}\PYG{p}{)}\PYG{p}{,} \PYG{p}{(}\PYG{o}{\PYGZhy{}}\PYG{n}{b} \PYG{o}{\PYGZhy{}} \PYG{n}{sqrt}\PYG{p}{(}\PYG{o}{\PYGZhy{}}\PYG{n}{Delta}\PYG{p}{)}\PYG{o}{*}\PYG{l+m+mi}{1}\PYG{n}{J}\PYG{p}{)}\PYG{o}{/}\PYG{p}{(}\PYG{l+m+mi}{2}\PYG{o}{*}\PYG{n}{a}\PYG{p}{)}
    
\PYG{n+nb}{print}\PYG{p}{(}\PYG{n}{equation\PYGZus{}seconde\PYGZus{}degre}\PYG{p}{(}\PYG{l+m+mi}{1}\PYG{p}{,} \PYG{o}{\PYGZhy{}}\PYG{l+m+mi}{3}\PYG{p}{,} \PYG{l+m+mi}{2}\PYG{p}{)}\PYG{p}{)}

\PYG{k}{def} \PYG{n+nf}{equation\PYGZus{}premier\PYGZus{}degre}\PYG{p}{(}\PYG{n}{a}\PYG{p}{,} \PYG{n}{b}\PYG{p}{)}\PYG{p}{:}
    \PYG{k}{return} \PYG{o}{\PYGZhy{}}\PYG{n}{b}\PYG{o}{/}\PYG{n}{a}

\PYG{k}{def} \PYG{n+nf}{equation\PYGZus{}seconde\PYGZus{}degre}\PYG{p}{(}\PYG{n}{a}\PYG{p}{,} \PYG{n}{b}\PYG{p}{,} \PYG{n}{c}\PYG{p}{)}\PYG{p}{:}
    \PYG{k}{if} \PYG{n}{a}\PYG{o}{==}\PYG{l+m+mi}{0}\PYG{p}{:}
        \PYG{k}{return} \PYG{n}{equation\PYGZus{}premier\PYGZus{}degra}\PYG{p}{(}\PYG{n}{b}\PYG{p}{,} \PYG{n}{c}\PYG{p}{)}
    \PYG{n}{Delta} \PYG{o}{=} \PYG{n}{b}\PYG{o}{*}\PYG{o}{*}\PYG{l+m+mi}{2} \PYG{o}{\PYGZhy{}} \PYG{l+m+mi}{4}\PYG{o}{*}\PYG{n}{a}\PYG{o}{*}\PYG{n}{c}
    \PYG{k}{if} \PYG{n}{Delta}\PYG{o}{\PYGZgt{}}\PYG{l+m+mi}{0}\PYG{p}{:}
        \PYG{k}{return} \PYG{p}{(}\PYG{o}{\PYGZhy{}}\PYG{n}{b} \PYG{o}{+} \PYG{n}{sqrt}\PYG{p}{(}\PYG{n}{Delta}\PYG{p}{)}\PYG{p}{)}\PYG{o}{/}\PYG{p}{(}\PYG{l+m+mi}{2}\PYG{o}{*}\PYG{n}{a}\PYG{p}{)}\PYG{p}{,} \PYG{p}{(}\PYG{o}{\PYGZhy{}}\PYG{n}{b} \PYG{o}{\PYGZhy{}} \PYG{n}{sqrt}\PYG{p}{(}\PYG{n}{Delta}\PYG{p}{)}\PYG{p}{)}\PYG{o}{/}\PYG{p}{(}\PYG{l+m+mi}{2}\PYG{o}{*}\PYG{n}{a}\PYG{p}{)}
    \PYG{k}{if} \PYG{n}{Delta}\PYG{o}{==}\PYG{l+m+mi}{0}\PYG{p}{:}
        \PYG{k}{return} \PYG{o}{\PYGZhy{}}\PYG{n}{b}\PYG{o}{/}\PYG{p}{(}\PYG{l+m+mi}{2}\PYG{o}{*}\PYG{n}{a}\PYG{p}{)}
    \PYG{k}{if} \PYG{n}{Delta}\PYG{o}{\PYGZlt{}}\PYG{l+m+mi}{0}\PYG{p}{:}
        \PYG{k}{return} \PYG{p}{(}\PYG{o}{\PYGZhy{}}\PYG{n}{b} \PYG{o}{+} \PYG{n}{sqrt}\PYG{p}{(}\PYG{o}{\PYGZhy{}}\PYG{n}{Delta}\PYG{p}{)}\PYG{o}{*}\PYG{l+m+mi}{1}\PYG{n}{J}\PYG{p}{)}\PYG{o}{/}\PYG{p}{(}\PYG{l+m+mi}{2}\PYG{o}{*}\PYG{n}{a}\PYG{p}{)}\PYG{p}{,} \PYG{p}{(}\PYG{o}{\PYGZhy{}}\PYG{n}{b} \PYG{o}{\PYGZhy{}} \PYG{n}{sqrt}\PYG{p}{(}\PYG{o}{\PYGZhy{}}\PYG{n}{Delta}\PYG{p}{)}\PYG{o}{*}\PYG{l+m+mi}{1}\PYG{n}{J}\PYG{p}{)}\PYG{o}{/}\PYG{p}{(}\PYG{l+m+mi}{2}\PYG{o}{*}\PYG{n}{a}\PYG{p}{)}

\PYG{c+c1}{\PYGZsh{} Dans cet exercice, les else ou elif sont inutiles.}
\end{sphinxVerbatim}

\begin{sphinxVerbatim}[commandchars=\\\{\}]
(2.0, 1.0)
\end{sphinxVerbatim}


\subsubsection{Coordonnée polaire d’un nombre complexe}
\label{\detokenize{cours1_fonctions_corr_exercices:coordonnee-polaire-d-un-nombre-complexe}}
\sphinxAtStartPar
On considère une nombre complexe \(z\) et sa représentation polaire : \(z = re^{i\theta}\)
\begin{enumerate}
\sphinxsetlistlabels{\arabic}{enumi}{enumii}{}{.}%
\item {} 
\sphinxAtStartPar
Ecrire une fonction qui à partir de \(r\) et \(\theta\) renvoie \(z\)

\end{enumerate}

\begin{sphinxVerbatim}[commandchars=\\\{\}]
\PYG{k+kn}{from} \PYG{n+nn}{math} \PYG{k+kn}{import} \PYG{n}{cos}\PYG{p}{,} \PYG{n}{sin}\PYG{p}{,} \PYG{n}{pi}

\PYG{k}{def} \PYG{n+nf}{from\PYGZus{}polar}\PYG{p}{(}\PYG{n}{r}\PYG{p}{,} \PYG{n}{theta}\PYG{p}{)}\PYG{p}{:}
    \PYG{k}{return} \PYG{n}{r}\PYG{o}{*}\PYG{n}{cos}\PYG{p}{(}\PYG{n}{theta}\PYG{p}{)} \PYG{o}{+} \PYG{l+m+mi}{1}\PYG{n}{J}\PYG{o}{*} \PYG{n}{r}\PYG{o}{*}\PYG{n}{sin}\PYG{p}{(}\PYG{n}{theta}\PYG{p}{)}
    
\PYG{n}{from\PYGZus{}polar}\PYG{p}{(}\PYG{l+m+mi}{1}\PYG{p}{,} \PYG{n}{pi}\PYG{o}{/}\PYG{l+m+mi}{2}\PYG{p}{)}
\end{sphinxVerbatim}

\begin{sphinxVerbatim}[commandchars=\\\{\}]
(6.123233995736766e\PYGZhy{}17+1j)
\end{sphinxVerbatim}


\subsubsection{Nombres premiers}
\label{\detokenize{cours1_fonctions_corr_exercices:nombres-premiers}}
\sphinxAtStartPar
Ecrire une fonction qui renvoie True si un nombre est premier et False sinon. Pour cela, on testera si le nombre
est divisible par les entiers successifs à partir de 2.

\sphinxAtStartPar
On admet que si \(n\) n’est pas premier, alors il existe un entier \(p \ge 2\) tel que \(p^2 \le n\) qui est un diviseur de \(n\).
\begin{enumerate}
\sphinxsetlistlabels{\arabic}{enumi}{enumii}{}{.}%
\item {} 
\sphinxAtStartPar
Ecrire la fonction à l’aide d’une boucle.

\item {} 
\sphinxAtStartPar
Combien y a t\sphinxhyphen{}il d’années au XXIème siècle qui sont des nombres premiers ?

\item {} 
\sphinxAtStartPar
Si vous n’êtes pas convaincu que c’est mieux avec des fonctions, faites le sans…

\end{enumerate}

\begin{sphinxVerbatim}[commandchars=\\\{\}]
\PYG{k}{def} \PYG{n+nf}{est\PYGZus{}premier}\PYG{p}{(}\PYG{n}{n}\PYG{p}{)}\PYG{p}{:}
    \PYG{n}{p} \PYG{o}{=} \PYG{l+m+mi}{2}
    \PYG{k}{while} \PYG{n}{p}\PYG{o}{*}\PYG{o}{*}\PYG{l+m+mi}{2}\PYG{o}{\PYGZlt{}}\PYG{o}{=}\PYG{n}{n}\PYG{p}{:}
        \PYG{k}{if} \PYG{n}{n}\PYG{o}{\PYGZpc{}}\PYG{k}{p}==0:
            \PYG{k}{return} \PYG{k+kc}{False}
        \PYG{n}{p} \PYG{o}{+}\PYG{o}{=} \PYG{l+m+mi}{1}
    \PYG{k}{return} \PYG{k+kc}{True}

\PYG{n}{est\PYGZus{}premier}\PYG{p}{(}\PYG{l+m+mi}{7}\PYG{p}{)}

\PYG{n}{count\PYGZus{}premier} \PYG{o}{=} \PYG{l+m+mi}{0}
\PYG{k}{for} \PYG{n}{annee} \PYG{o+ow}{in} \PYG{n+nb}{range}\PYG{p}{(}\PYG{l+m+mi}{2001}\PYG{p}{,} \PYG{l+m+mi}{2101}\PYG{p}{)}\PYG{p}{:}
    \PYG{k}{if} \PYG{n}{est\PYGZus{}premier}\PYG{p}{(}\PYG{n}{annee}\PYG{p}{)}\PYG{p}{:}
        \PYG{n}{count\PYGZus{}premier} \PYG{o}{+}\PYG{o}{=} \PYG{l+m+mi}{1}
\PYG{n+nb}{print}\PYG{p}{(}\PYG{l+s+sa}{f}\PYG{l+s+s1}{\PYGZsq{}}\PYG{l+s+s1}{Il y a }\PYG{l+s+si}{\PYGZob{}}\PYG{n}{count\PYGZus{}premier}\PYG{l+s+si}{\PYGZcb{}}\PYG{l+s+s1}{ années premières aux XXI siècle}\PYG{l+s+s1}{\PYGZsq{}}\PYG{p}{)}
\end{sphinxVerbatim}


\subsubsection{Mention}
\label{\detokenize{cours1_fonctions_corr_exercices:mention}}
\sphinxAtStartPar
Ecrire une fonction qui à partir de la note sur 20 donne la mention

\begin{sphinxVerbatim}[commandchars=\\\{\}]
\PYG{k}{def} \PYG{n+nf}{mention}\PYG{p}{(}\PYG{n}{note}\PYG{p}{)}\PYG{p}{:}
    \PYG{k}{if} \PYG{n}{note}\PYG{o}{\PYGZgt{}}\PYG{o}{=}\PYG{l+m+mi}{18}\PYG{p}{:}
        \PYG{k}{return} \PYG{l+s+s1}{\PYGZsq{}}\PYG{l+s+s1}{Felicitation}\PYG{l+s+s1}{\PYGZsq{}}
    \PYG{k}{if} \PYG{n}{note}\PYG{o}{\PYGZgt{}}\PYG{o}{=}\PYG{l+m+mi}{16}\PYG{p}{:}
        \PYG{k}{return} \PYG{l+s+s1}{\PYGZsq{}}\PYG{l+s+s1}{Très bien}\PYG{l+s+s1}{\PYGZsq{}}
    \PYG{k}{if} \PYG{n}{note}\PYG{o}{\PYGZgt{}}\PYG{o}{=}\PYG{l+m+mi}{14}\PYG{p}{:}
        \PYG{k}{return} \PYG{l+s+s2}{\PYGZdq{}}\PYG{l+s+s2}{Bien}\PYG{l+s+s2}{\PYGZdq{}}
    \PYG{k}{if} \PYG{n}{note}\PYG{o}{\PYGZgt{}}\PYG{o}{=}\PYG{l+m+mi}{12}\PYG{p}{:}
        \PYG{k}{return} \PYG{l+s+s2}{\PYGZdq{}}\PYG{l+s+s2}{Assez bien}\PYG{l+s+s2}{\PYGZdq{}}
    \PYG{k}{return} \PYG{l+s+s1}{\PYGZsq{}}\PYG{l+s+s1}{Pas de mention}\PYG{l+s+s1}{\PYGZsq{}}
\end{sphinxVerbatim}


\subsection{Exercices sur les nombres}
\label{\detokenize{cours2_nombres_corr_exercices:exercices-sur-les-nombres}}\label{\detokenize{cours2_nombres_corr_exercices::doc}}

\subsubsection{Fonctions mathématiques}
\label{\detokenize{cours2_nombres_corr_exercices:fonctions-mathematiques}}\begin{itemize}
\item {} 
\sphinxAtStartPar
Est ce que la fonction log est le logarithme décimal ou népérien ?

\item {} 
\sphinxAtStartPar
Calculer \(x = \sqrt{2}\) puis calculer \(x^2\). Que se passe\sphinxhyphen{}t\sphinxhyphen{}il ?

\item {} 
\sphinxAtStartPar
Calculer \(\arccos{\frac{\sqrt{2}}{2}}\) et comparer à sa valeur théorique.

\end{itemize}

\begin{sphinxVerbatim}[commandchars=\\\{\}]
\PYG{k+kn}{from} \PYG{n+nn}{math} \PYG{k+kn}{import} \PYG{n}{log}

\PYG{n+nb}{print}\PYG{p}{(}\PYG{n}{log}\PYG{p}{(}\PYG{l+m+mi}{10}\PYG{p}{)}\PYG{p}{)}
\end{sphinxVerbatim}

\begin{sphinxVerbatim}[commandchars=\\\{\}]
2.302585092994046
\end{sphinxVerbatim}

\begin{sphinxVerbatim}[commandchars=\\\{\}]
\PYG{k+kn}{from} \PYG{n+nn}{math} \PYG{k+kn}{import} \PYG{n}{sqrt}
\PYG{n}{sqrt}\PYG{p}{(}\PYG{l+m+mi}{2}\PYG{p}{)}\PYG{o}{*}\PYG{o}{*}\PYG{l+m+mi}{2}
\end{sphinxVerbatim}

\begin{sphinxVerbatim}[commandchars=\\\{\}]
2.0000000000000004
\end{sphinxVerbatim}

\begin{sphinxVerbatim}[commandchars=\\\{\}]
\PYG{k+kn}{from} \PYG{n+nn}{math} \PYG{k+kn}{import} \PYG{n}{acos}\PYG{p}{,} \PYG{n}{pi}

\PYG{n+nb}{print}\PYG{p}{(}\PYG{n}{acos}\PYG{p}{(}\PYG{n}{sqrt}\PYG{p}{(}\PYG{l+m+mi}{2}\PYG{p}{)}\PYG{o}{/}\PYG{l+m+mi}{2}\PYG{p}{)}\PYG{p}{)}
\PYG{n+nb}{print}\PYG{p}{(}\PYG{n}{pi}\PYG{o}{/}\PYG{l+m+mi}{4}\PYG{p}{)}
\end{sphinxVerbatim}

\begin{sphinxVerbatim}[commandchars=\\\{\}]
0.7853981633974483
0.7853981633974483
\end{sphinxVerbatim}


\subsubsection{Constante de structure fine}
\label{\detokenize{cours2_nombres_corr_exercices:constante-de-structure-fine}}
\sphinxAtStartPar
La constante de structure fine est définie en physique comme étant égale à
\begin{equation*}
\begin{split} 
\alpha = \frac{e^2}{2\epsilon_0 h c}
\end{split}
\end{equation*}
\sphinxAtStartPar
où
\begin{itemize}
\item {} 
\sphinxAtStartPar
\(e\) est la charge de l’électron et vaut \(1.602176634 \times 10^{-19} C\)

\item {} 
\sphinxAtStartPar
\(h\) est la constante de Planck et vaut \(6.626\,070\,15 \times 10^{-34} J s\)

\item {} 
\sphinxAtStartPar
\(\epsilon_0\) la permitivité du vide et vaut \(8.8541878128 \times 10^{-12} F/m\)

\item {} 
\sphinxAtStartPar
\(c\) la célérité de la lumière dans le vide, \(c=299792458 m/s\)

\end{itemize}

\sphinxAtStartPar
Définissez en Python les variables \sphinxcode{\sphinxupquote{e}}, \sphinxcode{\sphinxupquote{hbar}}, \sphinxcode{\sphinxupquote{epsilon\_0}} et \sphinxcode{\sphinxupquote{c}}. Calculez \(\alpha\) et \(1/\alpha\)

\begin{sphinxVerbatim}[commandchars=\\\{\}]
\PYG{k+kn}{from} \PYG{n+nn}{math} \PYG{k+kn}{import} \PYG{n}{pi}

\PYG{n}{e} \PYG{o}{=} \PYG{l+m+mf}{1.602176634E\PYGZhy{}19}
\PYG{n}{h} \PYG{o}{=} \PYG{l+m+mf}{6.62607015E\PYGZhy{}34}
\PYG{n}{epsilon\PYGZus{}0} \PYG{o}{=} \PYG{l+m+mf}{8.8541878128E\PYGZhy{}12}
\PYG{n}{c} \PYG{o}{=} \PYG{l+m+mi}{299792458}


\PYG{n}{alpha} \PYG{o}{=} \PYG{n}{e}\PYG{o}{*}\PYG{o}{*}\PYG{l+m+mi}{2}\PYG{o}{/}\PYG{p}{(}\PYG{l+m+mi}{2}\PYG{o}{*}\PYG{n}{epsilon\PYGZus{}0}\PYG{o}{*}\PYG{n}{h}\PYG{o}{*}\PYG{n}{c}\PYG{p}{)}
\PYG{n+nb}{print}\PYG{p}{(}\PYG{l+m+mi}{1}\PYG{o}{/}\PYG{n}{alpha}\PYG{p}{)}
              
\end{sphinxVerbatim}

\begin{sphinxVerbatim}[commandchars=\\\{\}]
137.0359990841083
\end{sphinxVerbatim}


\subsubsection{Précision des nombres}
\label{\detokenize{cours2_nombres_corr_exercices:precision-des-nombres}}\begin{itemize}
\item {} 
\sphinxAtStartPar
Soit \(x=1\) et \(\epsilon = 10^{-15}\). Calculez \(y=x + \epsilon\) et ensuite \(y - x\).

\item {} 
\sphinxAtStartPar
Pourquoi le résultat est différent de \(10^{-15}\).

\item {} 
\sphinxAtStartPar
Que vaut cette valeur ?

\end{itemize}

\begin{sphinxVerbatim}[commandchars=\\\{\}]
\PYG{n}{x} \PYG{o}{=} \PYG{l+m+mi}{1}
\PYG{n}{epsilon} \PYG{o}{=} \PYG{l+m+mf}{1E\PYGZhy{}15}
\PYG{n}{y} \PYG{o}{=} \PYG{n}{x} \PYG{o}{+} \PYG{n}{epsilon}
\PYG{n+nb}{print}\PYG{p}{(}\PYG{n}{y} \PYG{o}{\PYGZhy{}} \PYG{n}{x}\PYG{p}{)}
\end{sphinxVerbatim}

\begin{sphinxVerbatim}[commandchars=\\\{\}]
1.1102230246251565e\PYGZhy{}15
\end{sphinxVerbatim}

\begin{sphinxVerbatim}[commandchars=\\\{\}]
\PYG{n+nb}{print}\PYG{p}{(}\PYG{l+m+mi}{5}\PYG{o}{*}\PYG{l+m+mi}{2}\PYG{o}{*}\PYG{o}{*}\PYG{o}{\PYGZhy{}}\PYG{l+m+mi}{52}\PYG{p}{)}
\end{sphinxVerbatim}

\begin{sphinxVerbatim}[commandchars=\\\{\}]
1.1102230246251565e\PYGZhy{}15
\end{sphinxVerbatim}


\subsubsection{Calcul d’une dérivée}
\label{\detokenize{cours2_nombres_corr_exercices:calcul-d-une-derivee}}
\sphinxAtStartPar
On considère une fonction \(f(x)\). On rappelle que la dérivée peut se définir comme
\begin{equation*}
\begin{split}
f^\prime(x) = \lim_{\epsilon\rightarrow0}\frac{f(x+\epsilon) - f(x)}{\epsilon}
\end{split}
\end{equation*}
\sphinxAtStartPar
Pour calculer numériquement une dérivée, il faut évaluer la limite en prenant une valeur ‘petite’ de \(\epsilon\).

\sphinxAtStartPar
On prendra comme exemple \(f(x) = \sin(x)\).
\begin{itemize}
\item {} 
\sphinxAtStartPar
Calculer numériquement la dérivée de \(f\) en \(\pi/4\) en utilisant la formule pour \(\epsilon = 10^{-6}\).

\item {} 
\sphinxAtStartPar
Comparer à la valeur théorique \(\cos(x)\) pour différentes valeurs de \(\epsilon\) que l’on prendra comme puissance de 10 (\(\epsilon = 10^{-n}\)). Que se passe\sphinxhyphen{}t\sphinxhyphen{}il si \(\epsilon\) est trop petit ? trop grand ?

\item {} 
\sphinxAtStartPar
Ecrire la fonction \sphinxcode{\sphinxupquote{sin\_prime(x, epsilon)}} qui calcule la dérivée de sin en \(x\)

\item {} 
\sphinxAtStartPar
Ecrire une fonction qui prend une fonction quelconque et renvoie la fonction dérivée.

\end{itemize}

\begin{sphinxVerbatim}[commandchars=\\\{\}]
\PYG{k+kn}{from} \PYG{n+nn}{math} \PYG{k+kn}{import} \PYG{n}{sin}\PYG{p}{,} \PYG{n}{pi}\PYG{p}{,} \PYG{n}{cos}

\PYG{n}{x} \PYG{o}{=} \PYG{n}{pi}\PYG{o}{/}\PYG{l+m+mi}{4}
\PYG{n}{epsilon} \PYG{o}{=} \PYG{l+m+mf}{1E\PYGZhy{}11}

\PYG{n}{d} \PYG{o}{=} \PYG{p}{(}\PYG{n}{sin}\PYG{p}{(}\PYG{n}{x} \PYG{o}{+} \PYG{n}{epsilon}\PYG{p}{)} \PYG{o}{\PYGZhy{}} \PYG{n}{sin}\PYG{p}{(}\PYG{n}{x}\PYG{p}{)}\PYG{p}{)}\PYG{o}{/}\PYG{n}{epsilon} \PYG{o}{\PYGZhy{}} \PYG{n}{cos}\PYG{p}{(}\PYG{n}{x}\PYG{p}{)}
\PYG{n+nb}{print}\PYG{p}{(}\PYG{n}{d}\PYG{p}{)}

\PYG{k}{for} \PYG{n}{n} \PYG{o+ow}{in} \PYG{n+nb}{range}\PYG{p}{(}\PYG{l+m+mi}{1}\PYG{p}{,} \PYG{l+m+mi}{15}\PYG{p}{)}\PYG{p}{:}
    \PYG{n}{epsilon} \PYG{o}{=} \PYG{l+m+mi}{10}\PYG{o}{*}\PYG{o}{*}\PYG{p}{(}\PYG{o}{\PYGZhy{}}\PYG{n}{n}\PYG{p}{)}
    \PYG{n}{d} \PYG{o}{=} \PYG{p}{(}\PYG{n}{sin}\PYG{p}{(}\PYG{n}{x} \PYG{o}{+} \PYG{n}{epsilon}\PYG{p}{)} \PYG{o}{\PYGZhy{}} \PYG{n}{sin}\PYG{p}{(}\PYG{n}{x}\PYG{p}{)}\PYG{p}{)}\PYG{o}{/}\PYG{n}{epsilon} \PYG{o}{\PYGZhy{}} \PYG{n}{cos}\PYG{p}{(}\PYG{n}{x}\PYG{p}{)}
    \PYG{n+nb}{print}\PYG{p}{(}\PYG{l+s+s1}{\PYGZsq{}}\PYG{l+s+s1}{n=}\PYG{l+s+s1}{\PYGZsq{}}\PYG{p}{,} \PYG{n}{n}\PYG{p}{,} \PYG{n}{d}\PYG{p}{)}

\PYG{c+c1}{\PYGZsh{} Lorsque epsilon est trop petit, il y a des erreurs d\PYGZsq{}arrondi. Lorsque epsilon est trop grand, nous}
\PYG{c+c1}{\PYGZsh{} sommes loin de la limite}
\end{sphinxVerbatim}

\begin{sphinxVerbatim}[commandchars=\\\{\}]
5.365427460879424e\PYGZhy{}06
n= 1 \PYGZhy{}0.03650380828255784
n= 2 \PYGZhy{}0.0035472894973379576
n= 3 \PYGZhy{}0.0003536712121802177
n= 4 \PYGZhy{}3.535651724428934e\PYGZhy{}05
n= 5 \PYGZhy{}3.5355413900983734e\PYGZhy{}06
n= 6 \PYGZhy{}3.5344236126721995e\PYGZhy{}07
n= 7 \PYGZhy{}3.5807553921962665e\PYGZhy{}08
n= 8 3.050251939917814e\PYGZhy{}09
n= 9 3.635694267867251e\PYGZhy{}08
n= 10 9.245353623787977e\PYGZhy{}07
n= 11 5.365427460879424e\PYGZhy{}06
n= 12 \PYGZhy{}5.7368027853721415e\PYGZhy{}06
n= 13 0.00010528549967714351
n= 14 0.003435954573552613
\end{sphinxVerbatim}

\begin{sphinxVerbatim}[commandchars=\\\{\}]
\PYG{k}{def} \PYG{n+nf}{sin\PYGZus{}prime}\PYG{p}{(}\PYG{n}{x}\PYG{p}{,} \PYG{n}{epsilon}\PYG{p}{)}\PYG{p}{:}
    \PYG{k}{return} \PYG{p}{(}\PYG{n}{sin}\PYG{p}{(}\PYG{n}{x} \PYG{o}{+} \PYG{n}{epsilon}\PYG{p}{)} \PYG{o}{\PYGZhy{}} \PYG{n}{sin}\PYG{p}{(}\PYG{n}{x}\PYG{p}{)}\PYG{p}{)}\PYG{o}{/}\PYG{n}{epsilon}

\PYG{k}{def} \PYG{n+nf}{derivee}\PYG{p}{(}\PYG{n}{f}\PYG{p}{,} \PYG{n}{epsilon}\PYG{p}{)}\PYG{p}{:}
    \PYG{k}{def} \PYG{n+nf}{f\PYGZus{}prime}\PYG{p}{(}\PYG{n}{x}\PYG{p}{)}\PYG{p}{:}
        \PYG{k}{return} \PYG{p}{(}\PYG{n}{f}\PYG{p}{(}\PYG{n}{x}\PYG{o}{+}\PYG{n}{epsilon}\PYG{p}{)} \PYG{o}{\PYGZhy{}} \PYG{n}{f}\PYG{p}{(}\PYG{n}{x}\PYG{p}{)}\PYG{p}{)}\PYG{o}{/}\PYG{n}{epsilon}
    \PYG{k}{return} \PYG{n}{f\PYGZus{}prime}

\PYG{n}{derivee}\PYG{p}{(}\PYG{n}{sin}\PYG{p}{,} \PYG{n}{epsilon}\PYG{o}{=}\PYG{l+m+mf}{1E\PYGZhy{}8}\PYG{p}{)}\PYG{p}{(}\PYG{n}{x}\PYG{p}{)} \PYG{o}{\PYGZhy{}} \PYG{n}{cos}\PYG{p}{(}\PYG{n}{x}\PYG{p}{)}
\end{sphinxVerbatim}

\begin{sphinxVerbatim}[commandchars=\\\{\}]
3.050251939917814e\PYGZhy{}09
\end{sphinxVerbatim}


\subsubsection{Nombre complexe}
\label{\detokenize{cours2_nombres_corr_exercices:nombre-complexe}}\begin{itemize}
\item {} 
\sphinxAtStartPar
Ecrire une fonction qui calcule le module d’un nombre complexe \(z\)

\item {} 
\sphinxAtStartPar
Ecrire une fonction qui à partir de \(r\) et \(\theta\) renvoie le nombre \(z = re^{i\theta} = r\cos(\theta) + ir\sin(\theta)\)

\end{itemize}



\renewcommand{\indexname}{Index}
\printindex
\end{document}